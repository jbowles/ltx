
\section{Why Logic Matters to Linguistics}
This chapter contains brief historical information on many of the leading figures of logic, meta-mathematics, and formal systems in the last 150 years. It is not a complete a list and does not go into great detail of the specific contributions of these logicians and philosophers. There are many great sources that detail or collect the specific advances in logic and meta-mathematics. References are provided at the end of the book. In order to understand where theoretical linguistics comes from it is necessary (and interesting) to trace the logical lineage of the people who inadvertantly influenced the direction of the discipline. A thorough knowledge of the work and life of those discussed here is not necessary to appreciate the work of contemporary linguistics. It is'nt even necessary to \textsl{do} contemporary linguistic work, but it can be very helpful in fully appreciating potential directions of research---not to mention the need for and possibility of mathematicization of grammatical theories. On a last note, as a teacher of scientific writing I have always stressed to my students that the lives of research academics---physicists, biologists, mathematicians, and even linguists---do not fit into any stereotypical mold. Such a mold would force us to see the lives of such scientists as spent twittering away in libraries and labs. The converse seems almost always to be the case. Certainly, a lot of time is spent reading, learning, writing, and performing experiments. But science is also about having an active life and following your intuitions and passions---and enjoying what you do. As I tell my students, you have to enjoy what you do in order to get through all the boring literature and the hours spent learning particular formalisms. Nobody claims that these are exciting activities; what is exciting is what one can do after reading the literature and learning the formalisms. In other words, the life of a scientist is a rich life---both internally and externally. Besides providing a small tapestry of the lives (and names) of logicians who have influenced parts of linguistic research, I hope also to give a brief glimpse into the rich and varied life that the scientist and academic enjoys.

\section{Cantor: Infinity Comes in Different Sizes}
Infinity comes in different sizes! For a short while I taught courses on science writing and would often sneak into my lectures on argumentation surprising or interesting information. When I would tell students that infinity comes in different sizes and that there is a multitude of infinities (in fact, an infinite number of infinities) their reactions would range from disbelief to wonder. Rarely would a student show boredom upon receiving such information---probably becuase the idea of different sizes of infinity is so counter-intuitive. Nonetheless, the German mathematician Georg Cantor was able to show that this really is the case, though it arguably lead to his descent into madness.
Georg Cantor was born xxxxx

\section{Search for Boole}

\section{Frege}
Gottlob Frege is sometimes called the Father of Modern Logic. Ironically, the last sentence contains an example of one of Frege's major insights into language. We know that `Gottlob Frege' and `Father of Modern Logic' refer to (or name) the same human being (or object), but they each have a very different ``sense'' in which they \textsl{name} the same object.

\section{Carnap and the Logical Syntax}
xxxx student of Frege xxxxx

\section{Tell them G\"odel Sent You}
Kurt G\o"del has not been very influential in the world of linguistics, at least not directly. But he has privileged place in any list of influential logicians. His work is regarded as \textsl{the} most important in the twentieth-century. Some consider his work in logic second only to the work of Aristotle. Aristotle, of course, founded the study of logic by providing explicit definitions and schemata for the analysis of propositions. His logical systems remained virutally unchanged for 2,000 years. Nobody can replace Aristotle as the greatest logician, but only because he came first. When it comes to profound insight, however, G\"odel definitely has my vote for the greatest logician in all of world history. Most people would agree with this sentiment. Certainly, anyone with a reputation like this deserves to be mentioned, even if their work is not specially relevant to the topic at hand. Of course, you never know what the future has in store: G\o"del's work may find a relevancy it had not had before. Nonetheless, current linguistic science has not found much use for G\o"del.
G\"odel was infamously shy and reclusive. He died from complications due self-imposed starvation, the result of which was caused by his paranoia of being poisoned. He was born xxxxxxx 

\section{Tarski, Metamathematics, and Truth}
Alfred Teitelbaum was born January 14, 1901 in Warsaw Poland. By the time he was 34 years old, in 1935, he had changed his name to Tarski and established himself as one of rising stars in the world of logic. Certainly, at this time he was seen as one of the best, if not \textsl{the} best, logicians in Poland---which was one of the best places to study logic in the world. By 1935, Tarski had formed relationships, and in some cases given lectures to, many of the people who would also become world famous for their work in logic: Kurt G\"odel, Rudolf Carnap, W. V. O. Quine, Karl Popper, and many others associated with the Vienna Circle and the movement of logical positivism. For an excellent background on Tarski see the biography by Anita Burdman Feferman and Solomon Feferman titled \textsl{Alfred Tarski: Life and Logic}.

Besides being one of the greatest logicians of the twentieth-century (most consider his contributions second only to G\"odel), Tarski is particularly important to the early development of theoretical linguistics because of his theory of truth. I'm not aware of Tarski having any influence on syntactic theories, but he did have an impact in the field of semantics. His two most direct influences on natural language semantics came through (i) his metamathematical definition of truth, and (ii) the work of one of his students, Richard Montague. I will discuss (i) here, and (ii) in the nest section.
 
\section{Montague}
xxxx Montague semantics, formal grammar xxx
xxxx student of Tarksi xxxx








