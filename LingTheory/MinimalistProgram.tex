

The term generative has come to be associated with a vibrant, if not controversial, movement in the latter half of twentieth century linguistics. If you do not agree with generative theories of linguistics, you must still study them at most university graduate programs. The movement has not only defined its own general methods and philosophy of science, but many different approaches have defined themselves specifically in opposition to generative approaches. Consequently, the movement has had a large influence on those invested in it \textsl{and} those who believe it is the wrong approach.   










