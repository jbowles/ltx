\documentclass{wpblogentry}
%\documentclass{article}

\title{Eng 2020: Presupposition}
\tags{Linguistics, Pragmatics, English}
\category{Linguistics}
\usepackage{paperp}
\usepackage{paperc}
\maketitle

\begin{document}
This is a test post for my English 2020 class.
%=============================================
\section{Some Context}
Presupposition is a far-reaching, often studied, and complex phenomenon in natural language. We are not concerned here with teasing out nuance and detail, nor are too concerned with defining and disambiguating terms (e.g., implicature, presuppostion, assumption, \emph{et cetera}). What we want to get out of this is a sense that simple examples can lead to complex probems very quickly, and most importantly, that much of what we read and write in an academic domain presupposes a wealth of information. Without understanding this last fact (or at least having an intuitive sense when crucial presuppositions are being made), evaluating and analyzing complex scholarly discourse would not go over very well.   

\section{Examples}

Here are some examples os sentences that contain more information than we are given directly. They are fairly straighforward and are also `classic' examples in the literature. 
\begin{exe}
\ex Jon regrets that he failed.
\begin{xlist}
\ex Jon failed.
\end{xlist}

\ex The liquid in this tank has either stopped fermenting or it has not yet begun to ferment.
\begin{xlist}
\ex In the past, the liquid either was or was not fermenting.
\end{xlist}

\ex Jeri is a baby, but he is quiet.
\begin{xlist}
\ex Baby's are not quiet
\end{xlist}

\ex 
\begin{xlist}
\ex
\end{xlist}

\ex
\begin{xlist}
\ex
\end{xlist}
\end{exe}

These examples also have information that we are not given directly, but they are made more complicated by the fact that key words or constituents in the sentences are ambiguous (or polysemous).
\begin{exe}
 \ex John kicked the bucket.
\begin{xlist}
    \ex Jon physically struck a bucket with his leg.
    \ex Jon died.
\end{xlist}

\ex I need to make my appointments.
\begin{xlist}
    \ex I need to schedule an appointment.
    \ex I need to go to an appointment (on time).
\end{xlist}

\ex I see the light.
\begin{xlist}
    \ex I have visual confirmation of a light source.
    \ex I understand.
\end{xlist}
\end{exe}

\end{document}
