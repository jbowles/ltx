%this does not work===========
\documentclass{wpblogentry}
%\documentclass{article}

\title{Example}
\tags{Linguistics, Pragmatics, English}
\category{Linguistics}


\begin{document}
This is a test post for my English 2020 class.\\
==============================================

SOME CONTEXT:
Presupposition is a far-reaching, often studied, and complex phenomenon in natural language. We are not concerned here with teasing out nuance and detail, nor are too concerned with defining and disambiguating terms (e.g., implicature, presuppostion, assumption, \emph{et cetera}). What we want to get out of this is a sense that simple examples can lead to complex probems very quickly, and most importantly, that much of what we read and write in an academic domain presupposes a wealth of information. Without understanding this last fact (or at least having an intuitive sense when crucial presuppositions are being made), evaluating and analyzing complex scholarly discourse would not go over very well.   

EXAMPLES:

Here are some examples os sentences that contain more information than we are given directly. They are fairly straighforward and are also `classic' examples in the literature. 

These are in the equation format, which is defined in the renderer:

\begin{equation}
\text{Jon regrets that he failed.}
\end{equation}



Now here are the ExPex examples for which I defined the environment for:
something for href:
\href={http://linguisticlogic.wordpress.com/}{Linguistic Logic}=.

and href another way:
\href={Linguistic Logic}{http://linguisticlogic.wordpress.com/}=.

Something for includegraphics:
\includegraphics=[width=0.9\textwidth]{plantsymm}=

And includegraphics another way:
\includegraphics={plantsymm}=

End of the post.
\end{document}
