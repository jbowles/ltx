Languages depend on what you want to do in the future. If you want to go into academia look at what people there do (besides requisite C/C++); if you want to do Artificial Intelligence, Natural Language Processing, Machine Learning, Machine Translation, etc...; if you want to do web development, commercial software, etc...; see what people in those fields use.

But the best advice I ever got was not to focus so much on any one language but learn a little bit about different kinds of languages. This is because things change rapidly and fads can fade quickly. For example, if you focus all your energy on PHP you might be popular now, but who knows in ten years.

BUT, if you focus on learning different kinds of languages then you will be familiar with various paradigms. And, given that most languages are influenced by other languages, you can branch out by association or pick up the basics of one language pretty quickly.

For example, object-oriented languages are all the rage (ex: Ruby, Python,...); scripting languages too. Learn a functional language (ex: Scheme (or the mothership: LISP)). Learn a declarative language (ex: Prolog). HAVE FUN. DO WHAT EXCITES YOU. But pay attention to what is practical and realistic for your goals. If you want to design websites for a living you probably shouldn't invest a lot of time in something like Prolog or BASIC, or ALGOL, or LISP.

Over a year ago I started the same journey and decided on Python because the Natural Language Toolkit was built for Python. I messed around with Prolog and Scheme and Lisp and other AI, NLP type languages (many of them "leagacy" languages that are still relevant, by which I mean they have been around long enough to have a legacy). Now, I am investing heavily in Ruby, Java, C++ because the jobs I want require them (also SQL, Ruby on Rails, JavaScript and others).

The other reason to learn different kinds of languages: I did an internship at a company that was developing its own scripting language based on ECMAScript. Say you had a programmer who had awesome technique in only one kind of language---and this language was nothing like ECMAScript. There would be a considerable learning curve. I put this example as a hypothetical because this company had awesome programmers: this means they were proficient in a variety of programming languages.

Peter Norvig (co-author of a great AI book and director of research at Google) has a wonderful article called "Learn Programming in Ten Years." (http://norvig.com/21-days.html)
