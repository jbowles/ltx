\documentclass{wpblogentry}

\title{What `on' and `by' are Telling Us}
    \tags{Pragmatics, Semantics, Syntax }
    \category{english}

    \begin{document}

Sharon's comments about the use of `on' and `by' have motivated me to look into this a bit more. Let's recap:

I was taling about fallacies and how they may occur either ``on purpose'' or ``on accident.'' To this sharon objected and argued that ``on accident'' is not correct. I did some hand-waving and directed us to an example about conventions of grammatical use. I gave some examples about the `who/whom' distinction and the `never end a sentence in a `preposition' rule. To be fair, these really have nothing to do with the `on/by' distinction. Here I will directly evaluate the issue at hand. But before I do I want to reinforce my original caveat:

Many grammatical `rules' are the collection of conventional attitudes about what is or is not right. These conventions are enforced by a ruling elite, or whoever happens to be in power. That is, if you have an English teacher ora boss who insists on a certain conventional usage---claiming that it is a rule---and attaches a serious consequence to the breaking of that you rule, my advice is to bite the bullet and follow the `convention.'

I suspect we will find the same to be true with the `on/by' distinction, but I didn't want to commit to anything in class because I have never really looked at this before.

Let us begin:

We can think about ``on purpose'' and ``by mistake'' as assigning different degress of causality and intention to the agent of action. That is, when I say that ``I spilled the milk on purpose'' and ``I spilled the milk by accident'' I am assigning some level of causation and intent to my actions through the use of the prepositions `on' and `by.' This is easily testable by substitution: ``I spilled the milk with purpose'' assigns a kind of ownership of intent and causation to the agent. Whereas, ``I spilled the milk with accident'' simply sounds wrong; meaning we don't typiclaly assign full ownership to accidents. We assing casusation but probably not intention to an accident, whereas non-accidents are assigned both. We can say that ``on'' behaves in a similar fashion to ``with'' in assigning intent but saying that you did someting on accident with intent and causation is weird; an accident is by default something you did not intend (or, at least, did not intend the result or consequence). Interpreted this way, Sharon is right: ``I spilled the milk on accident'' is wrong because I did not intend for milk to spill all over the floor, though I still caused it to happen. It should be ``I spilled the milk by accident.'' Just as you would not \emph{normally} say ``I got married by accident'' as weddings are not typically things one happens to do without intent (\emph{sans} the ``Las Vegas'' scenario).

But what about the phrase ``I spilled the milk by design'' Certainly, it is not strange to assign intent and causation to the agent. Also, ``I spilled the milk with design'' strikes me as slightly strange but the ascription of intent and causation are still on target. From here it would be nice to see what the grammar and style books are telling us.

I didn't find many hits in grammar or style books. It doesn't appear to be an issue many books have decided to cover. But based on my cursory look I would say that the overwhleming consensus is that ``on accident'' is wrong. 

That is not the end of the story though. Indinaia State University linguist Leslie Barratt  has published a study about this topic (\verb=http://www.inst.at/trans/16Nr/01_4/barratt16.htm=). Generally, it seems, the issue with ``on accident'' versus ``by accident'' relates to the `who/whom' distinction discussed in class in that both phenomena seem to reflect historical changes and sociolinguistic variation in the (United States) English language.

All languages change over time and geography. To the linguist this is niether good nor bad but simply a fact (people in the South talk different than people in the West; and people in 1300's or  1500's England talk differently than people in 2000's England; pick up Chaucer or Shakespeare if you don't believe me). The shift of a language has typically been interpreted by more established members of society as an unravelling of the language due to insipid usage from youngsters or rebels. And \emph{vice versa} by less established members of society: they see it as a restricting limit imposed by those in charge. Either you way you see it, language change is a strange natural phenomenon that integrates parents, children, culture, and the nature of inherent human linguistic capacity. Large-scale changes take effect with minimal concerted direction (e.g., there are not committees that consciously direct the changes in English); yet these changes seem to be fairly systematic and orderly. This is probably the case with our `on/by' example.

It is worth quoting Barratt's (2006:1) first paragraph in its entirety: ``This paper addresses a recent language change in U.S English. The surprising aspect of this change is that it went unnoticed while it occurred among younger speakers in different parts of the country while older speakers continued to use the older form. The change was one affecting replacement in only a small part of the lexicon: that of by accident to on accident. The paper will also show that while by accident is still more common overall than on accident, and while older speakers still use by accident and often do not accept on accident, among younger speakers of U.S. English, on accident is common and, in fact, has equal acceptance with by accident.''

This does not mean, however, that because you are > 35 you will accept both forms. It merely makes an observation about a statistical likelihood. Next I will propose a possible cause for this shift (however, what comes below is only a rough speculation).

Historical change often ocurrs through what is called \emph{reanalysis, extension,} and \emph{borrowing} (see (Harris and Campbell 1995)). In this case, Without getting into too much detail, it is likely that given the accepted and established ``on purpose'' younger speakers may have started to analogically extend this syntactic pattern to ``by accident,'' giving us ``on accident.'' The distribution of these two phrases is very similar and this can certainly contributes to the extension (e.g.; ``I did it \ldots'', where each clause can go where the ellipses are). In other words, this is a change based on anlogy to a similar pattern: ``by accident'' is like ``on accident'' syntactically (i.e., Preposition + Noun) and so younger speakers who are less sophisticated about subtle semantic differences between the two prepositions `on' and `by' simply extend the pattern to include `on.' However, there are problems with this explanation as Barratt (2006) observes.

An interesting question then, is how do speakers who use ``on accident'' interpret the meaning of `on.' There are many answers and all of them would require detailed analysis. I give two unanalyzed answers here: (i) they have reanalyzed `on' in this case to mean something different; (ii) there a pragamtic issues that help determine the distribution of when speakers will use `on;' for example, when I used it in class I was still thinking about the purposive ``on purpose'' and trying to decide if we should discuss whether or not one can tell if someone uses a fallacy by accident---in other words, I was distracted and used `on' instead of `by' because it is still acceptable in my dialect--- with the added result that there is some tinge of irony (or paradox) in saying ``on accident'' in such a way as to mean one did something on purpose but pretends it was by accident, ``accidentally on purpose.'' This last analysis has actually been proposed---see Barratt (2006) for a reference to this analysis.  


IS IT RIGHT OR WRONG?

And here is the rub: as so often happens in questions of grammar, everyone is right. That is, you use the language you grow up around and if you can communicate within your community you are not doing it wrong. However, you always have to be aware of the conventions outside of your community and if you do not know those conventions (and use them when context dictates) you could potentially pay a high price.

And as is usually the case in science, there is not definitive answer. Is it wrong to say ``on accident''? There is no definitive yes or no answer. People who write (and care about) grammar books tend to be older more established members of society who reflect a conservative grammar. That means you have to know the conventions even if it is unfair: that's the way it is and if you want access to the upper rungs of the socieconomic ladder then you better learn to talk like the people who belong. On the other hand, it isn't very fair that conservative language users have to listen to thier linguistic conventions become butured by unsophisticated speakers and linguistic rebels; but there's nothing they can do about it either. 


POSTSCRIPT

With the advent of digital archives and web-based search engines we could search for the strings ``on/by accident'', ``on/by purpose'', ``with purpose/accident.'' Collecting this data would easy; interpreting it, on the other hand, probably not so. The reason for this is because prepositional `on/by' have various interpretations. For example, I might title a report ``On Accident Claims in the Insurance Industry'' which means something like ``\emph{about} accident claim in the insurance industry.'' Nonetheless, some linguistics student is probably working on a paper about this right now. 


REFERENCES

Barratt, Leslie. 2006. ``What speakers don't notice: Language changes can sneak in,'' \emph{Innovation and continuity in language and communication of different language cultures}, ed. Rudolf Muhr.

Harris, Alice and Lyle Campbell. 1995. \emph{Historical-syntax in cross-linguistic perspective}. Cambridge, Cambridge University Press.  

\end{document}
