\documentclass[11pt]{article}
\usepackage{paperp}
\usepackage{paperc}
\usepackage{morphgloss}
\begin{document}
\title{Irish Stuff}

\author{Joshua Bowles}



\date{\today}

\maketitle


\section{Pronunciation}
%-------------------------------------------------------------
						\begin{tabular}{c}
\begin{tabular}{|l|}
\hline
{\bf Long Vowels}\\
\hline
\hline 
\textsl{slender v}:\\
\ipa{\`i} --- s\ipa{\`i} (she)\\
\ipa{\`e} --- m\ipa{\`e} I, me)\\
\textsl{broad v}:\\
\ipa{\`a} --- l\ipa{\`a} (day)\\
\ipa{\`o} --- b\ipa{\`o} (cow)\\
\ipa{\`u} --- t\ipa{\`u} (you)\\
\hline
\end{tabular}
\begin{tabular}{|l|}
\hline
{\bf Short Vowels}\\
\hline
\hline
\textsl{slender v}:\\
i --- sin (that)\\
e --- te (hot)\\
\textsl{broad v}:\\
a --- cas (turn)\\
o --- bog (soft)\\
u --- tugann (gives)\\
\hline
\end{tabular}
\begin{tabular}{|l|}
\hline
{\bf Broad `l'}\\
\hline
\hline
lu\ipa{\`i} (lying)\\
lae (of a day)\\
l\ipa{\`a} (day)\\
l\ipa{\`o}n (lunch)\\
L\ipa{\`u} (Louth County)\\
d\ipa{\`i}ol (selling)\\
b\ipa{\`e}al (mouth)\\
f\ipa{\`a}l (hedge)\\
\ipa{\`o}l (drinking)\\
c\ipa{\`u}l (back)\\
\hline
\end{tabular}
\begin{tabular}{|l|}
\hline
{\bf Slender `l'}\\
\hline
\hline
l\ipa{\`i}ne (line)\\
l\ipa{\`e}ine (shirt)\\
le\ipa{\`a} (melting)\\
le\ipa{\`o}n (lion)\\
li\ipa{\`u} (a shout)\\
s\ipa{\`i}l (think)\\
b\ipa{\`e}il (of a mouth)\\
f\ipa{\`a}il (getting)\\
\ipa{\`o}il (of drinking)\\
s\ipa{\`u}il (eye)\\
\hline
\end{tabular}\\
\begin{tabular}{|l|l|}
\hline
{\bf Basic V Combinations} & {\bf More V} \\
\hline
\hline
ai (= \textturna): baile (town) & ao (= \ipa{\`i}): saor (free, cheap)\\
ia (= \ipa{\`i}a): bia (food), iad (them), iasc (fish) & ua (= \ipa{\`u}a): rua (reddish), fuar (cold), suas (up)\\ 
ea (= a): bean (woman) &  ei (= e): ceist (question)\\
eo (= \`o word initial): eolas (information) & i\`u (= \`u word initial): I\`uil (July) \\
uai (= \`ue C final): uair (hour), fuair (got) & iai (= \`ie C final): riail (rule)\\
oi (= \textreve): scoil & io (= i): mion (tiny)\\
{} & ui (= i): cuid (part), muid (we), duine (person)\\
\hline
\end{tabular}\\
\begin{tabular}{|l|l|}
\hline
{\bf Basic C} & {\bf Complex C (slender/broad)}\\
\hline
\hline
d = slender/broad & dh = y/\textgamma\\
t = slender/broad & gh = y/\textgamma, ow\\
l = slender/broad & th = h\\
n = slender/broad & ph = f\\
s = \textesh & ch = x/\textchi\\
r = \textfishhookr & bh, mh = v, ow\\
{} & -igh, -idh = y, \`i, a\\
{} & -adh = a\\
\hline
\end{tabular}\\
						\end{tabular}	

\vspace{1cm}
{\sc Some Rules of Thumb}
\begin{itemize}
\item Broad and slender /l/ occur relative to the (long or short) vowel environment: slender v = /i/--/\ipa{\`i}/, /e/--/\ipa{\`e}/; and broad = /a/--/\ipa{\`a}/, /o/--/\ipa{\`o}/, /u/--/\ipa{\`u}/.
\item A consonant in the middle of a word must be flanked by slender /i/ or /e/ or broad /a/, /o/, or /u/.
\item {\bf caol le caol agus leathan le leathan} = slender with slender and broad with broad.
\item For example, {\sl Feic} (see) + {\sl \ipa{\`a}il} (-ing) is {\bf not} *{\sl Feic\ipa{\`a}il}, but {\sl Faic{\bf e}\ipa{\`a}il} (seeing).
\end{itemize}
 









\newpage
\begin{exe} % sets up the top-level example environment
\ex\label{here} Here is one. % example with running number
\ex[*]{Here another is.} % judged ex. with running number
\ex Here are some with judgements.
\begin{xlist} % first embedding (alphabetical numbering)
\ex[]{A grammatical sentence am I.}
\ex[*]{A ungrammatical sentence is you.}
\ex[??]{A dubious sentence is she.}
\ex % just the number
\begin{xlist} % second embedding (roman numbering)
\ex[**]{Need one a second embedding?}
\ex[\%]{sometime.}
\end{xlist} % end second embedding
\ex Dare to judge me!
\end{xlist} % end first embedding
\ex This concludes the demonstration.
\end{exe}

%Bibliography------------------------------------------------------
%\bibliographystyle{linquiry2}
\bibliography{myrefs}



 
\end{document}

