\documentclass[margin,line]{resume}

\usepackage{graphicx,wrapfig}
\usepackage{url}
\usepackage[colorlinks=true, a4paper=true, pdfstartview=FitV,
linkcolor=blue, citecolor=blue, urlcolor=blue]{hyperref}
\pdfcompresslevel=9

\begin{document}
{\sc \Large {\bf Resume -- Joshua W. Bowles}}
\begin{resume}
	\begin{wrapfigure}{R}{0.7\textwidth}
		\begin{center}
	\includegraphics[width=0.48\textwidth]{closeupred}
	\end{center}
	\end{wrapfigure}

\section{Personal Information}
Joshua W. Bowles\\
Mail: bowlesling@gmail.com\\
% Web: \href{http://sites.google.com/site/linguisticsinai/}{LingAi}\\
Github: \href{https://github.com/jbowles}{jbowles}

\section{Purpose}
Acquire position as Developer
\section{Broad Interests}
Big Data, Linguistics, Artificial Intelligence
% \section{Focus} 
% Programming, Data Storage, Computational Linguistics
\section{Programming Interests} 
Ruby, Rails, JavaScript, Lisp
\section{Data Storage:\\ \small(interests)}
 PostgreSQL, MongoDB, Hadoop, Redis

\section{Employment}
{\bf Software Developer and Researcher}, \hfill 2010 -- present\\
	One On One Marketing, Lehi, UT\\
	\textsl{Development in Ruby and Rails:\\
	--- Maintain Redis-based message queue in Ruby and a MySQL database as an Operational Data Store for Business Intelligence.\\ 
	--- Design architecture for new tracking-code system using MySQL and parsing-expression-grammar in Ruby.\\ 
	--- 2nd lead Dev on Bayesian-statistics-based software for lead-gen inventory distribution in Rails.\\
	--- 3 Months research for an automated bid-system, including writing prototypes for a keyword classification tool and an ontology for sponsored search management.
	 				}
					
{\bf Computational Linguist (Intern)},  \hfill 2009 (July) -- (Nov.)\\
	Attensity Corporation, Salt Lake City, UT\\
	\textsl{Knowledge engineering, named entity recognition, argument structure analysis, categorization of lexical items, some work on developing methods for parsing fixed 												
					phrases and idioms.
					}\\
					
{\bf Adjunct Lecturer}, \hfill 2008 -- 2010\\
  Utah Valley University, Orem, UT\\
	\textsl{Research techniques; informal logic; pragmatics relevant to science composition.
						}

\section{Education}
%% {\bf Doctor of Philosophy, Linguistics}: begin 08-2010\\
%% Currently applying\\ Focus: Computational Pragmatics \& Semantics\\ Places: Stanford, UT Austin, UMass, Northwestern, Rutgers, UW-Madison, Brandeis
{\bf Masters degree in Linguistics}, \ 2008\\ University of Utah, Salt Lake City, UT\\ %\\ Thesis: {\sl Agreement in Tuyuca}\\ Advisor: Lyle Campbell
{\bf Bachelors degree in English and Linguistics}, \  2006\\ Metropolitan State College of Denver, CO %\\ Thesis: {\sl Dislocated Poetics}\\ Advisor: Paul Farkas

\section{Skills}
				{\bf Professional Programming:} Ruby, Ruby on Rails\\
				\\
        {\bf Interested Programming:} Python (mostly with the Natural Language ToolKit), Java (mostly with LingPipe), Lisp (my first love!), Haskell, Clojure\\
				\\
        {\bf Web:} CSS, HTML, jQuery\\
				\\
        {\bf Other:} \LaTeX, Emacs, Vim, SQL, bash, sed, awk, git, Markdown\\
				\\
        {\bf Platforms:} Linux (Ubuntu, Fedora, Arch), Windows (XP), Mac (OS X)
\section{Research}
{\sl Research in Linguistics, Logic, and Computation}
       \begin{itemize}
        \item[]  Formal semantics and pragmatics; Recursive procedures; Categorization/Tagging verb classes and nouns; Computable models of phrase structure; Parsing-Expression-Grammars; Knowledge-representation; "Agile" Ontologies
    \end{itemize}
{\sl Linguistic Research}
    \begin{itemize} % Use \item[] to prevent a bullet from appearing
     \item[] Formal models for context of discourse; Categorization/Tagging of verb classes and nouns; Anaphora of subjects/agents with events; Computable models of phrase structure
      \end{itemize}
{\sl Natural Languages}
        \begin{itemize}
         \item[] Native English; basic Spanish; beginning Portuguese and German;
           structural knowledge of various native American Indian languages
     \end{itemize}

\section{Current Projects:\\ \small(open source)}
		\begin{itemize}
		\item[1] {\bf Grand (Ruby)} {\small CSV file drop for auto-incrementing primary key tables in MySQL. I am a co-author.}
		\item[2] {\bf Pivot (Ruby)} {\small Essentially a Map-Reduce for any kind of data. I was not primary author, but have been extending it to feed the map-reduced data into decision trees, dynamic programming solutions, or any kinds of cool AI-friendly data types. Also working on integrating it with MongoDBs map-reduce for data-mining.}
		\item[3] {\bf StatsGitR (Ruby)} {\small This is a Ruby port of the Python script gitstats. I'll provide it as a gem to include in any app. It collects and graphs statistics for git repo commit behavior and generates html for browser consumption. Long term plans to incorporate this with continuous integration testing to collect statistics on testing coverage and coding... providing metrics such as number of lines of code per testing coverage per deployment. I'm also defining a notion of ``quality'' or ``risk'' for files so that I can provide metrics such as amount of refactoring per quality-risk per deployment (e.g., models are high quality because they are so integral to data and hence provide more risk for deployments, while changes to views are traditionally less ``risky'').} 
		\item[4] {\bf Aesop (Groovy and Grails)} {\small Browser application for field linguistics data entry that uses Natural Language Processing techniques to build a learning platform. Caches data entry as csv files or local database (for doing work in the field), and syncs data entry updates to hosted database.}
		\end{itemize}
% \section{Relevant\\ Work in\\ Progress}
% $\alpha$Merge, Symmetry, and Computability. Manuscript, Utah Valley University.
%
% Some Syntax, Semantics, and Pragmatics of English Evidentiality. Manuscript, Utah Valley University.
%
% Statistical Patterns for Syntactic and Pragmatic Structures in English Evidentiality. Project.


\section{References}
Available on request
\end{resume}
\end{document}






























