\documentclass[12pt]{article}

% Latex template for MIT CV.
% Instructions:  Fill in the commands below.
% You should not need to modify anything past the line marked
% with %%%%%%%%%%%%%%%%%%%%%%%%%%%%%%%%%%%%%%%%%%%%%%%%%%%%%%%%%%%%%%%%%%%%%%%%%%%%%
% Exception:  I didn't deal with teaching evaluation and other educational contribution.
% This template created September 2005 by Bradley C. Kuszmaul.

\newcommand{\datefilledin}{September 20, 2005}
\newcommand{\yourname}{Your Name} \newcommand{\birthday}{April
13, 2014} \newcommand{\education}{ School & Degree &
Date\\\hline%Don't delete this line
% Fill in your schools, degrees, & dates here suitable for a 3-column tabular
Your Alma Mater & S.B. & May 1900 \\
Your PhD Alma Mater & Ph.D. & June 1901 \\
% Massachusetts Institute of Technology & S.B. (course 6) & May 1984\\
% Massachusetts Institute of Technology & S.B. (course 18) & May 1984\\
% Massachusetts Institute of Technology & S.M. & May 1986\\
% Massachusetts Institute of Technology & Ph.D. & June 1994\\

}
\newcommand{\starsoc}{$\star$Socrates}
\newcommand{\academicappointments}{
Title & Institution & Dates \\\hline%Don't delete this line
% Fill in your titles, institutions, and dates suitable for a 3-column tabular
Postdoctoral fellow & MIT & 1902--1903 \\
Assistant professor & MIT & 1903--present \\
}
\newcommand{\citizenship}{
US
% Indicate citizenship (and immigration status if not USA)
}
\newcommand{\thesistitle}{
The Turbo Encabulator
% Title of thesis for most advanced degree
}
\newcommand{\fieldsofinterest}{
% Principal fields of interest
\begin{tabbing}
Turbines \\
Encabulators
\end{tabbing}
}
\newcommand{\otherlabmemberssamefield}{
% Names and ranks of other lab members in the same field
\begin{tabbing}
Anant Agarwal, Professor\\
Arvind, Professor\\
\end{tabbing}

}
\newcommand{\otherfacultysamefield}{
% Names and ranks of faculty in other departments in the same field (this is not just at MIT?)
\begin{tabbing}
Little Fish, Assistant Professor, Nowheresville State University\\
Big Name, Professor, Elsewhere University\\
\end{tabbing}
}
\newcommand{\nonmitexperience}{
% List non-MIT experience including military service.  List
% chronologically by starting date, including part-time and college
% summer jobs if professionally relevant 
%
% Make this suitable for 4-column tabular.
%
% The CSAIL template filled this up with horizontal and vertical
% lines.  As a matter of style, I hate those lines, so I'm not putting
% them in.  -Bradley 
Turbo Encabulators Corp. & CTO & August 1914 & September 1915 \\
}
\newcommand{\mitappointments}{
% List chronologically by starting date:  Include appointments such as
% Instructor or Instructor-G, but not teaching or research
% assistantships.  Include postdoctoral appointments, Lincoln
% Laboratory appointments, and appointments such as laboratory
% directory, etc.  Omit ``Electrical Engineering,''  etc. from
% titles. 
%
% Make this suitable for 3-column tabular
Postdoctoral fellow & 1902 & 1903 \\
Assistant professor & 1903 & present \\
}

\newcommand{\consultingrecord}{
% Chronologically by start date.
%
% Make this suitable for 3-column tabular
Turbo Encabulators Research Corp. & 1942 & 1944 \\
}
\newcommand{\mitservice}{
% Make this suitable for 3-column tabular
Chair, Committee for Incorporating Turbines Into the Curriculum & 1942 & 1948 \\
}

\newcommand{\profservice}{
% Professional service
% 3-column tabular

Workshop Chair, {\em The Turbo Encabulator Workshop,} June 8--9, 1919 & 1919 & 1919  \\\\
Program committee member, {\em International Conference on Turbo Encabulators} & 1920 & 1920 \\\\
}   
\newcommand{\awardsreceived}{
% 2-column tabular.
Piece Prize, for contributions to turbo encabulation & May 1923 \\\\
Big Screw & May 1924 \\\\
}
\newcommand{\organizations}{
% 2-column tabular.
Organization & Offices Held\\\hline%Don't delete this line 
The Institute of Electrical and Electronic Engineers & \\
The Association for Computing Machinery & \\
}
\newcommand{\patents}{
% Here is how to do it.  Uncomment out the begin{enumerate} and
% end{enumerate}, and make it look like this example.
%\begin{enumerate}
%\item Krishna A. Bharat and Robert C. Miller. ``Method for learning
%character patterns to interactively control the scope of a web
%crawler.'' US Patent \#6411952, issued June 25, 2002.
%\end{enumerate}
\begin{enumerate}
\item Your Name Here, ``Method for Turbo Encabulation'', US Patent 9,999,999,
  issued May 5, 1914.
\end{enumerate}

}
\newcommand{\summarystatement}{
     For a number of years work has been proceeding in order to bring
perfection to the crudely conceived idea of a machine that would not
only supply inverse reactive current for use in unilateral phase
detractors, but would also be capable of automatically synchronizing
cardinal grammeters.  Such a machine is the ``Turbo-Encabulator.''
Basically, the only new principle involved is that instead of power
being generated by the relative motion of conductors and fluxes, it
is produced by the modial interaction of magneto-reluctance and
capacitive directance.

(Note: In Real Life$^{\mbox{tm}}$, the Turbo Encabulator first
appeared in an article by J. H. Quick, in \emph{Student's Quarterly
Journal}, Institute of Electrical Engineers, London, 1944.)
}

\newcommand{\professionalstatement}{
% Your professional statement goes here
     For a number of years work has been proceeding in order to bring
perfection to the crudely conceived idea of a machine that would not
only supply inverse reactive current for use in unilateral phase
detractors, but would also be capable of automatically synchronizing
cardinal grammeters.  Such a machine is the "Turbo-Encabulator."
Basically, the only new principle involved is that instead of power
being generated by the relative motion of conductors and fluxes, it
is produced by the modial interaction of magneto-reluctance and
capacitive directance.

     The original machine had a base-plate of pre-fabulated amulite,
surmounted by a malleable logarithmic casing in such a way that the
two spurving bearings were in a direct line with the pentametric fan.
The latter consisted simply of six hydrocoptic marzelvances, so
fitted to the ambifacient lunar waneshaft that side fumbling was
effectively prevented.  The main winding was of the normal lotus-o-delta
type placed in panendermic semi-boloid slots in the stator, every
seventh conductor being connected by a non-reversible tremie pipe to
the differential girdlespring on the "up" end of the grammeters.

     Electrical engineers will appreciate the difficulty of nubing
together a regurgitative purwell and a supramitive wennel-sprocket.
Indeed, this proved to be a stumbling block to further development 
until, in 1942, it was found that the use of anhydrous nangling pins
enabled a kryptonastic bolling shim to be tankered.

     The early attempts to construct a sufficiently robust spiral
decommutator failed largely because of a lack of appreciation of the
large quasi-piestic stresses in the gremlin studs; the latter were
specially designed to hold the roffit bars to the spamshaft.  When,
however, it was discovered that wending could be prevented by a simple 
addition to the living sockets, almost perfect running was secured.

     The operating point is maintained as near as possible to the
h.f. rem peak by constantly fromaging the bitumogenous spandrels.
This is a distinct advance on the standard nivel-sheave in that no
dramcock oil is required after the phase detractors have been remissed.

     Undoubtedly, the turbo-encabulator has now reached a very high
level of technical development.  It has been successfully used for
operating nofer trunnions.  In addition, whenever a barescent skor
motion is required, it may be employed in conjunction with a drawn
reciprocating dingle arm to reduce sinusoidal depleneration.
}
\newcommand{\professionalregistration}{None.}
\newcommand{\majorproducts}{
\begin{description}
\item[Turbo Encabulator]: A system for turbo encabulation.
\end{description}

}


\newcommand{\teachingexperience}{
% This is in a 6-column longtabular
% The columns are Term, SubjectNumber, SubjectTitle, Role, CourseType, SurveyGiven?
% The Title is typeset as a ragged right paragraph
% If you taught at other institutions, do something like
%  \\ \multicolumn{6}{l}{Courses taught at Yale University} \\
% to give yourself a subheader for those courses.
% (But don't include the leading \\ on the first such subheader.)
% 
\multicolumn{6}{l}{Courses taught at MIT} \\
FT 1901 & 0.0001 & Turbo Engineering & T.A. & Lecture. & No \\
ST 1902 & 0.0002 & Encabulator Systems & Recitation Instructor & Lecture & Yes \\
\\ \multicolumn{6}{l}{Courses taught at Turbo Encabulator Institute}\\
FT 1904 & TE628a & Encabulators Applications & In Charge & Seminar & Yes \\
\\ \multicolumn{6}{l}{Courses taught at MIT} \\
ST 1905 & 0.0003 & Encabulation Lab & In Charge & Lab & Yes \\
}

% Warning:  I didn't create things for teaching evaluation data
%  or for other educational contribution

\newcommand{\books}{
% For books, you need to uncomment the enumeration, and then put in your \items
\begin{enumerate}
\item Your N. Here, ``Turbo Encabulators for Dummies'', MIT Press, 1928.
\end{enumerate}
}
\newcommand{\papersinrefereedjournals}{
%1. Brad A. Myers, Richard G. McDaniel, Robert C. Miller, Alan Ferrency, Andrew
%   Faulring, Bruce D. Kyle, Andrew Mickish, Alex Klimovitski, and Patrick Doane.
%   ``The Amulet Environment: New Models for Effective User Interface Software
%   Development.'' \emph{IEEE Transactions on Software Engineering}, v23 n6, June 1997,
%   pp. 347--365. EXAMPLE PLS DELETE

\item Your N. Here, B. Name, L. Fish, ``Turbo Encabulation Alternatives.''
\emph{IEEE Transactions on Turbine Engineering}, v23 n6, June 1928.
}

\newcommand{\papersinrefereedconferences}{
%1. Brad A. Myers, Francesmary Modugno, Rich McDaniel, David Kosbie, Andrew
%   Werth, Robert C. Miller, John Pane, James Landay, Jade Goldstein, and Matthew
%   A. Goldberg. ``The Demonstrational Interfaces Project at CMU.'' \emph{1996 AAAI
%   Spring Symposium on Acquisition, Learning and Demonstration: Automating
%   Tasks for Users}, March 1996. Technical Report SS-96-02, pp. 85--91.
%   EXAMPLE PLS DELETE

\item ``Fast Turbo Encabulation'',  1937 International Conference on Encabulation (ENCAB'33), Munich, Germany July 1937.  pp.~10145--10146. 
}

\newcommand{\othermajorpublications}{
\item Your Name Here.
``Checking Your Turbo Encabulator''
Technical Report MIT/\linebreak[0]{}LTE/TR-321, Massachussetts Institute of
  Technology, Laboratory for Turbo Encabulation, May 1948.
}

\newcommand{\internalmemos}{
\item Your Name Here.
``Turbo Encabulation Progress Report.''  Progress report to OSI, May 1949.
}

\newcommand{\invitedtalks}{
\item ``Turbid Encabulators in Systems Contexts.''
\begin{itemize}
\item RPI, March 6, 1960.
\item Northeastern University, March 13, 1961.
\item Northwestern University, March 15, 1961.
\item University of Illinois at Chicago, March 17, 1963.
\end{itemize}

\item ``Universial Encabulation.''
 ICEIE'55, April 27, 1955.
}

\newcommand{\researchcontractsandgrants}{
% This is 3 columns.  Each entry takes 3 lines
%   Line 1:  Year & Sponsor & Annual expenditure per year \tabularnewline
%                 & Project title  & \tabularnewline
%                 & Role in research & \tabularnewline\tabularnewline
% Note: Use \tabularnewline, not \\ 
% Put two \tabularnewlines after the third line of each entry.
 
1927 & Tabulator Corporation & \$2,600,000$/$3 years\tabularnewline
     & ``Proposal for Utilization of Advanced Tabulation Based Platforms in
                         Encabulationally Demanding Tasks,''
  with B. Name and L. Fish. \tabularnewline
 & Co-PI \tabularnewline\tabularnewline


1962 & NSF & \$205,000$/$4 years \tabularnewline
     & ``Using Encabulation in Turbo Applications.'' \tabularnewline
     & PI \tabularnewline\tabularnewline
}

% How many theses have been supervised?
\newcommand{\sbdone}{13}  % Number of SB degrees completed under my supervision
\newcommand{\sbundone}{1} % Number of SBs in progrewss
\newcommand{\smdone}{2}   % SMs completed
\newcommand{\smundone}{0} % SMs in progress
\newcommand{\mengdone}{3} % MEngs completed
\newcommand{\mengundone}{0} % Mengs in progress
\newcommand{\engdone}{0}   % Engineer's completed
\newcommand{\engundone}{0} % Engineer's in progress
\newcommand{\drsupdone}{0} % Number of doctoral theses supervised & completed 
\newcommand{\drsupundone}{0} % Number of doctoral theses supervised in progress
\newcommand{\drreaddone}{0} % Number of doctoral theses, as reader, completed 
\newcommand{\drreadundone}{0} % Number of doctoral theses, as reader, in progress


\newcommand{\bachelorstheses}{
\begin{itemize}
\item L. Fish, Encabulation Simulation.  (Spring 1968.)
\end{itemize}
}

\newcommand{\smtheses}{
%\begin{itemize}
%\end{itemize}
}

\newcommand{\mengtheses}{
%\begin{itemize}
%\end{itemize}
}

\newcommand{\engineerstheses}{
}
\newcommand{\doctoralthesessupervisor}{
}
\newcommand{\doctoralthesesreader}{
}

%%%%%%%%%%%%%%%%%%%%%%%%%%%%%%%%%%%%%%%%%%%%%%%%%%%%%%%%%%%%%%%%%%%%%%%%%%%%%
%%%%%%%%%%%%%%%%%%%%%%%%%%%%%%%%%%%%%%%%%%%%%%%%%%%%%%%%%%%%%%%%%%%%%%%%%%%%%
%               You should not need to change anything below this           %
%%%%%%%%%%%%%%%%%%%%%%%%%%%%%%%%%%%%%%%%%%%%%%%%%%%%%%%%%%%%%%%%%%%%%%%%%%%%%
%%%%%%%%%%%%%%%%%%%%%%%%%%%%%%%%%%%%%%%%%%%%%%%%%%%%%%%%%%%%%%%%%%%%%%%%%%%%%

\newcommand{\MIT}{MASSACHUSETTS INSTITUTE OF TECHNOLOGY}
\newcommand{\CSAIL}{Computer Science and Artificial Intelligence Laboratory}
\usepackage{array}
\usepackage{fullpage}
\usepackage{url}
\usepackage[dvips]{epsfig}
\usepackage{longtable}
\newcommand{\pushright}[1]{\begin{tabular}{ll}\mbox{ }&#1\end{tabular}}
\newcommand{\littlenote}[1]{\emph{\footnotesize #1}}
\newcommand{\standardtwotab}[1]{
\noindent
{\small
\begin{longtable}{p{3.6in}p{1.6in}}
#1
\end{longtable}
}
}
\newcommand{\standardthreetab}[1]{
\noindent
\hspace{-0.4in}
{\small
\begin{longtable}{>{\raggedright}p{4.0in}p{.9in}p{.9in}}
#1
\end{longtable}
}
}

\begin{document}
\special{papersize=8.5in,11in}%Tell dvips to generate letter paper (not the default, which is a4 on many systems
\begin{center}
\MIT\\[2ex]
\CSAIL\\
Summary Sheet\\[2ex]



\begin{tabular}{p{3.05in}>{\raggedright}p{3.05in}}
Date: \datefilledin      & Full Name: \yourname  \tabularnewline[1ex]
                         & Department: \CSAIL \tabularnewline
Date of Birth: \birthday & \\[1ex]
\end{tabular}

\end{center}

\noindent Education: \littlenote{List in chronological order by degree, bachelor's degree first}

\pushright{
\begin{tabular}{lll}
\education
\end{tabular}
}

\noindent Academic Appointments:

\pushright{
\begin{tabular}{lll}
\academicappointments
\end{tabular}
}
\vspace*{1ex}

\noindent Summary: % What is this?
\summarystatement
\clearpage
\begin{center}
Professional Statement of \yourname

\littlenote{Not more than one page}

\end{center}

\professionalstatement

\clearpage
\begin{center}
\MIT\\[2ex]
\CSAIL\\
Personnel Record\\[2ex]

\begin{tabular}{p{3.05in}p{3.05in}}
Date: \datefilledin      & Full Name: \yourname  \\[1ex]
                         & Department: \CSAIL \\
\end{tabular}
\end{center}

\begin{enumerate}
\item Date of Birth: \birthday
\item Citizenship: \citizenship

\noindent\littlenote{If not a US citizen, indicate immigration status.}

\item Education: \littlenote{List in chronological order by degree, bachelor's degree first.}

\noindent\begin{tabular}{lll}
\education
\end{tabular}

\item Title of Thesis for Most Advanced Degree:

\thesistitle

\item Principal Fields of Interest

\fieldsofinterest

\item Name and Rank of Other Laboratory Members in the Same Field: \littlenote{List alphabetically by rank.}

\otherlabmemberssamefield

\item Name and Rank of Faculty in Other Departments in the Same Field:

\otherfacultysamefield

\item Non-MIT Experience (including military service):  \littlenote{List chronologically by starting date.  Include part-time and summer jobs while in college if professionally relevant.}

\begin{tabular}{llll}
Employer & Position & Beginning & Ending \\\hline%Don't delete this line
\nonmitexperience
\end{tabular}

\item History of MIT Appointments: \littlenote{List chronologically by starting date:  Include appointments such as Instructor or Instructor-G, but not teaching or research assistantships.  Include postdoctoral appointments, Lincoln Laboratory appointments, and appointments such as laboratory directory, etc.  Omit ``Electrical Engineering,''  etc. from titles.}

\standardthreetab{
Rank & Beginning & Ending\\\hline%Don't delete this line
\mitappointments}

\item Consulting Record: \littlenote{List chronologically by start date.}

\standardthreetab{
Firm & Beginning & Ending\\\hline%Don't delete this line
\consultingrecord}

\item Department and Institute Committees, Other Assigned Duties:
  \littlenote{List chronologically by starting date; include
  activities such as committees, counseling, graduate admissions,
  etc.  Distinguish between department, laboratory, and Institute
  activities.  Do not include thesis or UROP supervision.}

\standardthreetab{Activity & Beginning & Ending\\\hline%Don't delete this line
\mitservice}

\item Professional Service: \littlenote{List chronologically by
  starting date.  Include positions such as committees, program chair,
  etc.}

\standardthreetab{
Activity & Beginning & Ending\\\hline%Don't delete this line
\profservice}

\item Awards Received: \littlenote{List chronologically; include
  teaching awards and competitive fellowships, such as Hertz and NSF.
  Do not include MIT-administered fellowships, research grants, or
  honorary societies.  List honorary societies under Item 14.}

\standardtwotab{
Award & Date\\\hline%Don't delete this line 
\awardsreceived}


\item Current Organization Membership:  \littlenote{Unless an
  abbreviation is widely known, spell out names of organizations.
  Include professional honorary societies.  For offices held (if any)
  include dates.  These are elected offices only; positions such as
  program chair are listed in Item 12.}

\standardtwotab{\organizations}

\item Patents and Patent Applications Pending:  \littlenote{List
  chronologically by date of issue or filing.  Number each item.}

\makeatletter
\renewcommand\theenumii{\@arabic\c@enumii}
\makeatother
\renewcommand\labelenumii{\theenumii.}
\patents

\item Professional Registration

\professionalregistration

\item Major New Products, Processes, Designs, or Systems.
  \littlenote{This section is for items that represent significant
  achievements requiring synthesis.  Each item description should be
  no more than two lines long.}

\majorproducts
\end{enumerate}

\clearpage

\begin{center}
Teaching Experience of \yourname
\end{center}

\pagestyle{myheadings}
\markboth{Teaching Experience}{Teaching Experience}
\begin{enumerate}
\item Teaching Experience

\littlenote{Repeat this heading on any additional pages; list subjects
taught in chronological order up to and including this term.  Include
summer session as well as major IAP subjects.  If you have taught at
another university, insert headings as needed to identify universities
and list those subjects chronologically.  Be sure to include subject
development.  Term designations: SU79 = summer session 1979;  FT79 =
fall term 1979;  ST80 = spring term 1980, etc.  Course time: lecture,
laboratory, design, seminar.  {\bf Please specify (yes/no) whether
  course evaluation survey given.}} 

{\small
\begin{longtable}{ll>{\raggedright}p{2.15in}>{\raggedright}p{1in}ll}
{\footnotesize Term} & {\footnotesize Subject} & {\footnotesize Title} & {\footnotesize Role} & {\footnotesize Course} & {\footnotesize Course} \\
     & {\footnotesize Number}  &       &      & {\footnotesize type}   & {\footnotesize evaluation} \\
     &         &       &      &        & {\footnotesize survey} \\      
     &         &       &      &        & {\footnotesize given} \\ \\
\teachingexperience
\end{longtable}
}

\item Teaching Evaluation Data

\begin{tabular}{lllllll}
\end{tabular}

\item Other Educational Contribution

\makeatletter
\renewcommand\theenumii{\@alph\c@enumii}
\makeatother
\renewcommand\labelenumii{\theenumii)}

\begin{enumerate}
\item Teaching materials developed that illustrate teaching
  effectiveness or innovativeness (e.g., course syllabi, lecture or
  recitation content, course handouts, student assignments,
  educational technology modules):
\item Education contributions, apart from classroom performance and
  supervision, such as new educational programs and curricula
  developed by the candidate (reference pertinent education
  publications or presentations in other sections of the FPR):
\end{enumerate}
\end{enumerate}

\clearpage
\markboth{Publications}{Publications}
\begin{center}
Publications of \yourname
\end{center}

{\baselineskip=11pt
\littlenote{ In Category 2, include only papers that have been
published or accepted for publication. Papers that have been submitted
but not yet accepted should be listed in Category 4 below. Papers that
are actively being prepared for publication should be listed on a
separate, unnumbered sheet for department review. Books in the final
editing or printing stage, books of which the candidate is the editor,
and books to which the candidate contributed a chapter, should usually
be listed in Category 4. Books for which the candidate was the editor
should only be included under Category 1 if he or she wrote a
significant amount of material for the book. Be sure to mark with **
any publications that are outgrowths of supervised student research
and a footnote of the form ** should be included at the bottom of the
page. Please note and follow the style shown in the examples below,
and be sure to include the location of conferences.
}

}

\makeatletter
\renewcommand\theenumii{\@arabic\c@enumii}
\makeatother
\renewcommand\labelenumii{\theenumii.}

\begin{enumerate}
\item Books
\books

\item Papers in Refereed Journals \littlenote{List chronologically by publication date; number each item.}

\begin{enumerate}
\papersinrefereedjournals
\end{enumerate}
\item Proceedings of Refereed Conferences
\begin{enumerate}
\papersinrefereedconferences
\end{enumerate}

\item Other Major Publications
\begin{enumerate}
\othermajorpublications
\end{enumerate}

\item Internal Memoranda and Progress Reports
\begin{enumerate}
\internalmemos
\end{enumerate}

\item Invited Lectures \littlenote{List chronologically; do not number; note style. Include papers given at
conferences without published proceedings. Talks given at several locations in succession can be
grouped as shown. Talks given in MIT subjects should not be listed here.}

\begin{itemize}
\invitedtalks
\end{itemize}
\end{enumerate}

\clearpage
\markboth{Research Contracts and Grants}{Research Contracts and Grants}
\begin{center}
Research Contracts and Grants of \yourname
\end{center}


\begin{longtable}{l>{\raggedright}p{3.8in}>{\raggedright}p{1.4in}}
Year & Sponsor & Annual Contract Expenditures \tabularnewline
     & Project Title &  \tabularnewline
     & Role in research (PI, Co-PI, Other) \tabularnewline\hline
\researchcontractsandgrants
\end{longtable}

\clearpage
\markboth{Supervision}{Supervision}
\begin{center}
Theses Supervised by \yourname
\end{center}

\newcounter{sbtotal}
\setcounter{sbtotal}{\sbdone}
\addtocounter{sbtotal}{\sbundone}

\newcounter{smtotal}
\setcounter{smtotal}{\smdone}
\addtocounter{smtotal}{\smundone}

\newcounter{mengtotal}
\setcounter{mengtotal}{\mengdone}
\addtocounter{mengtotal}{\mengundone}

\newcounter{engtotal}
\setcounter{engtotal}{\mengdone}
\addtocounter{engtotal}{\mengundone}

\newcounter{drsuptotal}
\setcounter{drsuptotal}{\drsupdone}
\addtocounter{drsuptotal}{\drsupundone}

\newcounter{drreadtotal}
\setcounter{drreadtotal}{\drreaddone}
\addtocounter{drreadtotal}{\drreadundone}

\newcounter{drtotal}
\setcounter{drtotal}{\value{drsuptotal}}
\addtocounter{drtotal}{\value{drreadtotal}}

\newcounter{drdone}
\setcounter{drdone}{\drsupdone}
\addtocounter{drdone}{\drreaddone}

\newcounter{drundone}
\setcounter{drundone}{\drsupundone}
\addtocounter{drundone}{\drreadundone}


\begin{tabular}{lrrrrrr}
                   & Total && Completed && In Progress \\
Bachelor's (6.AUP) & \thesbtotal   &&  \sbdone      && \sbundone & \\
SM                 & \thesmtotal   &&  \smdone      && \smundone \\
MEng               & \themengtotal &&  \mengdone    && \mengundone \\
Engineer's         & \theengtotal  &&  \engdone     && \engundone \\
Doctoral           & \thedrtotal   &&  \thedrdone   && \thedrundone \\
\mbox{ } As Supervisor & \multicolumn{2}{r}{\thedrsuptotal} & \multicolumn{2}{r}{\drsupdone} & \multicolumn{2}{r}{\drsupundone} \\
\mbox{ } As Reader     & \multicolumn{2}{r}{\thedrreadtotal} & \multicolumn{2}{r}{\drreaddone} & \multicolumn{2}{r}{\drreadundone} \\
\end{tabular}

\noindent \textbf{Bachelor's Theses (6.AUP)}

\bachelorstheses

\noindent \textbf{SM Theses}

\smtheses

\noindent \textbf{MEng Theses}

\mengtheses

\noindent \textbf{Engineer's Theses}

\engineerstheses

\noindent \textbf{Doctoral Theses, Supervisor}

\doctoralthesessupervisor

\noindent \textbf{Doctoral Theses, Reader}

\doctoralthesesreader

\newpage

\begin{center}
Postdoctoral Associates and Fellows Supervised by \yourname
\end{center}

\noindent
\begin{tabular}{llll}
\multicolumn{4}{l}{Current Postdocs} \\
Name & Dates of Appointment & Ph.D. Granting Institution & Curr. Position \\
\\
\multicolumn{4}{l}{Previous Postdocs} \\
Name & Dates of Appointment & Ph.D. Granting Institution & Curr. Position \\
\\
\end{tabular}


\end{document}

% LocalWords:  Akamai Kozyrakis Kunle Olukotun
