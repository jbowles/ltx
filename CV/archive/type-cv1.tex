\documentclass[margin,line]{resume}
 
\usepackage{graphicx,wrapfig}
\usepackage[names]{xcolor}
\definecolor{bolinkcolor}{rgb}{.4,.2,.0}
\usepackage[pdftex,colorlinks,breaklinks,
		linkcolor=bolinkcolor,
		citecolor=bolinkcolor,
		urlcolor=bolinkcolor,
		plainpages=false,
		pdfpagelabels,
		bookmarks=false,
		pdfstartview=FitH]{hyperref}
\urlstyle{rm}
\setlength{\topmargin}{-0.6in}  % Start text higher on the page 
\setlength{\textheight}{9.8in}  % increase textheight to fit more on a page
\setlength{\headsep}{0.2in}     % space between header and text
\setlength{\headheight}{12pt}   % make room for header
\usepackage{fancyhdr}  % use fancyhdr package to get 2-line header
%\renewcommand{\headrulewidth}{0pt} % suppress line drawn by default by fancyhdr
\lhead{\ttfamily \scriptsize Joshua Bowles} % force lhead all the way left
\rhead{\ttfamily \scriptsize Page \thepage}  % put page number at right
\chead{\ttfamily \scriptsize Last Updated \today}
\cfoot{}  % the footer is empty
\pagestyle{fancy} % set pagestyle for the document
\thispagestyle{empty} % this page does not have a header

\begin{document} 
{\sc \Large {\bf CV --- Joshua W. Bowles}}
\begin{resume}
   

\section{Education}
{\bf Doctor of Philosophy, Linguistics} \hfill {\bf begin Fall 2010} \\ 
Presently applying to programs in which I can focus on:\\ Computational Pragmatics, Formal Semantics, Syntactic Theory\\ Places: Stanford, MIT, UCLA, UC Santa Cruz, UT Austin,\\ Rutgers  

{\bf Master of Arts, Linguistics} \hfill {\bf 2008} (12-2008 \textgreater{} 08-2006) \\ 
University of Utah, Salt Lake City, UT\\ Thesis: \href{http://sites.google.com/site/bowleslinguistics/Home/research}{\emph{Agreement in Tuyuca}}\\ Advisor: Lyle Campbell 

{\bf Bachelors of Arts, English and Linguistics} \hfill {\bf 2006} (05-2006 \textgreater{} 08-2000) \\ Metropolitan State College of Denver, CO\\ Thesis: \href{http://sites.google.com/site/bowleslinguistics/Home/research/misc-1/symmetry-1/literature}{\emph{Dislocated Poetics}}\\ Advisor: Paul Farkas 
 

\section{Employment}
 {\bf Adjunct Lecturer}, Department of English, \hfill 2008 (August) -- present\\
    Utah Valley University, Orem, UT 		

{\bf Computational Linguist (Intern)}, Attensity Group, \hfill 2009 (July) -- (Nov.)\\
		U.S. Technology Center, Salt Lake City, UT  
		 
   
 
    {\bf Teaching Assistant}, Department of Linguistics, \hfill 2006 -- 2008\\ 
    University of Utah, Salt Lake City, UT 
 
\section{Research} 
{\bf Pragmatics, Semantics, Syntax}

{\sc Interests}\\ 
Evidentiality, Presupposition, Interrogatives, Computational Pragmatics

{\sc Other Areas}\\
Logic, Biolinguistics, Artificial Intelligence (NLP, knowledge representation, inference, context)
 
{\sc Current Projects}
\begin{enumerate} 
 \item Symmetry in X-bar structure
 \item Some syntax, semantics, and pragmatics evidentiality
\item English Corpus of Evidentiality Data
\item Computing fixed phrases for idioms and questions
 \end{enumerate} 


\section{Publications}

\section{{\sl Refereed\\ Conference Publications}}
(in press) The morphosyntax of tense-evidentials: An initial model. \emph{\href{http://ling.wisc.edu/lso/wpl-main.html}{LS0} Working Papers in Linguistics: Proceedings of the 6th Workshop in General Linguistics,} Volume 8. University of Wisconsin-Madison.

(in press) The Eastern Tukanoan languages and the typology of classifiers. \emph{Proceedings of the 33rd
Annual Meeting of the \href{http://linguistics.berkeley.edu/BLS/past_meetings.html}{Berkeley Linguistics Society.}} (Wilson Silva and Joshua Bowles)


\section{{\sl Notices\\ \& Reviews}}
(2009) Review of Heck (2008), \emph{On pied-piping: Wh-movement and beyond} (Mouton de
Gruyter). \href{http://linguistlist.org/issues/20/20-2283.html}{LinguistList}.

(2007) Review of Niyogi (2006), \emph{The computational nature of language learning and evolution} (MIT Press). {\it Word} 57.2: 178-183. \texttt{(pre-published version \href{http://sites.google.com/site/bowleslinguistics/Home/research/teaching-1/publications}{here})}

(in preparation) Notice of Sch\"afer (2008), \emph{The syntax of (anti-)causatives: External arguments in change-of-state contexts} (Amsterdam: John Benjamins). \href{http://www.elanguage.net/home.php}{eLanguage.}


(submitted) Notice of Booij (2007), \emph{The grammar of words} (Oxford University Press). \href{http://www.elanguage.net/home.php}{eLanguage.}


\section{{\sl Miscellaneous}}
(submitted) Language families. \emph{Encyclopedia of Native American History} ed. by \href{http://college.usc.edu/faculty/faculty1003494.html}{Peter C.Mancall}. New York: Facts on File. \texttt{(see submitted draft \href{http://sites.google.com/site/bowleslinguistics/Home/research/teaching-1/publications}{here})}

(submitted) Linguistic areas. \emph{Encyclopedia of Native American History} ed. by Peter C. Mancall. New York: Facts on File. \texttt{(see submitted draft \href{http://sites.google.com/site/bowleslinguistics/Home/research/teaching-1/publications}{here})}

(2008) Some questions about determining causal inference and criteria for evidence:
Response to Ladd, Dediu, and Kinsella (2008). \emph{\href{http://www.biolinguistics.eu.}{Biolinguistics}} 2.2-3: 247-255.

\section{Work in\\ Progress}

$\alpha$Merge, Symmetry, and Group Theory (will submit soon). \textsl{A look at what algebraic group theory may offer syntax relative to research in biolinguistics and Fibonacci patterns; relates to larger issues of computability and symmetry}. See a draft \href{http://sites.google.com/site/bowleslinguistics/Home/research}{\texttt{here}}.%Submit to:  Biolingusitics

Recursion for Computation's Sake (just started). %Submit to: Lingusitics and Philosophy 

Some Syntax, Semantics, and Pragmatics of Evidentiality in English (just started). \textsl{Draws a distinction between evidential morphosyntax and the concept of evidentiality. Analyzes English data and compares it to Tuyuca. Are the semantics the same?} %Submit to: Linguistics

English Corpus of Evidentiality Data (goal of 1 million sentences/clauses; presently at about .02\% of goal).

%Recursion, Merge, and Human Syntax: Evidentials in Pirah\~a (just started). \textsl{Look at Everett's claims for Pirah\~a syntax and its relation to the evidential system. Following work by Julie Legate on Warlpiri, I expect the rigid order of evidentials, focus, and other material in the Pirah\~a left periphery matches the functional hierarchy proposed by Guglielmo Cinque. I use this to argue that Pirah\~a is configurational and has `recursion.' } 

%Bare Nominals in Brazilian Portuguese (April 2008) {\sl Assuming Chierchia's Nominal Mapping Parameter, I look at Brazilian Portuguese. Following work done by Schmitt and Munn I argue that BP has unpronounced noun classifiers.} See a draft \href{http://sites.google.com/site/bowleslinguistics/Home/research}{\texttt{here}}.

%Noun-Classifiers and Quantification (not started). \textsl{Dependent on comparative data from South American Indian languages, Chinese. The hunch is that noun-classifiers can quantify objects according to `topological space.'}

Tuyuca Data Set. Latest version \texttt{December 2008 \href{http://sites.google.com/site/bowleslinguistics/Home/research}{here}}.

%Goethe's Leaves, Turing's Machine, and the Human Mind: An Exploration of Modern Linguistic Theory (expected summer 2012). \textsl{Book length manuscript surveys diversity of methods in linguistic research (`functional,' `generative,' and `computational'). For a general audience.} See \href{http://sites.google.com/site/bowleslinguistics/Home/research/misc-1}{\texttt{here}} for possible updated drafts.


\section{Talks and Conferences}
(2008a) Evidentials, Agreement, and Tuyuca: Initial Hypotheses. \emph{Linguistics Department
Student Conference} 7, University of Utah, April 10.

(2008b) Evidentiality and Agreement Interactions in Tuyuca: A Generative Approach.
\emph{Workshop in General Linguistics} 6, University of Wisconsin-Madison, April 4-5.

(2008c) Agreement and Tense-Evidentiality Morphs in Tuyuca. \emph{Workshop on Structure
and Constituency in the Languages of the Americas} 13, Queens University,
Ontario, Canada, March 28-30 (\texttt{unable to attend}).

(2008d) The ``Language of Observation'' in Linguistics (Informal Presentation). University of Utah Linguistics Reading Group, February 15.

(2007a) Neural correlates of Grammatical Gender: Functionalist and Formalist
Assumptions (Informal Presentation). University of Utah Linguistics Reading
Group, October 15.

(2007b) Classifiers and Typology in Eastern Tukanoan Languages. \emph{7th Biennial Meeting of the Association for Linguistic Typology}, Paris, France, September 25-28.
(Wilson Silva and Joshua Bowles)

(2007c) Formalism and Functionalism in Language Documentation. \emph{Department of
Foreign Languages and Literatures}, Universidade Federal do Amazonas, Brazil,
June 10.

(2007d) The Amazonian Languages Research and Documentation Group. Poster. \emph{3rd
Annual Conference on Endangered Languages and Cultures of the Native
Americas}, University of Utah, April 13-15.
(Wilson Silva and Joshua Bowles)

(2007e) The Eastern Tukanoan Languages and the Typology of Classifiers. \emph{33rd Annual Berkeley Linguistics Society Conference}, University of Berkeley, California,
February 9-11. (Wilson Silva and Joshua Bowles)

(2007f) A Typology of Nominal Classifiers in the Eastern Tukanoan Languages.
\emph{Department of Linguistics Speakers Series}, University of Utah, January 24. (Wilson Silva and Joshua Bowles)

(2006) Dislocated Poetics: Towards a Neurobiological Explanation for Poetry and Poetic
Language. \emph{Honors Faculty Colloquium}, Metropolitan State College of Denver,
Colorado, May 05.

(2004) Neurobiology, Poetry, and Dislocation. \emph{English Faculty Colloquium}, Metropolitan State College of Denver, Colorado, April 25. (Paul Farkas and Joshua Bowles)

(2002) Philosophy of Language, the Dialectic Method, and James Joyce. \emph{Honors Regional Conference}, University of Colorado, Boulder, March 24-25.


\section{Teaching}
Fall 2009 -- Fall 2008 \emph{Science \& Technology Writing and Research} ENG 2020, Utah Valley University (9 courses total)

Summer 2009  \emph{Humanities \& Social Science Writing and Research} ENG 2010, Utah Valley University

Summer 2009 \emph{Introduction to Composition} ENG 1010, Utah Valley University


{\section{\sl Teaching\\ Assistant}}
Spring 2008 -- Spring 2007 \emph{Introduction to Language} LING 1200, University of Utah\\
(Instructors: David Iannucci (1 semester), Mauricio Mixco (2 semesters))

Fall 2006 \emph{Cross-cultural Communication} LING 3600, University of Utah\\
(Instructor: Stephen Sternfeld)

\section{Languages}
English (native), German and Old English (1 year formal study each), Spanish (good speaking and reading), Portuguese (basic reading, beginning speaking), Tuyuca (structural and research knowledge)

\section{Computer} Linux/Unix environments; \LaTeX; recursive procedures \& recursion theory

\section{Programming}
Actively learning: Python, Lisp (clisp, scheme), and Prolog 


\section{Field Work}
{\bf 2007, June-August} Sao Gabriel da Cachoeria and Camanaus, Brazil. Work on all grammatical aspects of Desano and limited work on Cubeo; Tukanoan family. {\bf 2004, September} Denver, Colorado, USA. Limited work on phonology of Afghani Poshtun; Indo-Iranian family.


\section{Grants} 
{\bf 2007} Endangered Language Fund (Co-PI) \$3500  {\bf 2007} International Field Work Travel Funds, University of Utah \$300 


\section{Professional\\ Activities} 
{\bf 2007} Attended by invitation with Wilson Silva the 20th Anniversary Meeting of FOIRN (Federac\~ao das Organizac\~oes Indigenas do Rio Negro) Taracua, Brazil: June 25-29. {\bf 2007-2008} Volunteer work for \textsl{Conference on Endangered Languages and Cultures of the North Americas}, University of Utah. {\bf 2006} Co-founding member of research group: Amazonian Languages Research and Documentation, Center for American Indian Languages, University of Utah. {\bf 2003} Lead student organizer and introductory comments for ``Noam Chomsky: live at the Auraria Campus,'' Presented by MSCD Student Activities, Denver, Colorado, April 04.

\section{Misc.}
Professional Chef with extensive fine dining experience \hfill 1997 - present
 
\section{Research\\ Statement}
Much of my previous work has been in the morphosyntax and typology of American Indian Languages under the direction of Lyle Campbell; mostly looking at classifiers and evidentiality in Tukanoan languages with some research on ergativity in Cariban languages (both are South American Indian language families). Recently, I have developed an intense interest in theoretical syntax and formal/computational semantics and pragmatics. I am also very interested in Biolinguistics (for example, (a) the concepts of symmetry and complexity in nature and natural language, (b) computability of nature and how it relates to language, logic, and the mathematicization/formalization of linguistic theories, and (c) the physical tractability of linguistic formalizations).

\section{Personal\\ Information} 
Born: May 23, 1976\\
Orem, UT\\
\begin{tabular}{|c|}
	\hline
	bowlesling(\texttt{AT})gmail(\texttt{dot})com\\
\href{http://sites.google.com/site/bowleslinguistics}{http://sites.google.com/site/bowleslinguistics}\\
Department of English\\
Utah Valley University\\
800 University Parkway\\
Orem, UT 84058\\
\hline		
\end{tabular}

 \end{resume}
\end{document}