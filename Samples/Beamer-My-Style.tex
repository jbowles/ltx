\documentclass[hyperref={pdfpagelabels=false},pdfauthor={true},%handout
]{beamer}
\setbeamercovered{dynamic}
\setbeamertemplate{navigation symbols}{}
%\usepackage{beamerthemesplit}
\usepackage{lmodern}
\usetheme%{Berkeley}
%{Warsaw}{Frankfurt}%
{Berlin} 
%\usecolortheme{seahorse}{}%{rose}{}%{lily}{}
\newtheorem{argument}[theorem]{argument}
% For Handout: will provide light grey background for the slides.
%\mode<handout>{\setbeamercolor{background canvas}{bg=black!5}}
\begin{document}

\title{My Beamer}
\author[Bowles]{Joshua Bowles}
\date{\today: This conference presentation at this location}
\subtitle{Beamer}
\institute{Utah Valley University}


\begin{frame}
\titlepage
\end{frame}

\begin{frame}\frametitle{Contents}\tableofcontents
\end{frame}

\section{First}
\frame{\tableofcontents[currentsection]}
\begin{frame}
\frametitle{There Is No Largest Prime Number}
\framesubtitle{The proof uses \textit{reductio ad absurdum}.}
\begin{theorem}
There is no largest prime number.
\end{theorem}
\begin{proof}
\begin{enumerate}
\item<1-> Suppose $p$ were the largest prime number.
\item<2-> Let $q$ be the product of the first $p$ numbers.
\item<3-> Then $q + 1$ is not divisible by any of them.
\item<1-> Thus $q + 1$ is also prime and greater than $p$.\qedhere
\end{enumerate}
\end{proof}
\uncover<4->{The proof used \textit{reductio ad absurdum}.}
\end{frame}

\section{1}
\begin{frame}\frametitle{1.1}
ssssssssssssssss
\end{frame}

\section{2}
\begin{frame}\frametitle{2.2}
tttttttttttttttttttttt
\end{frame}

\section{3}
\begin{frame}\frametitle{3.3}
rrrrrrrrrrrrrrrrrrrrrrr
\end{frame}

\section{4}
\begin{frame}\frametitle{4.4}
uuuuuuuuuuuuuuu
\end{frame}


\end{document}