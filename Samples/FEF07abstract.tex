\documentclass[12pt]{article}
\pagestyle{empty}

%%%%%%%%%%%%%%%%%%%%%%%%%%%%%%%%%%%%%%%%%%%%%%%%%%
% Do not modify the dimensions of the page
\setlength{\topmargin}{-0.3in}
\setlength{\textheight}{9.9in}
\setlength{\oddsidemargin}{-0.5in}
\setlength{\textwidth}{7.5in}
% Do not modify the dimensions of the page
%%%%%%%%%%%%%%%%%%%%%%%%%%%%%%%%%%%%%%%%%%%%%%%%%%

% No paragraph indent or paragraph skip
\parindent=0pt \parskip=0pt

\begin{document}

\centerline{\bf AUTHOR'S INSTRUCTIONS FOR PREPARING USNCCM7 ABSTRACTS}

\vspace{12pt}

\centerline{ {\bf A.~Author}$^{\rm a}$ and {\bf B.~Author}$^{\rm b}$}

\vspace{12pt}

\centerline{$^{\rm a}$Department of Mechanical Engineering}
\centerline{The University of Michigan}
\centerline{Ann Arbor, Michigan 48109-2125}
\centerline{author1@engin.umich.edu}

\vspace{12pt}

\centerline{$^{\rm b}$Department of Mechanical Engineering}
\centerline{2256 Seamans Center}
\centerline{The University of Iowa}
\centerline{Iowa City, Iowa 52242-1527}
\centerline{author2@icaen.uiowa.edu}

\vspace{12pt}
\vspace{12pt}

For each conference presentation, an abstract is required that
summarizes the presentation. The abstract submission process has
several steps: (1) submission of a preliminary, initial abstract, by
October 20, 2006, to the appropriate minisymposium (2) notification 
of abstract acceptance, by November 17, 2006, and (3) submission of final 
abstract, in print ready form, by January 25, 2007.

\vspace{12pt}

The final abstract length should not exceed 1 page, as the book of
abstracts will adhere to a 1 page per abstract printing. The final
abstract must contain (1) the presentation title, (2) names of the
presentation authors, and (3) contact information for the
corresponding author, including authors' postal and e-mail
addresses. Templates for the preparation of the abstracts in LaTeX,
Microsoft Word, and FrameMaker formats may be found on the FEF07
website ({\em http://www.esc.sandia.gov/FEF07/FEFhome.html}). Once the abstract is
prepared please convert the abstract to PDF format. All abstracts must
be submitted in electronic form via the web. {\bf Only PDF files will be
accepted}. (LaTeX, Microsoft Word, or text files will NOT be accepted). 

\vspace{12pt}

The final abstract is to be prepared using the following format:

$\bullet$ Margins: 0.5" margins left and right, 0.6" margin 
at top, and 0.5" margin at bottom

$\bullet$ Title in 14 pt bold Helvetica (or similar san serif font), centered,
small caps

$\bullet$ Authors' names in Bold

$\bullet$ For LaTeX, 12pt article documentclass

$\bullet$ Embed all fonts in the PDF file, e.g., for Adobe
Distiller, select ``Job Options'', and click ``Embed Font''.

$\bullet$ 1 line spacing between title and authors

$\bullet$ 1 line spacing between authors and affiliations

$\bullet$ 2 line spacing between affiliations and abstract body

$\bullet$ 1 line spacing between abstract paragraphs

$\bullet$ No paragraph indents

$\bullet$ 1 line spacing between abstracts and references

$\bullet$ Equations and figures may be included

$\bullet$ ``References'' heading in Bold type as shown below

$\bullet$ References numbered and indented as shown below, No line
spacing between listed references


\vspace{12pt}

\parindent=0pt
{\bf References}

% \parindent=16pt
[1] T.~Belytschko, Y.~Lu, and L.~Gu, ``Element-Free 
Galerkin Methods," {\it International Journal for Numerical Methods in
Engineering\/}, v.~37, p.~229-256, 1994.


\end{document}
