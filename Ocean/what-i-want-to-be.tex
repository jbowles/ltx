\section{Early}\begin{wrapfigure}{R}{0.4\textwidth}
         \vspace{-1cm}
        \begin{center}
        \includegraphics[width=0.4\textwidth]{lalooks}
        \end{center}
         \vspace{-1cm}
    \end{wrapfigure}
Early in the morning, before the sun rises, I dress: put on cheap sneakers and vigourously slick the {\sl LA Looks} gel into my tangled mane of hair. In Junior High School hair is only one of significant variables that build into the social factor. With my cheap clothes and uncut, untamed hair I am by look rough and poor kid. My small stature undermines any kind of intimidation or fear that poor kids try leverage against their low socio-economic status. Because of this, I fell back on getting along. And one major part of getting along with people is tact. A pushover is an easy target for bullies and manipulators; and a dissident is an easy person to dismiss as difficult or non-friendly -- if ignoring them doesn't work. 

Despite, or perhaps becuase of, my homely status on the surface I was able to to make friends by a combination of pragmatic forthrightness, honesty, and self deprication. I think I made people feel comfortable through my ability to make fun of myself without actually hating myself. And I'm pretty sure my pragmatic view of the world communicated a sense of stability and and strength. However the puzzle fit together, I was a popular kid because I could talk to people without creating conflict and I could create conflict without challenging a person's self-defined status. Perhaps becuase of this I took on an interesting role of settign up fights between stoners and cowboys. I had an afinity with the stoners and rockers of the school, but could converse with the cowboys -- at least on a superficial level. I don't remember how many fights I coordinated but it probably wasn't that many: history, memory, and time having magnified the activity into something larger than it in fact was. 

\subsection{A Director?}\begin{wrapfigure}{R}{0.2\textwidth}
         \vspace{-1cm}
        \includegraphics[width=0.2\textwidth]{movie-director}
         \vspace{-1cm}
    \end{wrapfigure} 
  In a junior high school class the students sit while the teacher instructs. Joshua, an inattentive boy, is more than likely trying to talk to someone or find some other activity to keep busy. Whatever the case, he is not keeping busy with class work until a sheet of paper with instructions is passed to him. 
\begin{quote}
   Look through the list of careers and select one that you would like to do. Explain your choice and what you like about it.
\end{quote}
\subsubsection{Movie Director}
I am not sure what prompted me to select movie director; it certainly was not prophetic. And it certainly was not one of those life-changing events where the seemingly arbitrary mention of what one could do with their life imparted a light and a fire in me the burned to genuine passion. But the selection did have to unintended and unexpected consequences:
\begin{enumerate}
    \item I started thinking about what I would do for a career
\item I remembered this event vividly all my life
\end{enumerate}
I do not know why I remembered this small event so clearly all my life: it is just one those memories that has a strange significance for reasons we cannot see. If one wants to call it destiny or furtune is fine with me. I tend to see it like this:

\textcolor{green}{\begin{tabular}{|p{4in}|}
\hline
      \textcolor{black}{The theory of neuroscience (at the time of this writing: December 2009), and more specifically neuropsychology, hypothesizes that our personal identity is literally a string of re-played narrative memories. That is, it is a bit like a movie: each event of our lives plays like one single sensory event. Each event is sequenced together one by one, and unraveling of sequences at certain speed gives the illusion of the coherence of time; much like how movie is merely sequence of photographs strung together in a rapid linear progression. The events we experience are being captured and processed from the outside world and brought inside our bodies through our skin and other aparatus like eyes and ears -- all working in concert with the an emergent result that approximates something like fluid conscious experience. Such experiences are remembered and throughout our life we construct narrative sequences of these experiences. This sequence becomes our identity. We have some conscious control over what we remember and to what degree we obsess over such memories. We do not, however, have much control over how we remember. In other words, as the narrative gets told over and over again, we may omit or add varible pieces of information, retaining the overall structure. If specific experience happens to be quite vibrant, though perhaps not significant, we may retain it to a degree beyond our conscious will. But at this point, we are still excercising some degree of will in contemplating on the event or experience. As we grow older and look for meaning in life, we imbue many such retained experiences with a kind of personal narrative meaning. This becomes our identity. It is real and powerful. But, there is also a strong sense of arbitrariness to it\ldots much like a hurricane or tornado or supermassive black hole. But because we are humans we have to power to create a meaningful structure out of quite arbitrary events. This is an amazing property. We also, in an interesting way, lack a substantial ammount of power in constructing from new what things should mean. What I intend by this, is that what a culture has selected as meaningful is beyond our power when we are initially born into the culture. As time passes, we incorporate power and influence do some degree and retain a privelage to take part in re-constructing what a culture deems meaningful.}\\
\hline 
\end{tabular}}

All that from choosing to be a director? Yes. It was a powerful event in my life and I have contemplated it often. Even more powerful was the message retained from the simple classroom excercise: that only I had the control to become something in life. The sense of responsibility filled me both with awe and terror and has given me anxiety that day on. Even at the moment I write this paragraph\footnote{Christmas eve, 2009 I have finished all my PhD. applications and hope to get accepted to Stanford; but really I hope to get accepted anywhere because I have no idea what I will do for a career if not academia.} I am confronted with the ever-evolving sense of \textsl{what I want to}, \textsl{what I can do}, and \textsl{what I will be paid to do}. The only other day in my life that I remember being so overcome with the sense of my own personal engagement with the rest of the world was my baptism day at the age of eight; a full four years previous.

The experience described above reminds of one I read in Carl C. Jung's autobiography: he had a dream as child and in this dream he was church and saw God hovering above the church. Then God defecated on the church the young Jung woke up. The older Jung explained in detail the awe and terror he initially and continually received from the dream; it was a defining point in his life because of both the vividness of the dream and the continued contemplation of the dream's meaning. It is, perhaps, no accident he later became a psychologist. Coincidentally, it is probably not an accident that I have always turned to careers wher I was in control of the work: first as poet and novelist, then as an academic scholar. In these professions I can direct my own path of investigation and I am in charge, to some degree, of the direction I take my research.


   