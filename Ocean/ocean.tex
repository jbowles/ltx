%% Based on a TeXnicCenter-Template by Gyorgy SZEIDL.
%%%%%%%%%%%%%%%%%%%%%%%%%%%%%%%%%%%%%%%%%%%%%%%%%%%%%%%%%%%%%

%----------------------------------------------------------
%
\documentclass[a4paper,10pt]{book}%
%
%----------------------------------------------------------
% This is a sample document for the AMS LaTeX Book or Monograph Class
% Class options
%       --  Body text point size:
%                        8pt, 9pt, 10pt (default), 11pt, 12pt
%       --  Paper size:  letterpaper (8.5x11 inch, default), a4paper
%       --  Orientation: portrait(default), landscape
%       --  Print side:  oneside, twoside (default)
%       --  Quality:     final(default), draft
%       --  Title page:  titlepage, notitlepage
%       --  Start chapter on left:
%                        openright (no, default), openany
%       --  Columns:     onecolumn (default), twocolumn
%       --  Omit extra math features:
%                        nomath
%       --  AMS fonts (noamasfonts available):
%                        noamsfonts
%       --  PSAMSfonts (fewer AMSfontsizes)
%                        psamsfonts
%       --  Equation numbering (equation numbers on the left is the default)
%                        leqno (default), reqno
%       --  Equation centering (equations centered is the default)
%                        centeredtags (default}, tbtags (top, bottom)
%       --  Displayed equations (centered is the default)
%                        fleqn (flush left)
% For instance the command
%          \documentclass[a4paper,12p,reqno]{amsbook}
% ensures that the paper size is a4, fonts are typeset at the size 12p
% and the equation numbers are on the right side.
%
\usepackage[utf8x]{inputenc}
\usepackage{fancyhdr}
\lhead{\scriptsize Joshua Bowles} % force lhead all the way left
\rhead{\scriptsize Page \thepage}  % put page number at right
\chead{\scshape \scriptsize \ttfamily Last updated: \today}
\cfoot{} % the footer is empty
\pagestyle{fancy}
\usepackage{amsmath}
\usepackage{amsfonts}
\usepackage{amssymb}
\usepackage{amsthm}
\usepackage{natbib,times,latexsym}
\usepackage{makeidx}
\usepackage{graphicx,wrapfig}
%--------------------------------------------


\usepackage[usenames]{xcolor}
\definecolor{jblinkcolor}{rgb}{.0,.2,.4}
\usepackage[bookmarksdepth=3,colorlinks,breaklinks,
			linkcolor=jblinkcolor,
			citecolor=jblinkcolor,
			urlcolor=jblinkcolor,
			plainpages=false,
			bookmarks=false]{hyperref}
			\urlstyle{rm}
		
\makeindex
%-----------------------------------------BEGIN DOCUMENT---------------------------------
\begin{document}

\frontmatter 
\title{\Huge Ocean}
\author{\includegraphics[height=.15\textheight,width=1\textwidth]{ocean1}\\ \LARGE Joshua Bowles\\ \date{}}

\maketitle   
\thanks{\begin{center}{To all that matter.}
\end{center}}  
        


\tableofcontents



\chapter*{Preface}
I have wanted to begin this project many times; I finally started on Christmas eve of two-thousand nine. The guiding motivation behind such a project is to leave a record of my life for my children. 

Our children come to our lives knowing only a part of us. Much of a parent's early life, whether before having children or when the children were too young to remember who we were, is better not told; and rarely is. I understand this latter sentiment, but I also understand the importance of leaving a record. I also believe in children \textsl{eventually} knowing about their parents' lives. In many ways, our struggles as human individuals become thier struggles too. The failures and frustrations, false starts, dreams, and dark hours that have molded us into the people we are today also mold our children in ways not always obvious. Our successes as parents and individuals are singularities in the backdrop of the ocean of life: our private discovery of what it means to have a life is seldom passed on to our children in substantive form. It is a shame that our children rarely have access to the horizon of our past, that they cannot explore our own vast journey. 

This book is meant to make up for the lack of access children have to their parents' past. And though I have not written it yet, I am sure there will be things about me they wished were not true, but I beleive in the end they will be glad for the perspective. I hope my children do not read this until they are grown, and I hope that once they do read it, the book will shed light on their own lives and their own continuous journey through life.

\begin{flushright}
\noindent Joshua Bowles\\
\today
\end{flushright}

\begin{table}[!b]\caption{Timeline}
\begin{center}
\begin{tabular}{|l||l|}
\hline
Utah & Somewhere\\
\hline \hline
\textbf{Started} \date{December 24, 2009} & \textbf{Finished} Sometime\\
Jade: 14 & Jade: ? \\
Sophiamary: 3 & Sophiamary: ? \\
Isabella: 1 & Isabella: ? \\ 
\hline
\end{tabular}
\end{center}
\end{table}

 
\mainmatter

\chapter[What I Want To Be?]{What I Want To Be?}\begin{wrapfigure}{R}{0.99\textwidth}
         \vspace{-1cm}
        \begin{center}
        \includegraphics[height=.5\textheight,width=0.99\textwidth]{research2}
        \end{center}
         \vspace{-1cm}
    \end{wrapfigure}
\include{what-i-want-to-be}

\chapter[Love, Words, and Math]{Love, Words, and Math}\begin{wrapfigure}{R}{0.99\textwidth}
         \vspace{-1cm}
        \begin{center}
        \includegraphics[height=.6\textheight,width=0.99\textwidth]{mathy}
        \end{center}
         \vspace{-1cm}
    \end{wrapfigure}

\include{love-words-math}

\chapter[Kitchen Garden]{Kitchen Garden}\begin{wrapfigure}{R}{0.99\textwidth}
         \vspace{-1cm}
        \begin{center}
        \includegraphics[height=.6\textheight,width=0.99\textwidth]{mathy}
        \end{center}
         \vspace{-1cm}
    \end{wrapfigure}

The past is a strange beast. I had a dream last night about two people I once knew, Angus Hicks and Jerry Payne, and when I woke up I wanted to see them. I know both of these guys from my time in Denver. My Denver time is only time in life I felt like I truly made friends. I miss those guys.

For some time I have wanted to take you girls to Denver. I want to take you to eat at a restuarant. It was at this restuarant that I made my friends. Jerry Payne, Angus HIcks, Don Hudson, Teri Rippetto. They all revolve around ``the restaurant".

When the Fall season comes around I get remembrances from the past. Smells, the light of day, the feel of air... they all bring the past back to me. The past is dangerous becuase it bids us to reside in it as long as we can. But there is no moving forward in the past, and the remembrance of the past is never accurate. It is colored by the feeling of today. Sometimes the past is emberrasing, sometimes comforting. If remebering the past is everything, that is, if it includes a full spectrum of emotion the you are doing it right. If the past is nothing more than the same wave repeating itself then something is wrong.

The past is a paradox. It never changes. It always changes. This is what it means to be a human. If the past is stored in bytecode then it never changes. The solid state memory of an event never changes. The signature of an event written into magnetic disk never changes. But humans are not bytecode or solid state drives or magentic disks. We are not 800 millimeter tape, we are not flash drives. Events captured without interpretation are static waves repeating. But this doesn't dampen the paradox. When I look at pictures of you girls when you were 9 years or 2 years or 6 months I feel again, and again differently each time colored by the passage of time. And when I remember you... that too is colored by the passage time... by the transformation of flesh and bone and synapse.

I feel this for my own personal past. I miss many things and wish that many things were not there. But I remember it all anyway, I make myself remember the emberrasements and the mistakes just as much as the highlights and joys. I remember working at Potager and the friends I made. It was the only time in my adult like that I felt like I had real friends. Uncharacteristically of me, I miss them. I do not usually miss people, I do not usually want to see poeple after my time with them is done. You girls, your mother, and friends from Denver: I miss.
































%\bibliographystyle{linquiry2}
%\bibliography{myrefs}

%\printindex


\end{document}
