%% Based on a TeXnicCenter-Template by Gyorgy SZEIDL.
%%%%%%%%%%%%%%%%%%%%%%%%%%%%%%%%%%%%%%%%%%%%%%%%%%%%%%%%%%%%%

%----------------------------------------------------------
%
\documentclass[a4paper,10pt]{book}%
%
%----------------------------------------------------------
% This is a sample document for the AMS LaTeX Book or Monograph Class
% Class options
%       --  Body text point size:
%                        8pt, 9pt, 10pt (default), 11pt, 12pt
%       --  Paper size:  letterpaper (8.5x11 inch, default), a4paper
%       --  Orientation: portrait(default), landscape
%       --  Print side:  oneside, twoside (default)
%       --  Quality:     final(default), draft
%       --  Title page:  titlepage, notitlepage
%       --  Start chapter on left:
%                        openright (no, default), openany
%       --  Columns:     onecolumn (default), twocolumn
%       --  Omit extra math features:
%                        nomath
%       --  AMS fonts (noamasfonts available):
%                        noamsfonts
%       --  PSAMSfonts (fewer AMSfontsizes)
%                        psamsfonts
%       --  Equation numbering (equation numbers on the left is the default)
%                        leqno (default), reqno
%       --  Equation centering (equations centered is the default)
%                        centeredtags (default}, tbtags (top, bottom)
%       --  Displayed equations (centered is the default)
%                        fleqn (flush left)
% For instance the command
%          \documentclass[a4paper,12p,reqno]{amsbook}
% ensures that the paper size is a4, fonts are typeset at the size 12p
% and the equation numbers are on the right side.
%
\usepackage[utf8x]{inputenc}
\usepackage{fancyhdr}
\lhead{\scriptsize Joshua Bowles} % force lhead all the way left
\rhead{\scriptsize Page \thepage}  % put page number at right
\chead{\scshape \scriptsize \ttfamily Last updated: \today}
\cfoot{} % the footer is empty
\pagestyle{fancy}
\usepackage{amsmath}
\usepackage{amsfonts}
\usepackage{amssymb}
\usepackage{amsthm}
\usepackage{natbib,times,latexsym}
\usepackage{makeidx}
\usepackage{graphicx,wrapfig}
%--------------------------------------------


\usepackage[usenames]{xcolor}
\definecolor{jblinkcolor}{rgb}{.0,.2,.4}
\usepackage[bookmarksdepth=3,colorlinks,breaklinks,
			linkcolor=jblinkcolor,
			citecolor=jblinkcolor,
			urlcolor=jblinkcolor,
			plainpages=false,
			bookmarks=false]{hyperref}
			\urlstyle{rm}
		
\makeindex
%-----------------------------------------BEGIN DOCUMENT---------------------------------
\begin{document}

\frontmatter 
\title{\Huge Ocean}
\author{\includegraphics[height=.15\textheight,width=1\textwidth]{ocean1}\\ \LARGE Joshua Bowles\\ \date{}}

\maketitle   
\thanks{\begin{center}{To all that matter.}
\end{center}}  
        


\tableofcontents



\chapter*{Preface}
I have wanted to begin this project many times; I finally started on Christmas eve of two-thousand nine. The guiding motivation behind such a project is to leave a record of my life for my children. 

Our children come to our lives knowing only a part of us. Much of a parent's early life, whether before having children or when the children were too young to remember who we were, is better not told; and rarely is. I understand this latter sentiment, but I also understand the importance of leaving a record. I also believe in children \textsl{eventually} knowing about their parents' lives. In many ways, our struggles as human individuals become thier struggles too. The failures and frustrations, false starts, dreams, and dark hours that have molded us into the people we are today also mold our children in ways not always obvious. Our successes as parents and individuals are singularities in the backdrop of the ocean of life: our private discovery of what it means to have a life is seldom passed on to our children in substantive form. It is a shame that our children rarely have access to the horizon of our past, that they cannot explore our own vast journey. 

This book is meant to make up for the lack of access children have to their parents' past. And though I have not written it yet, I am sure there will be things about me they wished were not true, but I beleive in the end they will be glad for the perspective. I hope my children do not read this until they are grown, and I hope that once they do read it, the book will shed light on their own lives and their own continuous journey through life.

\begin{flushright}
\noindent Joshua Bowles\\
\today
\end{flushright}

\begin{table}[!b]\caption{Timeline}
\begin{center}
\begin{tabular}{|l||l|}
\hline
Utah & Somewhere\\
\hline \hline
\textbf{Started} \date{December 24, 2009} & \textbf{Finished} Sometime\\
Jade: 14 & Jade: ? \\
Sophiamary: 3 & Sophiamary: ? \\
Isabella: 1 & Isabella: ? \\ 
\hline
\end{tabular}
\end{center}
\end{table}

 
\mainmatter

\chapter[What I Want To Be?]{What I Want To Be?}\begin{wrapfigure}{R}{0.99\textwidth}
         \vspace{-1cm}
        \begin{center}
        \includegraphics[height=.5\textheight,width=0.99\textwidth]{research2}
        \end{center}
         \vspace{-1cm}
    \end{wrapfigure}
\section{Early}\begin{wrapfigure}{R}{0.4\textwidth}
         \vspace{-1cm}
        \begin{center}
        \includegraphics[width=0.4\textwidth]{lalooks}
        \end{center}
         \vspace{-1cm}
    \end{wrapfigure}
Early in the morning, before the sun rises, I dress: put on cheap sneakers and vigourously slick the {\sl LA Looks} gel into my tangled mane of hair. In Junior High School hair is only one of significant variables that build into the social factor. With my cheap clothes and uncut, untamed hair I am by look rough and poor kid. My small stature undermines any kind of intimidation or fear that poor kids try leverage against their low socio-economic status. Because of this, I fell back on getting along. And one major part of getting along with people is tact. A pushover is an easy target for bullies and manipulators; and a dissident is an easy person to dismiss as difficult or non-friendly -- if ignoring them doesn't work. 

Despite, or perhaps becuase of, my homely status on the surface I was able to to make friends by a combination of pragmatic forthrightness, honesty, and self deprication. I think I made people feel comfortable through my ability to make fun of myself without actually hating myself. And I'm pretty sure my pragmatic view of the world communicated a sense of stability and and strength. However the puzzle fit together, I was a popular kid because I could talk to people without creating conflict and I could create conflict without challenging a person's self-defined status. Perhaps becuase of this I took on an interesting role of settign up fights between stoners and cowboys. I had an afinity with the stoners and rockers of the school, but could converse with the cowboys -- at least on a superficial level. I don't remember how many fights I coordinated but it probably wasn't that many: history, memory, and time having magnified the activity into something larger than it in fact was. 

\subsection{A Director?}\begin{wrapfigure}{R}{0.2\textwidth}
         \vspace{-1cm}
        \includegraphics[width=0.2\textwidth]{movie-director}
         \vspace{-1cm}
    \end{wrapfigure} 
  In a junior high school class the students sit while the teacher instructs. Joshua, an inattentive boy, is more than likely trying to talk to someone or find some other activity to keep busy. Whatever the case, he is not keeping busy with class work until a sheet of paper with instructions is passed to him. 
\begin{quote}
   Look through the list of careers and select one that you would like to do. Explain your choice and what you like about it.
\end{quote}
\subsubsection{Movie Director}
I am not sure what prompted me to select movie director; it certainly was not prophetic. And it certainly was not one of those life-changing events where the seemingly arbitrary mention of what one could do with their life imparted a light and a fire in me the burned to genuine passion. But the selection did have to unintended and unexpected consequences:
\begin{enumerate}
    \item I started thinking about what I would do for a career
\item I remembered this event vividly all my life
\end{enumerate}
I do not know why I remembered this small event so clearly all my life: it is just one those memories that has a strange significance for reasons we cannot see. If one wants to call it destiny or furtune is fine with me. I tend to see it like this:

\textcolor{green}{\begin{tabular}{|p{4in}|}
\hline
      \textcolor{black}{The theory of neuroscience (at the time of this writing: December 2009), and more specifically neuropsychology, hypothesizes that our personal identity is literally a string of re-played narrative memories. That is, it is a bit like a movie: each event of our lives plays like one single sensory event. Each event is sequenced together one by one, and unraveling of sequences at certain speed gives the illusion of the coherence of time; much like how movie is merely sequence of photographs strung together in a rapid linear progression. The events we experience are being captured and processed from the outside world and brought inside our bodies through our skin and other aparatus like eyes and ears -- all working in concert with the an emergent result that approximates something like fluid conscious experience. Such experiences are remembered and throughout our life we construct narrative sequences of these experiences. This sequence becomes our identity. We have some conscious control over what we remember and to what degree we obsess over such memories. We do not, however, have much control over how we remember. In other words, as the narrative gets told over and over again, we may omit or add varible pieces of information, retaining the overall structure. If specific experience happens to be quite vibrant, though perhaps not significant, we may retain it to a degree beyond our conscious will. But at this point, we are still excercising some degree of will in contemplating on the event or experience. As we grow older and look for meaning in life, we imbue many such retained experiences with a kind of personal narrative meaning. This becomes our identity. It is real and powerful. But, there is also a strong sense of arbitrariness to it\ldots much like a hurricane or tornado or supermassive black hole. But because we are humans we have to power to create a meaningful structure out of quite arbitrary events. This is an amazing property. We also, in an interesting way, lack a substantial ammount of power in constructing from new what things should mean. What I intend by this, is that what a culture has selected as meaningful is beyond our power when we are initially born into the culture. As time passes, we incorporate power and influence do some degree and retain a privelage to take part in re-constructing what a culture deems meaningful.}\\
\hline 
\end{tabular}}

All that from choosing to be a director? Yes. It was a powerful event in my life and I have contemplated it often. Even more powerful was the message retained from the simple classroom excercise: that only I had the control to become something in life. The sense of responsibility filled me both with awe and terror and has given me anxiety that day on. Even at the moment I write this paragraph\footnote{Christmas eve, 2009 I have finished all my PhD. applications and hope to get accepted to Stanford; but really I hope to get accepted anywhere because I have no idea what I will do for a career if not academia.} I am confronted with the ever-evolving sense of \textsl{what I want to}, \textsl{what I can do}, and \textsl{what I will be paid to do}. The only other day in my life that I remember being so overcome with the sense of my own personal engagement with the rest of the world was my baptism day at the age of eight; a full four years previous.

The experience described above reminds of one I read in Carl C. Jung's autobiography: he had a dream as child and in this dream he was church and saw God hovering above the church. Then God defecated on the church the young Jung woke up. The older Jung explained in detail the awe and terror he initially and continually received from the dream; it was a defining point in his life because of both the vividness of the dream and the continued contemplation of the dream's meaning. It is, perhaps, no accident he later became a psychologist. Coincidentally, it is probably not an accident that I have always turned to careers wher I was in control of the work: first as poet and novelist, then as an academic scholar. In these professions I can direct my own path of investigation and I am in charge, to some degree, of the direction I take my research.


   

\chapter[Love, Words, and Math]{Love, Words, and Math}\begin{wrapfigure}{R}{0.99\textwidth}
         \vspace{-1cm}
        \begin{center}
        \includegraphics[height=.6\textheight,width=0.99\textwidth]{mathy}
        \end{center}
         \vspace{-1cm}
    \end{wrapfigure}

There have been three essentials in my life since the age of about 17; I leave you to guess them from them title. Each essential represents as domain of my life that has range both far-reaching and long-traveled. Each domain started in one place and is now in a very different place. In a word, the function of each domain is the same, but the value of the variables changes considerably.

\section{Love}
I met my wife in 1995, dated her for a year, and did not see her agian until four months before we got married in 2005.

\section{Words}
I began experimenting with writing around 16 and developed an weird interest in writing poetry, which then changed to writing novels. I entrolled in college as way to make connections with the publishing world. I enrolled in linguistics courses to help me understand language so I could write good novels, but the intellectual satisfaction proved to be intoxicating. Simultaneously, I was coming to the hard truth that I wasn't really that good at fiction: I always prefered analysis over exposition. Tearing things apart, instead of weaving them together, is a bad habit for a fiction writer. The rise in intellectual challenge coupled with the decline in fictive ability left me longing for words in way that poetry or fiction could not satisfy.

(After years of isolation, study, writing, submitting poems and stories, and hearing my friends and family say my writing was very abstract, I finally realized I was no good at fiction. When I look back on it, I was always more interested in analysis than exposition. Another things is now obvious when I look back: I was always more interested in finding a poetic excuse to read about physics or mathematics than I was in honing a poetic ``voice.'')

\section{Math}
I never liked arithmetic. My teachers were buffoons and gym instructors; but I was always attracted to symbolic exoticism and the beauty of proofs. It wasn't until I was almost done with undergraduate college that I had to take a class called Art of Critical Thinking. It was in actual fact a baby introduction to logic. I spent hours in the library trying to decipher syllogistic and first order predicate logic. It was frustrating and exhiliarating. I did not do well in that class because of some intersting factors out of my control. For one, the midterm exam consisted on two sheets of paper, both doubl-sided. I got a bad grade on it and after we got the results went with a friend to coffee shop to compare answers (she had done considerably better than I had). When we were there, I noticed that I had on-sided exams when everyone else had two-sided exams. That means I got zero for roughly half of the questions. Also, I never showed up for the final; which means 0. The final was in December 2001, the year of \texttt{9/11}. It was a strange time and I was a member of student government: we were busy organizing events that would help everyone make sense out the attack and the racist, nationalistic pride that was taking hold in the country.\footnote{I personally had friend who was student body president of the University of Colorado, Denver who was almost brutally attacked by four white guys in a truck. He was walking home one night from buying groceries and a truck driving by threw a beer bottle at him, strking him in the back of the head. The four white guys in the truck stopped and started calling my friend ``raghead,'' threatening to attack him. Luckily, they drove off.} Miraculously, I got a C out of the class and when I say my teacher and thanked him for the generous grade, reminding him of my incomplete final and and non-existent final (which were enough to easily ensure a failing grade), he said I deserved my grade. I still do not know if that was compliment or insult, but at the time I took it as humble complement.

Next semester I took the Introduction to Symbolic Logic course and was very excited. I noticed many interesting connections between mathematical logic and language, unfortuantely, there were no faculty equiped to deal with my questions. I was left with a lot of self-doubt about such connections, not sure if I should invest time trying to make such connections more realizable or if the connections I saw were in fact realizable. I went so far as to email Martin Davis and Noam Chomsky, both of whom were elegent in response, but obviously much to busy to dedicate time to my education. It was not until my first formal semantics class that I saw such connections between formal logica and natural language realized in all thier beauty.

Since that time I have dedicated more and more time to investigating mathematics and language. One thing I learned through not having a mentor able enough to deal with my questions was to trust my academic instincts about nature of mathematics and language. To this day, I explore what seem to me viable methods of inquiry that apparently no one else is exploring; only time will tell if it is good gamble.

I used to say that I am not a mathematician, I am a linguist. But as I study more math, and devote more time to mathematical investigations, my negative statement seems less and less the case. It seems now that my interest in mathematics is not so much about making sure nobody thinks I am a mathematician, but making sure I do mathematics well and not caring about what people classify me as.

\chapter[Kitchen Garden]{Kitchen Garden}\begin{wrapfigure}{R}{0.99\textwidth}
         \vspace{-1cm}
        \begin{center}
        \includegraphics[height=.6\textheight,width=0.99\textwidth]{mathy}
        \end{center}
         \vspace{-1cm}
    \end{wrapfigure}

The past is a strange beast. I had a dream last night about two people I once knew, Angus Hicks and Jerry Payne, and when I woke up I wanted to see them. I know both of these guys from my time in Denver. My Denver time is only time in life I felt like I truly made friends. I miss those guys.

For some time I have wanted to take you girls to Denver. I want to take you to eat at a restuarant. It was at this restuarant that I made my friends. Jerry Payne, Angus HIcks, Don Hudson, Teri Rippetto. They all revolved around ``the restuarant".
































%\bibliographystyle{linquiry2}
%\bibliography{myrefs}

%\printindex


\end{document}
