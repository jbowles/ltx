There have been three essentials in my life since the age of about 17; I leave you to guess them from them title. Each essential represents as domain of my life that has range both far-reaching and long-traveled. Each domain started in one place and is now in a very different place. In a word, the function of each domain is the same, but the value of the variables changes considerably.

\section{Love}
I met my wife in 1995, dated her for a year, and did not see her agian until four months before we got married in 2005.

\section{Words}
I began experimenting with writing around 16 and developed an weird interest in writing poetry, which then changed to writing novels. I entrolled in college as way to make connections with the publishing world. I enrolled in linguistics courses to help me understand language so I could write good novels, but the intellectual satisfaction proved to be intoxicating. Simultaneously, I was coming to the hard truth that I wasn't really that good at fiction: I always prefered analysis over exposition. Tearing things apart, instead of weaving them together, is a bad habit for a fiction writer. The rise in intellectual challenge coupled with the decline in fictive ability left me longing for words in way that poetry or fiction could not satisfy.

(After years of isolation, study, writing, submitting poems and stories, and hearing my friends and family say my writing was very abstract, I finally realized I was no good at fiction. When I look back on it, I was always more interested in analysis than exposition. Another things is now obvious when I look back: I was always more interested in finding a poetic excuse to read about physics or mathematics than I was in honing a poetic ``voice.'')

\section{Math}
I never liked arithmetic. My teachers were buffoons and gym instructors; but I was always attracted to symbolic exoticism and the beauty of proofs. It wasn't until I was almost done with undergraduate college that I had to take a class called Art of Critical Thinking. It was in actual fact a baby introduction to logic. I spent hours in the library trying to decipher syllogistic and first order predicate logic. It was frustrating and exhiliarating. I did not do well in that class because of some intersting factors out of my control. For one, the midterm exam consisted on two sheets of paper, both doubl-sided. I got a bad grade on it and after we got the results went with a friend to coffee shop to compare answers (she had done considerably better than I had). When we were there, I noticed that I had on-sided exams when everyone else had two-sided exams. That means I got zero for roughly half of the questions. Also, I never showed up for the final; which means 0. The final was in December 2001, the year of \texttt{9/11}. It was a strange time and I was a member of student government: we were busy organizing events that would help everyone make sense out the attack and the racist, nationalistic pride that was taking hold in the country.\footnote{I personally had friend who was student body president of the University of Colorado, Denver who was almost brutally attacked by four white guys in a truck. He was walking home one night from buying groceries and a truck driving by threw a beer bottle at him, strking him in the back of the head. The four white guys in the truck stopped and started calling my friend ``raghead,'' threatening to attack him. Luckily, they drove off.} Miraculously, I got a C out of the class and when I say my teacher and thanked him for the generous grade, reminding him of my incomplete final and and non-existent final (which were enough to easily ensure a failing grade), he said I deserved my grade. I still do not know if that was compliment or insult, but at the time I took it as humble complement.

Next semester I took the Introduction to Symbolic Logic course and was very excited. I noticed many interesting connections between mathematical logic and language, unfortuantely, there were no faculty equiped to deal with my questions. I was left with a lot of self-doubt about such connections, not sure if I should invest time trying to make such connections more realizable or if the connections I saw were in fact realizable. I went so far as to email Martin Davis and Noam Chomsky, both of whom were elegent in response, but obviously much to busy to dedicate time to my education. It was not until my first formal semantics class that I saw such connections between formal logica and natural language realized in all thier beauty.

Since that time I have dedicated more and more time to investigating mathematics and language. One thing I learned through not having a mentor able enough to deal with my questions was to trust my academic instincts about nature of mathematics and language. To this day, I explore what seem to me viable methods of inquiry that apparently no one else is exploring; only time will tell if it is good gamble.

I used to say that I am not a mathematician, I am a linguist. But as I study more math, and devote more time to mathematical investigations, my negative statement seems less and less the case. It seems now that my interest in mathematics is not so much about making sure nobody thinks I am a mathematician, but making sure I do mathematics well and not caring about what people classify me as.