\documentclass{article}
\usepackage{articlep}
\usepackage{articlec}
\usepackage{morphgloss}
%=================for the LaTeX section============================
% Math-mode symbol & verbatim
\def\W#1#2{$#1{#2}$ &\tt\string#1\string{#2\string}}
\def\X#1{$#1$ &\tt\string#1}
\def\Y#1{$\big#1$ &\tt\string#1}
\def\Z#1{\tt\string#1}

% A non-floating table environment.
\makeatletter
\renewenvironment{table}%
   {\vskip\intextsep\parskip\z@
    \vbox\bgroup\centering\def\@captype{table}}%
   {\egroup\vskip\intextsep}
\makeatother

% All the tables are \label'ed in case this document ever gets some
% explanatory text written, however there are no \refs as yet. To save
% LaTeX-ing the file twice we go:
\renewcommand{\label}[1]{}

% % A4 page setup
% \topmargin -45pt
% \textwidth=532pt
% \oddsidemargin=-40pt \evensidemargin\oddsidemargin
% \textheight 682pt
% %=======================================================================

\begin{document}

\title{Variables and their Standard Uses}
\author{Josh Bowles}

\maketitle 

\section{General Mathematics}
Mathematics has many conventions; below is general subset of the more common variables. Many of the symbols also have other conventional uses, for which they might stand for a constant or a specific function rather than a variable. I am not too rigorous in applying a strict ontology or classification here.\\

 \begin{center}{\bf Mathematical Conventions}\end{center}
\ex $a_i$ is often used to denote a term of a sequence.
\ex $a, b, c, d$ (sometimes extended to $e$ and $f$) usually play similar roles or are made to represent parallel notions in a mathematical context. They often represent constants.
The coefficients in an equation, for example the general expression of a polynomial or a Diophantine equation are often $a, b, c, d, e, f$.
\ex $e$ also stands for Euler's number: the base of an exponential function  $= 2.7182818\ldots$.
\ex $f, g$ (sometimes $h$) commonly denote functions.
\ex $i, j, k$ are often used as subscripts or index variables; this is also true in syntax and formal semantics.($i$ can also stand for the imaginary number $i = \sqrt{-1}$; conversely $i^2 = -1$). 
\ex $l, w$ are often used to represent the length and width of a figure.
\ex $m, n$ usually denote integers and usually play similar roles or are made to represent parallel notions in a mathematical context.
\ex $n$ typically denotes a count of objects, or, in statistics, the number of individuals.
\ex $p, q$ often represent prime numbers, or relatively prime numbers; in logic they typically represent propositional variables.
\ex $p, q, r$ usually play similar roles or are made to represent parallel notions in a mathematical context.
\ex $r$ often denotes a remainder or modulus.
\ex $r, s, t$ usually play similar roles or are made to represent parallel notions in a mathematical context.
\ex $u, v$ usually play similar roles or are made to represent parallel notions in a mathematical context, such as denoting a vertex (graph theory).
\ex $w, x, y, z$ usually play similar roles or are made to represent parallel notions in a mathematical context, such as representing unknowns in an equation.
\ex $x, y, z$ correspond to the three Cartesian axes. In many two-dimensional cases,$y$ will be expressed in terms of $x$; if a third dimension is added, $z$ is expressed in terms of $x$ and $y$.
\ex $z$ is a common variable for a complex number.
\ex $\alpha, \beta, \gamma, \theta, \varphi$ commonly denote angle measures.
\ex $\epsilon$ usually represents an arbitrarily small positive number.
\ex $\lambda$ is used for eigenvalues.
\ex $\delta$ often denotes a sum, or the standard deviation in a statistical context.
\xe \xe \xe \xe \xe \xe \xe \xe \xe \xe \xe \xe \xe \xe \xe \xe \xe \xe \xe \xe 

\section{Linguistics: Syntax and Formal Semantics}
Many of the variables and their standard conventional uses in linguistics come from mathematics and/or mathematical logic. This is largely because they were simply adopted wholesale in the early days and perhaps only later acquired more specific and linguistically relevant meanings. 

% This section will be split into two subsections: the first for syntax and semantics---two fields that borrowed heavily from the older and more advanced fields of mathemtics and (mathematical) logic; the second section is for phonetics. One could argue that the International Phonetic Alphabet is a standardized set of variables and/or constants meant to symbolize or represent acoustic phenomena. Whether or not you buy this argument, the alphabet is interesting and I will show it below.

 \begin{center}{\bf Linguistics Conventions}\end{center}
\ex $A$ and $A\text{-bar}$ (variants for latter include $ \bar{A}, \ A^{\prime}$) denote a specific syntactic tree position correlated with an argument, subject, or agent.
% \ex $a_i$ is often used to denote a term of a sequence.
% \ex $a, b, c, d$ (sometimes extended to $e$ and $f$) usually play similar roles or are made to represent parallel notions in a mathematical context. They often represent constants.
% The coefficients in an equation, for example the general expression of a polynomial or a Diophantine equation are often $a, b, c, d, e, f$.
\ex $e$ stands for an empty category.
\ex $i$ is often used as a subscript to track coreference between to two constituents.
\ex \wh{} and \Wh{} are used for a specific subclass of interrogatives; originally adopted as shorthand for English words (for example, but not neessarily limited to \textsl{what, where, who, when, how}). These variables presently refer to this special subclass of interrogatives in all languages, despite orthography
% \ex $f, g$ (sometimes $h$) commonly denote functions.
% \ex $i, j, k$ are often used as subscripts or index variables; this is also true in syntax and formal semantics.($i$ can also stand for the imaginary number $i = \sqrt{-1}$; conversely $i^2 = -1$). 
% \ex $l, w$ are often used to represent the length and width of a figure.
% \ex $m, n$ usually denote integers and usually play similar roles or are made to represent parallel notions in a mathematical context.
% \ex $n$ typically denotes a count of objects, or, in statistics, the number of individuals.
% \ex $p, q$ often represent prime numbers, or relatively prime numbers; in logic they typically represent propositional variables.
% \ex $p, q, r$ usually play similar roles or are made to represent parallel notions in a mathematical context.
% \ex $r$ often denotes a remainder or modulus.
% \ex $r, s, t$ usually play similar roles or are made to represent parallel notions in a mathematical context.
% \ex $u, v$ usually play similar roles or are made to represent parallel notions in a mathematical context, such as denoting a vertex (graph theory).
.  
% \ex $w, x, y, z$ usually play similar roles or are made to represent parallel notions in a mathematical context, such as representing unknowns in an equation.
% \ex $x, y, z$ correspond to the three Cartesian axes. In many two-dimensional cases,$y$ will be expressed in terms of $x$; if a third dimension is added, $z$ is expressed in terms of $x$ and $y$.
% \ex $z$ is a common variable for a complex number.
% \ex $\alpha, \beta, \gamma, \theta, \varphi$ commonly denote angle measures.
% \ex $\epsilon$ usually represents an arbitrarily small positive number.
% \ex $\lambda$ is used for eigenvalues.
% \ex $\delta$ often denotes a sum, or the standard deviation in a statistical context.
% \xe \xe \xe \xe \xe \xe \xe \xe \xe \xe \xe \xe \xe \xe \xe \xe 
\xe \xe \xe \xe

\newpage
\section{\LaTeX{} Symbols and their commands}

\begin{table}
\begin{tabular}{*8l}
\X\alpha        &\X\theta       &\X o           &\X\tau         \\
\X\beta         &\X\vartheta    &\X\pi          &\X\upsilon     \\
\X\gamma        &\X\gamma       &\X\varpi       &\X\phi         \\
\X\delta        &\X\kappa       &\X\rho         &\X\varphi      \\
\X\epsilon      &\X\lambda      &\X\varrho      &\X\chi         \\
\X\varepsilon   &\X\mu          &\X\sigma       &\X\psi         \\
\X\zeta         &\X\nu          &\X\varsigma    &\X\omega       \\
\X\eta          &\X\xi                                          \\
                                                                \\
\X\Gamma        &\X\Lambda      &\X\Sigma       &\X\Psi         \\
\X\Delta        &\X\Xi          &\X\Upsilon     &\X\Omega       \\
\X\Theta        &\X\Pi          &\X\Phi
\end{tabular}
\caption{Greek Letters}\label{greek}
\end{table}

\begin{table}
\begin{tabular}{*8l}
\X\pm           &\X\cap         &\X\diamond             &\X\oplus     \\
\X\mp           &\X\cup         &\X\bigtriangleup       &\X\ominus    \\
\X\times        &\X\uplus       &\X\bigtriangledown     &\X\otimes    \\
\X\div          &\X\sqcap       &\X\triangleleft        &\X\oslash    \\
\X\ast          &\X\sqcup       &\X\triangleright       &\X\odot      \\
\X\star         &\X\vee         &\X\lhd$^b$             &\X\bigcirc   \\
\X\circ         &\X\wedge       &\X\rhd$^b$             &\X\dagger    \\
\X\bullet       &\X\setminus    &\X\unlhd$^b$           &\X\ddagger   \\
\X\cdot         &\X\wr          &\X\unrhd$^b$           &\X\amalg     \\
\X+             &\X-
\end{tabular}

$^b$ Not predefined in a format based on {\tt basefont.tex}.
     Use one of the style options\\
     {\tt oldlfont}, {\tt newlfont}, {\tt amsfonts} or {\tt amssymb}.

\caption{Binary Operation Symbols}\label{bin}
\end{table}


\begin{table}
\begin{tabular}{*8l}
\X\leq          &\X\geq         &\X\equiv       &\X\models      \\
\X\prec         &\X\succ        &\X\sim         &\X\perp        \\
\X\preceq       &\X\succeq      &\X\simeq       &\X\mid         \\
\X\ll           &\X\gg          &\X\asymp       &\X\parallel    \\
\X\subset       &\X\supset      &\X\approx      &\X\bowtie      \\
\X\subseteq     &\X\supseteq    &\X\cong        &\X\Join$^b$    \\
\X\sqsubset$^b$ &\X\sqsupset$^b$&\X\neq         &\X\smile       \\
\X\sqsubseteq   &\X\sqsupseteq  &\X\doteq       &\X\frown       \\
\X\in           &\X\ni          &\X\propto      &\X=            \\
\X\vdash        &\X\dashv       &\X<            &\X>            \\
\X:
\end{tabular}

$^b$ Not predefined in a format based on {\tt basefont.tex}.
     Use one of the style options\\
     {\tt oldlfont}, {\tt newlfont}, {\tt amsfonts} or {\tt amssymb}.

\caption{Relation Symbols}\label{rel}
\end{table}

\begin{table}
\begin{tabular}{*{5}{lp{3.2em}}}
\X,     &\X;    &\X\colon       &\X\ldotp       &\X\cdotp
\end{tabular}
\caption{Punctuation Symbols}\label{punct}
\end{table}

\begin{table}
\begin{tabular}{*6l}
\X\leftarrow            &\X\longleftarrow       &\X\uparrow     \\
\X\Leftarrow            &\X\Longleftarrow       &\X\Uparrow     \\
\X\rightarrow           &\X\longrightarrow      &\X\downarrow   \\
\X\Rightarrow           &\X\Longrightarrow      &\X\Downarrow   \\
\X\leftrightarrow       &\X\longleftrightarrow  &\X\updownarrow \\
\X\Leftrightarrow       &\X\Longleftrightarrow  &\X\Updownarrow \\
\X\mapsto               &\X\longmapsto          &\X\nearrow     \\
\X\hookleftarrow        &\X\hookrightarrow      &\X\searrow     \\
\X\leftharpoonup        &\X\rightharpoonup      &\X\swarrow     \\
\X\leftharpoondown      &\X\rightharpoondown    &\X\nwarrow     \\
\X\rightleftharpoons    &\X\leadsto$^b$
\end{tabular}

$^b$ Not predefined in a format based on {\tt basefont.tex}.
     Use one of the style options\\
     {\tt oldlfont}, {\tt newlfont}, {\tt amsfonts} or {\tt amssymb}.

\caption{Arrow Symbols}
\end{table}

\begin{table}
\begin{tabular}{*8l}
\X\ldots        &\X\cdots       &\X\vdots       &\X\ddots       \\
\X\aleph        &\X\prime       &\X\forall      &\X\infty       \\
\X\hbar         &\X\emptyset    &\X\exists      &\X\Box$^b$     \\
\X\imath        &\X\nabla       &\X\neg         &\X\Diamond$^b$ \\
\X\jmath        &\X\surd        &\X\flat        &\X\triangle    \\
\X\ell          &\X\top         &\X\natural     &\X\clubsuit    \\
\X\wp           &\X\bot         &\X\sharp       &\X\diamondsuit \\
\X\Re           &\X\|           &\X\backslash   &\X\heartsuit   \\
\X\Im           &\X\angle       &\X\partial     &\X\spadesuit   \\
\X\mho$^b$      &\X.            &\X|
\end{tabular}

$^b$ Not predefined in a format based on {\tt basefont.tex}.
     Use one of the style options\\
     {\tt oldlfont}, {\tt newlfont}, {\tt amsfonts} or {\tt amssymb}.

\caption{Miscellaneous Symbols}\label{ord}
\end{table}

\begin{table}
\begin{tabular}{*6l}
\X\sum          &\X\bigcap      &\X\bigodot     \\
\X\prod         &\X\bigcup      &\X\bigotimes   \\
\X\coprod       &\X\bigsqcup    &\X\bigoplus    \\
\X\int          &\X\bigvee      &\X\biguplus    \\
\X\oint         &\X\bigwedge
\end{tabular}
\caption{Variable-sized  Symbols}\label{op}
\end{table}


\begin{table}
\begin{tabular}{*8l}
\Z\arccos &\Z\cos  &\Z\csc &\Z\exp &
           \Z\ker    &\Z\limsup &\Z\min &\Z\sinh \\
\Z\arcsin &\Z\cosh &\Z\deg &\Z\gcd &
           \Z\lg     &\Z\ln     &\Z\Pr  &\Z\sup  \\
\Z\arctan &\Z\cot  &\Z\det &\Z\hom &
           \Z\lim    &\Z\log    &\Z\sec &\Z\tan  \\
\Z\arg    &\Z\coth &\Z\dim &\Z\inf &
           \Z\liminf &\Z\max    &\Z\sin &\Z\tanh
\end{tabular}
\caption{Log-like Symbols}\label{log}
\end{table}


\begin{table}
\begin{tabular}{*8l}
\X(             &\X)            &\X\uparrow     &\X\Uparrow     \\
\X[             &\X]            &\X\downarrow   &\X\Downarrow   \\
\X\{            &\X\}           &\X\updownarrow &\X\Updownarrow \\
\X\lfloor       &\X\rfloor      &\X\lceil       &\X\rceil       \\
\X\langle       &\X\rangle      &\X/            &\X\backslash   \\
\X|             &\X\|
\end{tabular}
\caption{Delimiters\label{dels}}
\end{table}

\begin{table}
\begin{tabular}{*8l}
\Y\rmoustache&  \Y\lmoustache&  \Y\rgroup&      \Y\lgroup\\[5pt]
\Y\arrowvert&   \Y\Arrowvert&   \Y\bracevert
\end{tabular}
\caption{Large Delimiters\label{ldels}}
\end{table}

\begin{table}
\begin{tabular}{*{10}l}
\W\hat{a}     &\W\acute{a}  &\W\bar{a}    &\W\dot{a}    &\W\breve{a}\\
\W\check{a}   &\W\grave{a}  &\W\vec{a}    &\W\ddot{a}   &\W\tilde{a}\\
\end{tabular}
\caption{Math mode accents}\label{accent}
\end{table}

\begin{table}
\begin{tabular}{*4l}
\W\widetilde{abc}       &\W\widehat{abc}                        \\
\W\overleftarrow{abc}   &\W\overrightarrow{abc}                 \\
\W\overline{abc}        &\W\underline{abc}                      \\
\W\overbrace{abc}       &\W\underbrace{abc}                     \\[5pt]
\W\sqrt{abc}            &$\sqrt[n]{abc}$&\verb|\sqrt[n]{abc}|   \\
$f'$&\verb|f'|          &$\frac{abc}{xyz}$&\verb|\frac{abc}{xyz}|
\end{tabular}
\caption{Some other constructions}\label{other}
\end{table}






\end{document}
