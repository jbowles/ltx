\documentclass{article}
\usepackage{jbpaperpackages}
\usepackage{jbpapercommands}

\begin{document}

\title{Statement of Research \\ 
\small Marsden Grant for PhD Research:\\
Morphology of English---Laurie Bauer,\\
 University of Victoria, Wellington}

\author{Joshua Bowles}

\maketitle


\section{Background}



\section{Research}
\textsc{{\bf Proposal Title:}} {\sl Word Formation, Productivity, and the Construction of Neologistic and Idiomatic Words Across Varieties of English}

\begin{itemize}
\item[] \begin{center}\textsc{Description of Proposal}\end{center}
\item  Use any and all avialable corpora for varieties of English (New Zealand, American, British, Australian; possibly older forms of English: old English, Chaucerian, Shakesperian).
\item  Using successful models of word formation to compare possible with actual tokens/types of idiomatic and neologistic words.
\item  Applying current techniques for measuring productivity to get some perspective on the productivity of general word formation in English varieties and, specifically, those types of word formation that produce idiomatic and/or neologistic words.
\item  Use the results to propose applications to predicting idiomatic or neologistic words, increase precision in morpholgical parsing of new forms, and general development for other tasks relevant to computation and/or processing of idiomatic and neologistic word forms.
\end{itemize}

\begin{example}\label{example1} Given the$P-value$\footnote{$P = \frac{n_1^{aff}}{N^{aff}}$, where P equals the number of hapaxes $n_1$ with a given affix {\sl aff} divided by total number of tokens $N$ with affix {\sl aff}.} of a morpheme we might be able to predict its probable use for a new idiomatic form. \cite{plagproductive04} demonstrates the P value for \textsl{-wise}, showing that it is comparatively higher than other forms such as \textsl{-tion}. Based on this, we might expect a new idiom or neologism to be formed from \textsl{-wise} rather than \textsl{-tion}, suggesting a prediction for (\ref{for}) over (\ref{tion}). We might even predict (\ref{for}) over (\ref{shly}), but this case is far from being even suggestively clear.
\begin{exe}
\ex She walked into the room \textsl{front-wise}.\label{for}
\ex She walked into the room with alot of \textsl{frontion}.\label{tion}
\ex She walked into the room \textsl{frontishly}.\label{shly}
\end{exe}
\end{example}

In other words, we might predict, dependent on $P-value$, an adverbial formation over the nominal formation in this case. Clearly things are more complicated than I have demonstrated, but Example (\ref{example1}) illustrates the general idea.    


% \begin{itemize}
% \item  Inflection and polysemy in the use of verbs of perception across varieties of English.
% \item  Ex: see, hear, think, infer, believe, feel, know, et cetera.
% \item  Use any and all avialable corpora for varieties of English (American, New Zealand, British, Australian,and possibly older forms of English: old English, Chaucerian, and Shakesperian).
% \item  \ldots
% \item  This work originates from my MA thesis.
% \end{itemize}
% 
% 
% 
% 
% DISCUSSION:
% are guiding principles in the selection of probable words from the domain of possible words?? Are these guiding principles sociolinguistic in nature, and/or is there a way to capture this formally? That is, if we assume outright that historical changes in English are arbitrary and random, then we should expect to find so systematicity. This is not the case. Yet, appeal is often made either to apparent arbitrariness or to a pre-selected theoretical view when puzzles or problems present themselves (e.g., how to segment 'defer', 'recieve', 'refer'; zero-derivation; allomorphy; polysemy). I tend to apply a more optimistic attitude: given good empirical data and a willingness to be flexible in theoretical committment, apparent arbitrariness in language systems will eventually reveal some kind of order. That is, what appears to have no guiding principle of organization typically reveals order when we look deeper. And this has been the case in linguistics for at least the last century: the deeper we look particlar phenomenon the more systematicity we find.
% ALTERNATIVES:
% \begin{itemize}
% 
% \item BACKGROUND: has typically been in morphosyntax of native american indian languages, with MA thesis on agreement in Tuyuca under the direction of Lyle Campbell.
% 
% \item Word formation and Productivity in general
% 
% \item inflectional morpholgy of evidential verbs (see-saw, hear-heard) and whether or not there are inflectional restraints on possible evidential uses/interpretations.
% 
% \item  nominal classifiers/classification and compounding in English---does it even exist? ( -berry, -way, ...)
% 
% \item  Word formation and idioms; 
% 
% \item Measures of overall morphological productivty
% 
% \item  Using successful word formation models to make segmentation prediction better for both free and bound roots;  
% 
% \item  Successful models of productivity and word formation for making stemmers better in both inflectional and derivational cases;
% \end{itemize}

\bibliography{myrefs}
\end{document}
