% Operational: $0,000 Foot Perspectives'
% Author: Josh Bowles
\documentclass[10pt]{article}
\usepackage{tikz}
%%%<
\usepackage{verbatim}
\usepackage[active,tightpage]{preview}
\PreviewEnvironment{tikzpicture}
\setlength\PreviewBorder{100pt}%
%%%>
\begin{comment}
:Title: Operational: 40,000 Foot Perspective
:Tags: Systems
:Author: Josh Bowles
\end{comment}
\usetikzlibrary{positioning,decorations.pathreplacing,shapes}
\usepackage[english]{babel}
\usepackage{microtype}
\usepackage[hmargin=1.5cm,vmargin=1cm]{geometry}
\usepackage{amsmath}
\DeclareMathOperator{\p}{p}
\newcommand*{\sys}{\text{sys}}
\newcommand*{\testp}{\text{test}+}
\begin{document}
\begin{tikzpicture}[%
  % common options for blocks:
  block/.style = {draw, fill=blue!30, align=center, anchor=west,
              minimum height=0.65cm, inner sep=0},
  % common options for the circles:
  ball/.style = {circle, draw, align=center, anchor=north, inner sep=0}]

% text explaining how p{SQL} and p{NoSQL} behave as they
% pass through the sieves
\node[anchor=west,text width=12cm] (note1) at (-1,3) {\huge{Operational Sketch: 40,000 foot view.}};
%Layers indicate areas of focus.\\ \sl{Time frame: 1 year fully operational}

% 	API LAYER
\node[inner sep=0,anchor=east,text width=3.3cm] (note1) at (1.2,-1.5) {
   \textbf{API Layer}};

% Circle for API
\node[ball,text width=2.5cm,fill=gray!50] (all) at (6,0) {API\\ From world (twitter, facebook, ...)};

%     PERSISTENCE LAYER
\node[inner sep=0,anchor=east,text width=3.3cm] (note1) at (1.2,-4) {
   \textbf{Persistence Layer}};

% two circles showing split
\node[ball,fill=gray!10,text width=2.3cm,anchor=base] (pcan) at (3.5,-6) {Batch Processing};
\node[ball,fill=gray!10,text width=2.3cm,anchor=base] (pncan) at (8.5,-6)
   {Transactional Processing};

% arrows showing split from API to batch & SQL
\draw[->,thick,draw=purple!80] (all.south) to [out=270,in=90] (pcan.north);
\draw[->,thick,draw=purple!80] (all.south) to [out=270,in=90] (pncan.100);


% transition split from API
\node[anchor=north,text width=10cm,inner sep=.05cm,align=center,fill=white]
  (why1) at (6,-3.7) {Persistence splits to transactional and batch};


% draw the WEBSOCKET PUBLISHER
\node[block,anchor=north,text width=9.0cm,fill=green!20] (tray1) at
   (6,-8.8) {Web Socket Publisher (colored nodes are subscribers)};

% Background Jobs
%\node[anchor=north,text width=10cm,inner sep=.05cm,align=center,fill=white]
 %(why1) at (6,-8.2) {Background Job Processing};

%     PUB SUB LAYER
\node[inner sep=0,anchor=east,text width=3.3cm] (note1) at (1.2,-9) {
   \textbf{Pub-Sub Layer}};

% text explaining Pub Sub
% pass through the sieves
\node[anchor=west,text width=8cm] (note1) at (-6,-10.6) {Web socket Publisher waits for socket subscribers. Colored nodes are subscribers. Overlapping similar colored nodes represent sockets with similar content.};

%     DESIGN LAYER
\node[inner sep=0,anchor=east,text width=3.3cm] (note1) at (1.2,-12) {
   \textbf{Design Layer}};

% arrows showing the circles passing through the sockets
\draw[->,thick,draw=purple!80] (3.5,-7.3) -- (3.5,-8.6);
\draw[->,thick,draw=purple!80] (8.5,-7.3) -- (8.5,-8.6);
\draw[<->,thick,draw=black!100] (4.4,-7) -- (7.6,-7);

% websocket nodes
\node[ball,text width=1.2cm,fill=red!100] (can) at (7,-11.5) {A.1};
\node[ball,text width=1.2cm,fill=red!50] (can2) at (8,-11.5) {A.2};
\node[ball,text width=1.2cm,fill=yellow!50] (can3) at (9.9,-11.5) {B.1};
\node[ball,text width=1.2cm,fill=yellow!20] (can4) at (10.9,-11.5) {B.2};
\node[ball,text width=1.2cm,fill=yellow!100] (can5) at (11.9,-11.5) {B.3};
\node[ball,text width=1.2cm,fill=blue!20] (can6) at (1.2,-11.5) {C.1};
\node[ball,text width=1.2cm,fill=blue!100] (can7) at (2.2,-11.5) {C.2};
\node[ball,text width=1.2cm,fill=blue!50] (can8) at (3.2,-11.5) {C.3};
\node[ball,text width=1.2cm,fill=green!50] (can9) at (5,-11.5) {D.1};


% arrows showing the output of the sieves formed the fraction
\draw[->,thick,draw=red!100] (tray1.south) to [out=0,in=75] (can);
\draw[->,thick,draw=red!50] (tray1.south) to [out=0,in=75] (can2);
\draw[->,thick,draw=yellow!50] (tray1.south) to [out=0,in=75] (can3);
\draw[->,thick,draw=yellow!20] (tray1.south) to [out=0,in=75] (can4);
\draw[->,thick,draw=yellow!100] (tray1.south) to [out=0,in=75] (can5);
\draw[->,thick,draw=blue!20] (tray1.south) to [out=0,in=75] (can6);
\draw[->,thick,draw=blue!100] (tray1.south) to [out=0,in=75] (can7);
\draw[->,thick,draw=blue!50] (tray1.south) to [out=0,in=75] (can8);
\draw[->,thick,draw=green!50] (tray1.south) to [out=0,in=75] (can9);

\end{tikzpicture}
\end{document}
