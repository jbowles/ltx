\documentclass{article}
\usepackage{paperp,paperc}

\begin{document}

\title{Recursion: Review}


\maketitle

\begin{document}
\section{Introduction}
Reveiwing a book about recursion in linguistics is almsot like reviewing the last 60 years of linguistics. Both Bloomfield and Sapir tried to introduce formalizations into their linguistic theorizing, as did Harris, Bar-Hillel, and many of their colleagues. Chomsky loosely adopted Post's ideas and has been informally alluding to them since then. One particularly flourishing domain of research in this latter line has been the `biolingusitic' approach to Merge (Uriagereka, Boeckx, Medeiros, Soscen, xxxx).  Two basic lines of criticism about recursion have emerged: (i) Joshi pointed out long ago that contextual dependency is a serious problem and that indexicalty in natural language is an underestimated phenomenon and implied that any theory would be insufficeint without a model that could account for these problems; (ii) Jackendoff and xxx more recently point out that `recursion' is too loose a term and we need to be more specific. Everrett branches out from both (i) and (ii). He has pointed out that `recursion' is too vague a term while arguing that culture (as a type of contextual dependency) can determine the `shape' of syntactic structure. I provide this synopsis becusae the book under review here is the result of direct responses to this `history' of recursion. The book grew out of presentations at a conference organized by Everrett (of which there are now a few). Another semi-relevant note: most of the authors are old enough to have participated in, or been aware of, the deveopment of the ideas of linguistic `recursion.' This is semi-relevant because one cannot discount the importance that socio-cultural issues can have on the direction of academic research, and in this case the personal and cultural stake in linguistic research seems to have played a significant role. I will not discuss such soci-cultural issues becuase that is not the goal here, but I think it is worth noting (and for those of you who have been paying attention to current `debates' on this topic, the personal dimension is not lost on you).
There is probably no better way to konw someone's linguistic-theoretical bias then through asking them questions about what they think of recursion in human language.

Probably the best way to reveal one's theoretical biases is by way of comments on recursion. In some sense, the (linguistic) field has become so polarized on this issue that a few informal comments on the characterization of recursion in natural language can unfortunately be enough to lay the ground for an unfair characterization of one's view of human language on the whole. At times, one feels that the careful treatment of scientific topics we aspire to has been replaced by a form of political rhetoric that makes sweeping generalizations and employs overused cliches to win sophistic favor at any cost (and you should hope that I am exaggerating here).

I do my best not to fall into a blind adherence to my own theoretical persuasions in reviewing this work; but I cannot promise that interpretation of my comments won't fall into blind adherence of your own position. (And at this point, if you feel I am being overly dramatic about the sociological nature of the debate about recursion, then you should consider yourself a blissfully unaware scholar and try to keep it that way). The truth is that this is `charged' topic and many people have a large personal stake in advocating one position or the other.

I will say this: painting with a broad brush we can split the debate about recursion into two camps: believers and non-believers. That is, recursive procedures may occur despite the lack of overt structural evidence for them. The status of relative clause embedding or recursive nominalization is rarely disputed where it is obvious. Instead, disputes arise over (a) the nature of underdetermined data or data that depend strongly on interpretation due to the lack of native speaker linguistic analysis, and (b) the hypothesis that recursive Merge as a binary combinatorial procedure for compositionality in human language capacity is a `first principle.' Both (a) and (b) are only contested where claims for the universality of their application is asserted. In other words, the assertion of Merge as a 'first principle' is intended to apply to all human beings and consequently all human languages. This assertion is generally countered by the lack of justification and in some cases the the methodological disagreement that linguistics is not physics and there is reason why we should expect to find any `first principles.' Evolutionary biological counter-claims also exist; they seek to show that given a standard evolutionary framework (neo-Darwinian) recursive Merge is just not the sort of thing we expect to see. This neo-Darwinian counter-argument is typically met with a neo-Thompsonian (D'arcy Thomspon) claim that many varities of iterated structures appear in nature and that Merge yields just the sort of iterated patterns one finds in biology, and therefore, it is just as likely to be a product of biology (and physics) as anything else one finds in nauture.

Much of this debate has been on the biology and culture frontier (see also Lassersohn for interestign culture-biology theories).

\subsection{Some Stuff}
It strikes me as strange that linguists committed to a ``recursive theory'' like Merge (i.e., that recursively computable  procedures are at the core of the engine of linguistic compositionality) have not done more to construct concrete and computable models of their linguistic structures in programming languages well-suited for generative modelling. That is, some programming languages have thier foundation in the lambda calculus and provide explicit formalization of lambda abstraction; this provides a nice real-world tool for modeling semantic compositionality. Or, for example, Unger's (2010) thesis models the syntax-semantics of displacement and scope in the fucntional programming language Haskell; widely used by many linguists and computer scientists. In other words, I find it strange that theorists committed to `recursion' have not done more to construct real-world models of linguistic compositionality; even Galileo built concrete models. And, I believe, building such models would help make assumptions more explicit and could potentially draw the lines between different linguistic theories in a way that helps calm the fever of rhetoric about recursion. Overall, I find it strange that linguists even fight about such issues: competing theories are good for investigation; what is so wrong with taking a different approach (i.e., one that has a different set of assumptions about the nature of human language capacity)? Proverbially, we are all blindly feeling at the elephant; arguing about our approaches (and in some cases name-calling) doesn't get us any closer to a unified explanation.

Having implied a very pratical approach to lingusitic theorizing I should mention that my roots in linguistics stretch back to philosophy and I have thought long and hard about the nature of lingusitic methodology. But I have also spent many hours analyzing the structure of languages foreign to common linguistic theories and have put alot of energy into describing grammatical parts of `exotic' languages in as neutral a way as possible, despite my theoretical biases. My commments are motivated by a desire to find the truth, not advocate a position.

I have to admit that I was very taken with Lauri Kartunnen's parable of the Knight and the Princess in his CL address: I agree with Kartunnen that at times theoretical linguists get too caught up in questions instead of solutions. If one wants to measure linguistic methodology by other sciences then one needs also to admit that solutions that work drive progress just as much as broad theories that `might' work. One needs to admit that implementation here-and-now is just as important as long-range theorizing. Kartunnen states that many of the problems in morphology and syntax are essentially solved, at least in a computational domain. One can't just ignore these results. Neither can one ignore the fact that the computer is a tool for investigation and that in one way or another computational tools have always been a part of linguisitcs; it is time for linguistics departments to offer core courses in training students to use these tools---no matter your theoretical persuasion.

Whatever  you feel about Chomsky and Everret you must admit that they have both played their part in bringing to the linguistic public issues of foundational concern to our field. They have also both played a role in bringing linguistics to a popular audience through their own advocacy of socio-cultural issues; ironically they emerge as duelists instead of collaborating advocates. Their role of duelling linguists is perhaps facilitated by the larger socio-cultural context of 20th century ``linguistic wars.'' I mention these peripheral issues because they underlie the motivations for which this book was produced: a conference organized by Everret to argue against the notion of recursion as a primitive for (human) natural language capacity outside of the influence of cultural constaints.

Before reviewing the book it should be noted that debates and disagreements about the use, application, and defintion of recursion is by no means limited to linguistics. Early 20th century logicians struggled with it, in particular Stephen Cole Kleene, Kurt Godel, Emil Post, and Alonzo Church. And in modern computer science Robert Soare has pointed out problems with use of the term `recursion.' Pullum and Schulz recognize this also; it is time for rest of us to grapple with these issues.

If one takes a broad look at our field oen has to admit that there is special ammount of chaos that has dominated it. Old approaches are stricken from the record, only to pop up again usually unkown to its advocates. Both Bloomfield and Sapir recognized the need for formalization; Harris and Bar-Hillel did more than recognize, the latter a pionering figure in machine translation; concepts from early generative semantics is now being reviewed by some; concepts from people like Carnap or Tarski have always had a strange relationship to formal lingusitics; statistical an dprobablistic approaches are gaining ground while discrete formalists still stick their guns. In terms of Kuhnian paradigms it would be a fair assessment to say that the field in general is still under quit a bit of flux; although many subareas are finding stability in formalizing techniques and mathematizing approaches.



\section{The Book}

\section{Discussion}
Here I will touch on only a few issues. I try to steer clear of common (and un-enlightening) labels such as Minimalist or Functionalist and avoid as much as posible the rhetorical vitriol that has come to permeate the linguistic `debate' on recursion.

When I introduce the major concepts of syntax to undergraduates in Science and Technology composition courses I make a special note of highlighting the fact that what we know about capital ``L'' Language (i.e., the human capacity to use human languges) is a complex chain of inferences made from highly variable empirical data. How we collect such data and the kinds of data used can have a dramatic impact on the hypotheses that will emerge. In other words, we cannot see Language nro can we ``put it under the microscope.'' What we must do is draw inferences from observed language (lowercase ``l'') data. Such data can only at best represent the effect of the mechanisms of Language given a particular language's structure.

Whether or not the nature of the actual mechanisms of Language can be recovered, and what the origin of such mechanisms is, is a deep linguistic puzzle (and perhaps a genuine mystery). Scholars such as Everret and Levinson place such origins in the nature and history of human cultures; others, such as Boeckx or Chomsky, place such origins in the formal patterns of a neo-Thompsonian (D'arcy Thompson) biology. Each split has its nuances, but the overall difference boils down to the methodological use (or non use) of a first-principles approach. To clarify, a formal biological approach emphasizes recursion as a first-principle for the origin of language. (But of course, one can find competing explanations more along the lines of an adaptionist approach utilizing notions such as cultural evolution; see the work of Ladd \& Dediu, Kinsella. Additionally, other formal approaches do not necessitate recursion as a first principle; see Pullum and Shulz \& Pullum). The cultural approach does not seem to appeal a first-principles but instead proposes pragmatic maxims---dependent on cultural constraints---that shape the overall structure of the semantic-syntactic-phonologic phenomena.

It is not my intention to represent the formal-biological approach as bottom-up and the cultural constriants approach as top-down; such broad labels don't clarify the issues. But in a certain sense these vague labels can hint to us that both approaches are not exclusive: they simply tackle the enormous difficulty of linguistic research from different angles. And, I argue, current linguistic research---in practice---rarely conforms to the strict boundary line between the `formal-biological' and `cultural constaints' approaches. And when an advocate of one approach crosses into the ``territory'' of the other it should not be treated as an opportunity to attack, but to engage in what science does best: the dialectic of competing approaches.

I have never understood why the two approaches need to treated as exclusive to each other: culture or formal-biology. I don't see why linguists generally can't formulate zones of convergence between, treating the ``or'' as inclusive and dependent on individual research foci.

One thing seems fairly clear: the incorporation of statistical and probabilistic
applications to recursive linguistic structure is a ripe field. To ground a
theory of the language faculty on first-principles embedded in physical reality
\textsl{without} allowing statistical or probabilistic models is to ignore the
wealth of success such models have had in other areas of science and presupposes
the unfounded claim that the language faculty is somehow different than other
almost all other physical phenomena (given that most physical phenomena can be
modeled by statistical or probabilistic means). And with the langauge faculty
hypothesized to be split into narrow and broad, it is certainly no longer
execusable to ignore the modeling of linguistic phenomena through incorporating
probablistic inference to recursively computable procedures. My own feeling,
probably shared by many, is that linguistics was late to scientific party and
missed out the chance to carve its own unique formalization. Recursive syntax is
the exemplary story: adopted from mathematical logic and empirically modified to
help explain Language, and sometimes strucutres in particular languages. It was,
arguably, one of the first attempts to transform quantified formal methods into
a linguistic science ({\sl adoption of the lambda calculus by} people like
Hamblin, Montague, Lambek, and Lewis deserve credit too). But as has been
pointed out numerously (e.g., handbook of KR), many of the early 20th century
algebraic innovations in set theory, model theory, and the predicate calculus
have proven to be too rigid to account for the dynamic nature of the real world;
and why should human language be any different? Recursion, it seems, is here to
stay not only because it exemplifies the history of linguistic attempts at
formalization in the `computer age,' but because it is a such a basic
combinatorial process. However, this alone does not provide the depth of
explanation we need, and this has been pointed out over and over again.


\section{Personal}
JOSHUA BOWLES is a lecturer in the English Department at Utah Valley University
where he teaches Science and Technical writing with a foucs on the
applicability of disciplines such as formal pragmatics, artificial
intelligence, and linguistics to rhetoric and composition. Joshua has been
working on a book for laymen that surveys the general theoretical landscape of
modern linguistics. He obtained his MA in (typological and minimalist)
linguistics from the University of Utah in 2008 and hopes to soon begin PhD
research on syntax-semantics-pragmatics of context (e.g., time and space) in natural language.


\end{document}
