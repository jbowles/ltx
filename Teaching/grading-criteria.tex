\documentclass{article}
\usepackage{articlep, articlec}


\begin{document}
\title{Criteria for Papers}
\maketitle
\tableofcontents

\section{Introduction}
What follows are examples pulled off of a number of websites that have published a
set of writing criteria for students in various classes. The point here is to show how they are very similar and that such devices can be overly obvious. In this class we are taking a very intuitive approach based on Gricean Maxims---and if you are paying attention you will see how Gricean Maxims do not really differ from the comments below; except the Gricean Maxims are more compact. I also provide these comments because {\bf they can be useful}. However, {\bf we shouldn't become too dependent on them}. Trying to hit ``the target points'' for paper is a tired adn uninspired way to write. This approach seems to demoralize the dynamic, creative, and imaginative aspect to writing. 

The following points should really be read {\sl after} you have written a number of papers, {\sl not before}. Once you get into the habit of setting up the expectation of a paper before you have written a word, you can't help but try to meet that structure from the very beginning. Our approach in this class is more intuitive and simple: ``just write the freakin' thing'' --- write what you find interesting and inspiring. Then, after you have spent a good time writing, we will go over the things mentioned below.
\section{The basic explanation}
From \href{http://mendota.english.wisc.edu/~WAC/page.jsp?id=101&c_type=article&c_id=4}{University of Wisconsin}, Steve Kantrowitz (adapted from Nancy Langston)
\subsection{Structure}
Begin your paper with a brief description of the narrative, or a brief episode
from it that suggests or illustrates your thesis. Give your thesis statement,
which is a concise statement of your central argument. Then build your argument
in a series of well-structured paragraphs. Each paragraph should have a topic
sentence, and 3 to 5 sentences that clearly support that topic sentence. Each
paragraph should explain ONE idea, not 3 or 4. Each paragraph should have a
clear connection to the ONE idea, not 3 or 4. Pay attention to transitions! Each
paragraph should have a clear connection to the next. End with a strong
conclusion that explains what your thesis tells us about the era of the Civil
War.

\subsection{Analysis}
Remember that each paragraph should advance your argument. Support your thesis
with evidence from your narrative, always remembering to explain what that
evidence means. Where necessary, provide context from other course marterial,
but don’t lean too heavily on textbooks or lectures. Your analysis should offer
specific insights into aspects of this history that other course materials
describe in general terms; it may also suggest how your evidence challenges
other historian’s analyses. Without trying to make too broad a claim about the
entire Civil War, show how your narrator’s experience of change tells us
something interesting and important about the era.

\subsection{Style}
Clarity comes from knowing what you mean and saying it plainly. Don’t try to write like a writer--write like a person who wants to be understood. We will reward clear, active, powerful writing. PLEASE do not use the passive voice. Do not start sentences with “It is\ldots,” “There is\ldots” or “There are\dots.” Use active verbs. Revise your paper to remove wordiness, redundancy, passive voice, and inactive verbs. Make sure that your grammar and spelling are correct. Careless errors, especially run-ons and comma splices, WILL lower your grade. 

\paragraph{This} is an example of BAD writing: “There were changes in southern society during the war that made southerners turn their anti-government beliefs against the south.”

\paragraph{This} is an example of BETTER writing: “Many white southerners interpreted wartime taxation and conscription as the same sort of interference with southern ‘domestic relations’ that the Confederacy founders had promised to prevent.”\\

What’s the difference? In the first sentence, “There were changes” is in the passive voice and offers no specifics. What sort of changes occurred, and in what context? The passive voice allows you to evade these questions, but specificity and context are essential to good history. “Southerners” is too general; the group in question consists of many (but not all) white southerners. “Anti-government beliefs” and “the south” also lacks precision. White southerners tended to resist some forms of political authority, but not others; this dynamic shaped both the Confederate state (which was not the same thing as “the south”) and the emerging opposition to that state’s policies.

\subsection{Originality}
Although you can get a good grade (A/B) for a paper based on arguments presented in lectures or readings, “A” papers must offer more original insights and arguments. We strongly encourage you to think for yourselves, building on evidence and arguments from the course but pushing your insights further than what we cover in lectures. 


\section{Scalar Criteria}
The following represents a very small collection of the kinds of things you might have been exposed to. I have taken liberally from the internet and, unfortunately, did not document my sources very well. 
\subsection{}
From \href{http://mendota.english.wisc.edu/~WAC/page.jsp?id=101&c_type=article&c_id=4}{University of Wisconsin}, Steve Kantrowitz (adapted from Nancy Langston)
\subsubsection{The Superior Paper (A)}
\paragraph{Structure:} Your thesis is clear, insightful, original, sophisticated, even exciting. All ideas in the paper flow logically; your argument is identifiable, reasonable, and sound. You have excellent transitions. Your paragraphs have solid topic sentences, and each sentence clearly relates to that topic sentence. Your conclusion is persuasive.
\paragraph{Analysis:} You support every point with at least one example from your primary sources. You integrate quoted material into your sentences well. Your analysis is fresh and exciting, posing new ways to think of the material.
\paragraph{Style:} Your sentence structure, grammar, spelling, and citations are excellent. You have NO run-on sentences or comma splices. Your writing style is lively, active, and interesting. You use active verbs, and do not use the passive voice. You are not wordy or redundant.
\paragraph{Originality:} Your arguments show a great deal of independent insight and originality.\\


\subsubsection{The Very Good Paper (A/B)}
\paragraph{Structure:} Your thesis is clear, insightful, and original. Your argument flows logically and is sound. You may have a few unclear transitions. You end with a strong conclusion.
\paragraph{Analysis:} You give examples to support most points, and you integrate quotations into sentences. Your analysis is clear and logical, and even makes sense.
\paragraph{Style:} Your sentence structure, grammar, spelling, and citations are good. You have no more than one run-on sentence or comma-splice. Your writing style is solid and clear. You use active verbs and do not use the passive voice. You are not wordy or redundant.
Originality: Your arguments show independent thought.

\subsubsection{The Good Paper (B)}
\paragraph{Structure:} Your thesis is clear, but may not be insightful, original, or easily identified. Your argument is generally clear and appropriate, although it may wander occasionally. You may have a few unclear transitions, or paragraphs without strong topic sentences. You may end without much of a conclusion.
\paragraph{Analysis:} You give evidence to support most points, but some evidence may appear where inappropriate. Your argument usually makes sense, although some gaps in logic may exist.
\paragraph{Style:} Your writing style is clear, but not always lively, active, or interesting. You sometimes use the passive voice. You may become wordy or redundant. Your sentence structure, grammar, and spelling are strong despite occasional lapses.
Originality: You do a solid job of synthesizing course material but do not develop your own insights or conclusions.

\subsubsection{The Borderline Paper (B/C)}
\paragraph{Structure:} Your thesis may be unclear, vague, or unoriginal, and it may provide little structure for the paper. Your paper may wander, with few transitions, few topic sentences, and little logic. Your paragraphs may not be organized coherently.
\paragraph{Analysis:} You give examples to support some but not all points. Your points often lack supporting evidence, or else you use evidence inappropriately, often because there may be no clear point. Your quotations may be poorly integrated into sentences. You may give a quote, but then fail to analyze it or show how it supports your argument. Your logic may fail, or your argument may be unclear. Your end may dwindle off without a conclusion.
\paragraph{Style:} Your writing style is not always clear, active, or interesting. You use the passive voice, or become wordy or redundant. You have repeated problems in sentence structure, grammar, punctuation, citation style, or spelling. You may have several run-on sentences or comma splices.
Originality: You do a fair job of synthesizing course material but do not develop your own insights or conclusions.

\subsubsection{The ``Needs Help'' Paper (C)}
\paragraph{Structure:} Your thesis is difficult to identify, or it may be a bland restatement of an obvious point. Your structure may be unclear, often because your thesis is weak or non-existent. Your transitions are confusing and unclear. Your paragraphs show little structure. The paper is a loose collection of statements, rather than a cohesive argument.
\paragraph{Analysis:} Your examples are few or weak. You fail to support statements, and the evidence you give is poorly analyzed, poorly integrated into the paper, or simply incorrect. Your argument may be impossible to identify. Ideas may not flow at all, often because there is no argument to support.
\paragraph{Style:} Your writing style has problems in sentence structure, grammar, and diction. You have frequent major errors in citation style, punctuation, and spelling. You may have many run-on sentences and comma splices.
Originality: You do a confusing or poor job synthesizing material presented in lectures and sections, and you do not develop your own insights or conclusions. 

\subsubsection{The Bad Paper (D or F)}
A bad paper shows minimal lack of effort or comprehension. The arguments are very difficult to understand owing to major problems with mechanics, structure, and analysis. The paper has no identifiable thesis, or an incompetent thesis. It’s difficult to tell that you’ve come to class.

\subsection{}
From \href{http://isites.harvard.edu/fs/html/icb.topic58474/GradingPapers.html}{Harvard University}, Lewis Hyde, edited by Sue Lonoff, with thanks to Richard Marius's writing handbook.\\
The following remarks are intended to give you a sense of criteria for grading papers. Note that four topics recur: thesis, use of evidence, design (organization), and basic writing skills (grammar, mechanics, spelling). In courses with multiple graders or teaching fellows, it is essential that a unform grading standard be discussed and adapted by those grading students' work.

\subsubsection{The Unsatisfactory Paper.}
The D or F paper either has no thesis or else it has one that is strikingly vague, broad, or uninteresting. There is little indication that the writer understands the material being presented. The paragraphs do not hold together; ideas do not develop from sentence to sentence. This paper usually repeats the same thoughts again and again, perhaps in slightly different language but often in the same words. The D or F paper is filled with mechanical faults, errors in grammar, and errors in spelling.

\subsubsection{The C Paper.}
The C paper has a thesis, but it is vague and broad, or else it is uninteresting or obvious. It does not advance an argument that anyone might care to debate. "Henry James wrote some interesting novels." "Modern cities are interesting places."

The thesis in the C paper often hangs on some personal opinion. If the writer is a recognized authority, such an expression of personal taste may be noteworthy, but writers gain authority not merely by expressing their tastes but by justifying them. Personal opinion is often the engine that drives an argument, but opinion by itself is never sufficient. It must be defended.

The C paper rarely uses evidence well; sometimes it does not use evidence at all. Even if it has a clear and interesting thesis, a paper with insufficient supporting evidence is a C paper.

The C paper often has mechanical faults, errors in grammar and spelling, but please note: a paper without such flaws may still be a C paper.

\subsubsection{The B Paper.}
The reader of a B paper knows exactly what the author wants to say. It is well organized, it presents a worthwhile and interesting idea, and the idea is supported by sound evidence presented in a neat and orderly way. Some of the sentences may not be elegant, but they are clear, and in them thought follows naturally on thought. The paragraphs may be unwieldy now and then, but they are organized around one main idea. The reader does not have to read a paragraph two or three times to get the thought that the writer is trying to convey.

The B paper is always mechanically correct. The spelling is good, and the punctuation is accurate. Above all, the paper makes sense throughout. It has a thesis that is limited and worth arguing. It does not contain unexpected digressions, and it ends by keeping the promise to argue and inform that the writer makes in the beginning.

\subsubsection{The A Paper.}
The A paper has all the good qualities of the B paper, but in addition it is lively, well paced, interesting, even exciting. The paper has style. Everything in it seems to fit the thesis exactly. It may have a proofreading error or two, or even a misspelled word, but the reader feels that these errors are the consequence of the normal accidents all good writers encounter. Reading the paper, we can feel a mind at work. We are convinced that the writer cares for his or her ideas, and about the language that carries them.

The sure mark of an A paper is that you will find yourself telling someone else
about it.

\subsection{}
From \href{http://stenzel.ucdavis.edu/standard.htm}{UC Davis, John Stenzel}
\subsubsection{An A paper is excellent in nearly all respects:}
Ideas---An A paper is an interesting and sophisticated response to the topic. The central idea or thesis is clearly stated, worthy of development, and suitably specific; usually the thesis constitutes a thoughtful answer to a question worth asking. The paper recognizes the complexity of the topic or question, acknowledging contradictions, qualifications or limits of the thesis while sustaining logical development. Support---It uses evidence appropriately and effectively, providing sufficient and convincing support for its main ideas. If it uses outside sources, it does so with a competent controlling touch, critically evaluating as necessary, and showing clear comprehension. It appropriately defines terms and limits its scope, and cites useful illustrative examples. Organization and Coherence---An A paper has a logical structure appropriate to the subject, purpose, audience, and discipline. Usually, transitional sentences lead the reader from one idea to the next, and/or identify the logical relations between ideas and thesis. Paragraphs make clear points that support the main idea, and paragraphs demonstrate coherence and continuity. Style---An A paper shows a clear command of English prose, with words chosen for their precise meanings, and an appropriate level of specificity and sophistication. Sentence style fits the audience and purpose; sentences are varied, yet clearly structured and carefully focused--neither long and rambling nor short and choppy. Mechanics---An A paper contains few, if any, errors in spelling, punctuation, or grammar, and observes all applicable conventions of format and citation.

\subsubsection{B paper is solid in most respects:}
Ideas---B paper has a clearly stated thesis that responds appropriately to the topic. It demonstrates understanding of the question, acknowledging the central idea's complexity or significance, but it may handle the ideas in a less sophisticated and effective way. Support---The B paper offers reasons for supporting the points it makes, using varied kinds of evidence, but the evidence may need further evaluation or qualification. Connections between main ideas and evidence may need some clarifying, and definition of terms may not be smoothly accomplished, but the logic is solid. Examples do support a thesis, but development may be somewhat incomplete. Organization---The B paper demonstrates a logical progression of ideas, and offers the reader transitional links; each paragraph relates to the paper's central idea, but the connections may be less sophisticated and effective than those of the A paper. Sentences coherently support their paragraph's topic sentences. Style---The prose of a B paper is accurate and effective, but may sometimes be too general; sentences are mostly clear and well-structured, though there may be an occasional awkward or ineffective construction. Mechanics---B paper may contain a few mechanical or grammatical errors, but they do not impede understanding; format and other considerations are substantially correct and appropriate for the subject or discipline.

\subsubsection{C paper is an adequate response to the topic:}
Ideas---The C paper responds to the topic, but presents its central idea in general terms, not striking an appropriate level of specificity and precision. The paper may not offer insights beyond the most obvious, and the thesis does not engage the topic's key questions with sufficient clarity and control. The paper may restate the question unnecessarily, or may overlook important aspects. Support---If it acknowledges other views, the C paper may exhibit only a basic comprehension of source material, with some lapses in understanding. Definitions may be simply dictionary quotations, not integrated into a flow of ideas, and the relevance of examples may not be clear. C papers often inappropriately depend on unsupported opinion or personal experience, or assume that the evidence speaks for itself; there may be lapses in logic and the development may be perfunctory. Organization---The C paper may list ideas or arrange them ineffectively rather than using a logical structure; transitions are likely to be sequential (first, second, third) rather than logical links. While each paragraph relates to the central idea, the flow of ideas may not be smooth, and arrangement of sentences may occasionally be ineffective. Style---The C paper usually exhibits some vague word choice or inappropriately general terms, and though sentence structure is generally correct there may be some sentences that are wordy or unfocused or choppy. Meaning may be diffused through cumbersome constructions and may inappropriately depend on jargon or buzz-words. Mechanics---The C paper may contain some minor mechanical or grammatical errors, but they are not enough to impede understanding.

\subsubsection{D paper does not adequately respond to the assignment:}
Ideas---The paper does not have a clear central idea, or responds to the assignment in a simplistic or perfunctory way. The thesis may be vague or unrecognizable, may be too obvious to be developed effectively, or may demonstrate only a surface-level approach to the topic. Support---The D paper may show a misunderstanding of sources, or may rely too heavily on them at the expense of clear exposition; it may depend on clich\`es or overgeneralizations for support, or may offer little evidence of any kind. The D paper's examples are not convincing: it may rely on personal narrative when an essay is called for, or it may simply summarize when analysis is required. Paragraphs may be too short to do justice to the topic. Organization---D papers often have random organization, using few or inappropriate transitions; paragraphs may lack clear logical links to the central idea, and paragraph length may be inappropriate (long undivided blocks or choppy short units). D papers often contain paragraphs with little relevance to the topic, or whose relevance requires considerable authorial explanation. Style---D paper may be overly vague and abstract, or overly personal and specific, but in any case its style is inadequate for the task at hand. It usually contains frequent awkward or ungrammatical sentences, or employs ``correct'' sentences that are inappropriately simple or monotonous. Mechanics---The mechanical and grammatical errors in a D paper are severe enough or frequent enough to impede a reader's understanding. Format may be inappropriate, or may indicate neglect or misreading of instructions.

\subsubsection{F or Failing papers}
A paper may fail because its flaws exceed those allowed for D papers in any sub-category, including sentence-level competence; major and repeated deviations from accepted English usage and grammar may fail a paper. A paper may fail because it is off topic, of inappropriate length, or too full of logical or other flaws; moreover, papers that lean inappropriately hard on source material (with or without acknowledgment) also fail, as do papers whose coherence is detectable only to the writer.

\subsection{}
From \href{http://www.usm.maine.edu/eng/100APaperCPaper.htm}{University of Maine}
\subsubsection{A}
The principal characteristic of the ``A'' paper is its rich content. The information is presented in such a way that the reader feels significantly taught by the author.  The writer sustains a thoughtful, analytic argument, looking at ideas from more than one point of view, asking difficult questions and following them up with analysis.  Sometimes a paper achieves an A because a student develops a thoughtful and well-defined interpretive approach and an awareness of his or her own position in relation to the positions in the assigned readings.

An A paper must demonstrate the writer making substantial interpretive connections between the ideas of two or more texts.

It is also marked by stylistic finesse: the title and opening paragraph are engaging; the transitions are substantive rather than superficial; the phrasing is tight, fresh, and highly specific; the sentence structure is varied; and the tone contributes to the meaning of the paper. Sentence-level error must be minimal.

Often an A paper has one or two "B" or even "C" moments, but they do not significantly detract from the overall force of the paper.

Finally the "A" paper leaves the reader with a sense of having read—and being eager to reread—a complete, satisfying piece of work.

\subsubsection{B}
The ``B'' paper is significantly more than competent. It delivers substantial information—substantial in both quality and interest. The paper does everything a C essay does but offers a sustained and meaningful structure and a project that is more complex than what one finds in a C-range paper. The paper might tackle a significant contradiction, problem, or moment of connection in the readings and develop it in a sustained way.

The paper shows the student beginning to take interpretive risks, responding to the assignment and to the readings in thoughtful and distinctive ways.

The use of words in the ``B'' paper is more precise and concise than in the "C" paper.

The paper demonstrates coherence in its overall presentation: the relationships between the paper's parts are clear. The transitions between paragraphs are for the most part smooth, and the sentence structure is skillfully varied.

B papers may include ``C'' moments in otherwise well-reasoned and well-developed analyses.

Sentence-level error must be minimal.  Sentence structure is varied, with competent use of subordination.

\subsubsection{C}
The ``C'' paper is competent: it meets the assignment, has few mechanical errors, and is reasonably well organized and developed.  C papers demonstrate the student's ability to work with more than one reading and to create meaningful connections between assigned readings. 

C papers comment on and use the ideas in the readings rather than just summarizing them. 

Papers often achieve a passing grade by demonstrating one outstanding or two significant moments of analysis in an otherwise flawed or undistinguished performance.

C papers often create coherent relationships between paragraphs even if they have not developed a larger organizational structure.

In a C paper there is evidence of an emerging project—something the student wants the paper to accomplish.

A C paper has sentence-level errors under control. Although errors may appear on each page, they do not significantly impede the meaning of the essay. Sentence structure is somewhat varied and there is some use of subordination.  There are fewer than three of the following kinds of errors per page: mixed construction, fragments, verb endings.

\subsubsection{D}
This paper resembles a rough draft. It may reveal some organization, but what is presented is neither clear nor effective. It may contain the germ of some good ideas, but these are not well developed or unified. 

A D paper may do one thing really well and another not at all—for instance, it may be full of interesting ideas but entirely without formal control.  Or it may be very correct and neat but present no original ideas at all. 

A D paper may overgeneralize about the reading or depend largely on undirected summary. Or it may depend on uncritical personal response in order to avoid dealing with the reading directly.

It is unable to make a meaningful connection between two of the assigned readings. It might place quotes or other key conceptual terms from the two works side by side, implying but never analyzing or explaining the connection. It might include summaries of two or three works followed by some analysis of individual works but never sustain the analysis or show connections between the works. Alternatively, these papers sometimes attempt a series of connections that do not make much sense.

A D paper often has a significant pattern of sentence-level error, especially with sentence boundaries, verbs, and mixed construction. 

\subsubsection{F}
An F paper does not engage with the assigned readings and does not work effectively with quotations.

An F paper demonstrates a serious lack of basic reading comprehension or an inability to grasp the outline of an author's argument.

It has no coherent sense of project, little sense of the connections between paragraphs, and/or no organizational structure.

It has significant sentence-level error that makes the essay difficult to follow. A paper should not pass if the following kinds of errors occur more than once or twice a page: fragments, mixed constructions, incorrect verb endings.

It does not meet the assignment’s minimum page-length (4.5 pages for most papers in College Writing). 

\subsection{}
From \href{http://www-personal.umich.edu/~mmanty/teaching/grading.html}{University of Michigan}
\subsubsection{A/A–} Paper offers a clearly stated, interesting thesis which is supported with valid and sound arguments. The paper shows that the writer has thought about the assignment and developed his or her own ideas about it, instead of just offering minimal responses to the different components of the assignment. Interpretations of theories are sophisticated and supported with textual evidence; more than one source is considered. Writing is between good and brilliant: the organization of the paper is clear, prose is good and grammar flawless.
\subsubsection{B/B+} Paper offers a clearly stated thesis which is supported with for the most part valid and sound arguments. The paper stays on topic, considering all the relevant aspects of the assignment. Interpretations of theories are plausible and supported with textual evidence; more than one source is considered. Writing, including outline and grammar, is solid.
\subsubsection{B–} Paper offers a thesis and attempts to support it with arguments. However, the thesis is simplistic and/or the arguments weak or unconnected to the thesis. Interpretations are weak or problematic, textual evidence minimal or weak. Paper only uses one textual source. Writing and organization have problems that affect readability
\subsubsection{C/C+} Paper offers a minimal thesis and minimal or no arguments in its support. Interpretations thoroughly misguided and/or unsupported with any evidence. Writing — both at the level or paper organization and grammar — seriously problematic.
\subsubsection{D+/C–} No thesis, no arguments or no textual evidence. Organization incoherent, writing very awkward and unintelligible.
\subsubsection{D} No thesis, no arguments, no envidence. Writer has no conception of most rudimentary aspects of writing (paragraphs, outline).
\subsubsection{E} The paper displays a fundamental lack of understanding of the principles that guide scholarly endeavors. Examples include but aren’t limited to gross mistakes in citing source materials as well as significant errors in framing the paper (e.g., writing a short story instead of an essay).
\section{Other Sources}
See the following places for this material or material very similar in nature:
\pex Small List of Writing Criteria Sources
\a \href{http://www.u.arizona.edu/~sturman/syllabus/107labs/writing.html}{University of Arizona: Struman}
\a \href{http://agenvlaw.aers.psu.edu/AG301W/gradetips.htm}{Penn State Agricultural and Environmental Law}
\xe



\end{document}
