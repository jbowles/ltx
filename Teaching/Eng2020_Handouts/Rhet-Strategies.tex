\documentclass{article}
% Change "article" to "report" to get rid of page number on title page
\usepackage{amsmath,amsfonts,amsthm,amssymb}
\usepackage{linguex}
\usepackage{setspace}
\usepackage{Tabbing}
\usepackage{fancyhdr}
\usepackage{lastpage}
\usepackage{extramarks}
\usepackage{chngpage}
\usepackage{soul,color}
\usepackage{graphicx,float,wrapfig}

% In case you need to adjust margins:
\topmargin=-0.45in      %
\evensidemargin=0in     %
\oddsidemargin=0in      %
\textwidth=6.5in        %
\textheight=9.0in       %
\headsep=0.25in       

% Homework Specific Information
\newcommand{\hmwkTitle}{Rhetorical Strategies:\\ What writer's do and why}
\newcommand{\hmwkDueDate}{Feb. 24 2010}
\newcommand{\hmwkClass}{Eng 2020}
\newcommand{\hmwkClassTime}{}
\newcommand{\hmwkClassInstructor}{Joshua Bowles}
\newcommand{\hmwkAuthorName}{Your Name:}
% Setup the header and footer
\pagestyle{fancy}                                                       %
\lhead{}                                                 %
\chead{\hmwkClass\ (\hmwkClassInstructor\ \hmwkClassTime): \hmwkTitle}  %
\rhead{\firstxmark}                                                     %
\lfoot{\lastxmark}                                                      %
\cfoot{}                                                                %
\rfoot{Page\ \thepage\ of\ \pageref{LastPage}}                          %
\renewcommand\headrulewidth{0.4pt}                                      %
\renewcommand\footrulewidth{0.4pt}                                      %

% This is used to trace down (pin point) problems
% in latexing a document:
%\tracingall

%%%%%%%%%%%%%%%%%%%%%%%%%%%%%%%%%%%%%%%%%%%%%%%%%%%%%%%%%%%%%
% Some tools
\newcommand{\enterProblemHeader}[1]{\nobreak\extramarks{#1}{#1 continued on next page\ldots}\nobreak%
                                    \nobreak\extramarks{#1 (continued)}{#1 continued on next page\ldots}\nobreak}%
\newcommand{\exitProblemHeader}[1]{\nobreak\extramarks{#1 (continued)}{#1 continued on next page\ldots}\nobreak%
                                   \nobreak\extramarks{#1}{}\nobreak}%

\newlength{\labelLength}
\newcommand{\labelAnswer}[2]
  {\settowidth{\labelLength}{#1}%
   \addtolength{\labelLength}{0.25in}%
   \changetext{}{-\labelLength}{}{}{}%
   \noindent\fbox{\begin{minipage}[c]{\columnwidth}#2\end{minipage}}%
   \marginpar{\fbox{#1}}%

   % We put the blank space above in order to make sure this
   % \marginpar gets correctly placed.
   \changetext{}{+\labelLength}{}{}{}}%

\setcounter{secnumdepth}{0}
\newcommand{\homeworkProblemName}{}%
\newcounter{homeworkProblemCounter}%
\newenvironment{homeworkProblem}[1][Problem \arabic{homeworkProblemCounter}]%
  {\stepcounter{homeworkProblemCounter}%
   \renewcommand{\homeworkProblemName}{#1}%
   \section{\homeworkProblemName}%
   \enterProblemHeader{\homeworkProblemName}}%
  {\exitProblemHeader{\homeworkProblemName}}%

\newcommand{\problemAnswer}[1]
  {\noindent\fbox{\begin{minipage}[c]{\columnwidth}#1\end{minipage}}}%

\newcommand{\problemLAnswer}[1]
  {\labelAnswer{\homeworkProblemName}{#1}}

\newcommand{\homeworkSectionName}{}%
\newlength{\homeworkSectionLabelLength}{}%
\newenvironment{homeworkSection}[1]%
  {% We put this space here to make sure we're not connected to the above.
   % Otherwise the changetext can do funny things to the other margin

   \renewcommand{\homeworkSectionName}{#1}%
   \settowidth{\homeworkSectionLabelLength}{\homeworkSectionName}%
   \addtolength{\homeworkSectionLabelLength}{0.25in}%
   \changetext{}{-\homeworkSectionLabelLength}{}{}{}%
   \subsection{\homeworkSectionName}%
   \enterProblemHeader{\homeworkProblemName\ [\homeworkSectionName]}}%
  {\enterProblemHeader{\homeworkProblemName}%

   % We put the blank space above in order to make sure this margin
   % change doesn't happen too soon (otherwise \sectionAnswer's can
   % get ugly about their \marginpar placement.
   \changetext{}{+\homeworkSectionLabelLength}{}{}{}}%

\newcommand{\sectionAnswer}[1]
  {% We put this space here to make sure we're disconnected from the previous
   % passage

   \noindent\fbox{\begin{minipage}[c]{\columnwidth}#1\end{minipage}}%
   \enterProblemHeader{\homeworkProblemName}\exitProblemHeader{\homeworkProblemName}%
   \marginpar{\fbox{\homeworkSectionName}}%

   % We put the blank space above in order to make sure this
   % \marginpar gets correctly placed.
   }%

%%%%%%%%%%%%%%%%%%%%%%%%%%%%%%%%%%%%%%%%%%%%%%%%%%%%%%%%%%%%%


%%%%%%%%%%%%%%%%%%%%%%%%%%%%%%%%%%%%%%%%%%%%%%%%%%%%%%%%%%%%%
% Make title
\title{\textmd{\textbf{\hmwkClass:\ \hmwkTitle}}\\\normalsize\vspace{0.1}\\\vspace{0.1in}}
\date{\today}
%%%%%%%%%%%%%%%%%%%%%%%%%%%%%%%%%%%%%%%%%%%%%%%%%%%%%%%%%%%%%

\begin{document}

\maketitle

\begin{homeworkProblem}
\texttt{Directions}: \textbf{Circle all the rehtorical strategies/devices}

\ex. Fallacies 
\a. ``Senator Y, did you stop taking bribes yet?''
\b. I believe that aliens visit this planet, therefore, it must be true; also, they told me they visit the planet, and so, it must be true.

\ex. Analogy: Metaphor \& Simile; Symbolism \& Anthropomorphism
\a. ``The economy is a runaway train''
\b. ``The cell is looking to destroy another cell, that's its job''

\ex. Repetition
\a. ``Spend, spend, spend, spend\ldots why, why why?''
\b. ``The results are valid\ldots The results are valid\ldots the results are valid.''

\ex. Hyperbole, or Overstating the claim
\a. ``There a million facts that support my statement.''
\b. ``The evidence clearly shows that climate change is\ldots''

\ex. Appeal to authority
\a. ``Well, he is a nobel prize wining physicist, so\ldots''

\ex. Allusion; Sarcasm \& Irony
\a. ``He is the Einstein of chemistry''

\ex. Thesis
\a. ``I argue that \texttt{This Is The Case}''

\ex. Inference: Deduction/Induction/Abduction
\a. ``X is the case in this sample, therefore, it must hold for the general case''

\ex. Logos; Pathos; Ethos

\end{homeworkProblem}

\newpage
\begin{homeworkProblem}
\texttt{Directions}: \textbf{Mark which student is {\sl less} personal or subjective}  
\\
 
\textbf{Student A:} I found that scientists engage in research in order to make discoveries and generate new ideas. Such research by scientists is hard work and often involves collaboration with other scientists which leads to discoveries which make the scientists famous. Such collaboration may be informal, such as when they share new ideas over lunch, or formal, such as when they are co-authors of a paper.\\

\textbf{Student B:} It was hard work to research famous scientists engaged in collaboration and I made many informal discoveries. My research showed that scientists engaged in collaboration with other scientists are co-authors of at least one paper containing their new ideas. Some scientists make formal discoveries and have new ideas.

\end{homeworkProblem}


\end{document}

%%%%%%%%%%%%%%%%%%%%%%%%%%%%%%%%%%%%%%%%%%%%%%%%%%%%%%%%%%%%%
