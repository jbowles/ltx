
%HANDOUT
\documentclass[11pt,twoside,letterpaper]{article}

\usepackage{amsthm}

\theoremstyle{definition}
\newtheorem{definition}{Definition}


\begin{document}
\author{Joshua Bowles}
\title{Handout: Considerations for Peer Review}
\date{\today}


\maketitle



\section{Discursive Context}
Keep in mind the structure, force, and general feel (i.e., your personal impression) of the \texttt{Argument}, as defined below.

\begin{definition} \textsc{Argument}, (Parsons 1996: 5)\footnote{Parson, Terrance. 1996. What is an argument? {\sl The Journal of Philosophy} 93:164-185. Or you can retrieve a Word document copy from Parsons' faculty page.}\\
An argument is a `task' that employs a {\sl reasoning structure} in a {\sl setting} with a {\sl target}.
\end{definition}

\begin{enumerate}
\item \textbf{Setting}\\
Set of assumptions about the world; assumed rules, principles, propositions.

\item \textbf{Target}\\
The goal --- is the goal actually met?

\item \textbf{Reasoning Structure}\\
A sequence of statements meant to reach a {\sc Target} in a specific {\sc Setting}.
\end{enumerate}

\section{Some questions to keep in mind}
\begin{enumerate}
\item Is there a clear sense of purpose by the first page?
\item[a] Do you know what the paper is about?
\item[b] Can you tell what the topic is and how the author approaches it.

\item Do you get a sense that there is a strategy?
\item[a] Is the strategy consistent throughout?

\item Does the author seem knowledgeable by the first page?
\item[a]  Do you get a sense that the author is a credible writer on this topic?
\end{enumerate}

\section{Ethos, Pathos, Logos}

\begin{definition}{\sc Ethos}\\
A shared ethical stance or commitment. Where the word {\sc Ethical} comes from.
\end{definition}

\begin{definition}{\sc Pathos}\\
Emotional or eliciting a response of strong feeling. This where the words {\sc Empathy, Sympathy, Pathetic} come from.
\end{definition}

\begin{definition}{\sc Logos}\\
Rational thought, reason, logic. This is where the word {\sc Logic} comes from. It is also used in terms of {\sc the word} or {\sc the breath} of God/Spirit---but in the sense that it is assumed that a divine being is the most rational of all.
\end{definition}

Rhetorical strategies are built around these three concepts. We use all three in any good argument or analysis, but we can also choose to emphasize one over the other.

\end{document}  