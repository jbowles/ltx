x\documentclass{article}
\usepackage{articlep,articlec}

\begin{document}
\title{Eng 2020 Terminology}
\date{\today}
\maketitle

%\section{Logic, Argumentation, and Rhetoric}}
\pex {\sc Logic and Argumentation}
\a {\bf Argument} An activity that employs a reasoning structure in a setting with a task/goal. 
\a {\bf Belief} Conviction that a proposition is justified and true.
\a {\bf Claim} A proposition that, ideally, has some type of evidence or
justification to support it.
\a {\bf Context} Combination of audience, purpose, and intent; the communicative situation.
\a {\bf Evidence} Information used to support claims and assertions; e.g., {\sc Facts, Statistics, Logical Inferences, Testimony, Empirical Data}. 
\a {\bf Fallacy} False reasoning that appears to be true.
\a {\bf Justification} \textsl{Why} a belief is true.
\a {\bf Knowledge} Justified true belief.
\a {\bf Logic} The reasoning structure of a sequence of statments; there are a number of types of logic: inductive, deductive, and abductive.
\a {\bf Proposition} A statement that is true or false.
\a {\bf Syllogism} A logical form of argument in which the conclusion follows directly fom the premises.
\a {\bf Warrant} The motivation for thinking there is a justification for a
belief.
\xe

\pex {\sc Rhetoric and Composition}
\a {\bf Audience} The set of people who will read your text. Appeals to an audience consist of {\sc Ethos, Pathos, Logos}. There is an {\sl ideal} and a {\sl real} audience.
\a {\bf Composition} To put together or build from smaller parts.
\a {\bf Genre} Categories of conventional language-related and rhetorical domains. 
\a {\bf Rhetoric} The technique (or technical skill) of structuring information effectively in order to communicate.
\xe

 %% %\section{Language and Science Terms}
 %% \pex {\sc Language and Linguistics}
 %% \a {\bf Pragmatics} How language is used; the order and manner in which we present information. Remember Gricean Maxims and the quote ``It is not what you say, but how you say it.'' 
 %% \a {\bf Semantics} Word meaning.
 %% \a {\bf Syntax} The strucutre of phrases and sentences; typically what we call grammar.
 %% \xe

%% \pex {\sc Science \& Technology Related}
%% \a {\bf }
%% \a {\bf }
%% \a {\bf }
%% \a {\bf }
%% \a {\bf }
%% \a {\bf }
%% \a {\bf }
%% \a {\bf }
%% \xe

\end{document}

