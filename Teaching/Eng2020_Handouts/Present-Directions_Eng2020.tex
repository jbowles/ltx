
%HANDOUT

\documentclass[11pt]{article}


\author{Eng2020: Joshua Bowles}
\title{Research Presentations}
\date{Beginning \today}
\begin{document}

\maketitle



\section{Some Guidelines}
\begin{enumerate}
\item Only 5 - 10 minutes.

\item  Your presentation should be based on the work you have done in this class. That is, the two papers and the annotated bibliography. You should also try to use the rough draft of your final paper as a guideline for your talk.

 \item  Talk about {\bf one} of the aspects below (or all):
    \begin{enumerate}
\item What you are arguing for in your paper.
 \item Technical details about your topic of research.
 \item Why you are interested in your topic.
 \item How hard was it to do research for this topic.
 \item What you learned about: Academic thinking; Research; The life of scientists or academics; How social factors can determine scientific issues; Why science and technical writing is boring
    \end{enumerate}
\item  Do not read---a presentation is a talk. 
    \begin{enumerate}
\item If you use a powerpoint, try not to read directly off the page.
\item If you have a handout, try not to read directly off the page.
    \end{enumerate}
\item  Be prepared to answer questions!
\end{enumerate}
Basically, what I am looking for is a sense that you have spent some time thinking and reading about your topic. I want to know that you are \textsl{somewhat} knowledgeable about your research --- of course, I don't expect you to be an expert. 


 \end{document}