\documentclass [11pt]{article}
\usepackage{setspace,amssymb,latexsym,amsmath,amscd,epsfig,amsthm,wasysym}
\usepackage{graphicx}

\font\minihelv=phvr at 6pt
\font\helv=phvr at 10pt
\font\bighelv=phvr at 20pt
\font\hugehelv=phvr at 36pt
\font\mybigfont=phvr at 16pt
\font\mymediumfont=phvr at 14pt
\font\mediumhelv=phvr at 14pt
\font\mybfit=ptmbi at 12pt

\def\minivskip{\vskip 1.5mm}
\def\myspace{\phantom{\Biggr\|}}
\def\leavespace{\vskip 4mm}

\def\cosec{\hbox{cosec}}
\def\sec{\hbox{sec}}
\def\cotan{\hbox{cotan}}

\parindent=0pt
\setlength{\evensidemargin}{0.0cm}
\setlength{\oddsidemargin}{0.0cm}
\setlength{\topmargin}{-1.5cm}
%\setlength{\baselineskip}{20pt}
\setlength{\textwidth}{17cm}
\setlength{\textheight}{23.5cm}
\hoffset = -0.25cm

\begin{document}
\begin{figure}
\includegraphics[width=3in]{closeupred}\includegraphics[width=3in]{greenflower}
\end{figure}

   \begin{center}

     {\bighelv LING 1010 Introduction to Language - Fall 2009} \\
     {\mediumhelv Syntax, Semantics, and Phonology} \\
     \ \\
     \ \\

     \begin{tabular}{ l r }

       \begin{tabular}{l}
          Instructor: Joshua Bowles          \\                                             
          Office:                    \\                                                  
          E-mail: bowlesling@gmail.com     \\                                               
          O:          \\                                             
                                            \\
       \end{tabular}

       &

       \begin{tabular}{l}
          Class: MWF 9:00-9:50 in Arjona 407 \\
          Office Hours: MWF 10:15-11:15 \\                                                  
          \phantom{Office Hours: }and by appointment                \\                                               
                                            \\        
                                            \\                                     
       \end{tabular}

     \end{tabular}

   \end{center}

   \ \\

   {\bf Textbook}\\
   Elementary Mathematical Modeling: Functions and Graphs by Mary Ellen Davis and C. Henry Edwards, 
   Prentice Hall (ISBN 0-13-096202-3) \\
   \ \\

   {\bf Who Should take this course ?}\\
   The course is intended primarily for students who are prepared for college-level algebra, but do not intend 
   to take calculus. If you do plan to take calculus, you should be in math 109Q. If you are just looking for
   the easiest "Q" around, math 102Q is a safer bet. \\
   \ \\

   {\bf Course Outline and Grades}\\
   We will use mathematical models to gain an understanding of real world problems and situations. We will 
   study the use of linear, quadratic, natural, exponential, logarithmic, and polynomial models. \\
   The exams are {\it tentatively} scheduled as follows:
   \begin{itemize}
     \item Exam I: Ch.1 and Ch.2 on September 24th (in class)
     \item Exam II: Ch.3 and Ch.4 on October 22nd (in class)
     \item Exam III: Ch.5 and Ch.6 on November 19th (in class)
     \item Final Exam : cumulative (i.e. Ch.1 through Ch.8) {\it probably} on December 13th
   \end{itemize}
   Each in class exam is budgeted at 45 minutes and counts for 20\% of your grade,
   the final is on the common exam day for math classes and counts for 25\%.
   Quizzes and homeworks will count for 15\% of your final grade.\\
   There will not be a curve, but there will be 10\% extra credit on each exam.
   The letter grade will be assigned according to the standard table:   
   93-100 gives A, 90-92 gives A-, 87-89 gives B+, 83-86 gives B, 80-82 gives B-,
   77-79 gives C+, 73-76 gives C, 70-72 gives C-, 67-69 gives D+, 63-66 gives D,
   60-62 gives D-, and 00-59 gives F.\\
   \ \\

   {\bf Notes on the Calendar}\\
   There will be no classes on September 6th (Labor Day weekend), and December 22nd, 24th and 26th
   (Thanksgiving Break). The last day to withdraw is September 13th and the Add/Drop deadline
   is November 1st. \\
   \ \\  
 
   {\bf Calculator Policy}\\
   A graphing calculator can be handy, but is not at all necessary for the course.
   In fact, you are encouraged to purchase a simple scientific calculator. As long
   as it has $sin$, $cos$, $tan$, $log$ and $exp$ on it, you should be fine. If you
   are spending more than $\$20$, you are probably already buying more than you need.
   \ \\

   {\bf Class Conduct}\\
   Class atmosphere will be quite relaxed. Just a few guidelines to make sure:
   \begin{itemize}
     \item  Arriving a few minutes late is tolerated as long as you make an effort
       to minimize the disturbance for other students. 
     \item  Eating and drinking in class should be reduced to a minimum. It is not
       forbidden, but please make sure that you are not disturbing others with noise
       and enticing aromas.
     \item  Dozing off in an early morning class is not a cardinal sin, but be
       sure not to make a spectacle of yourself. It is extremely disrespectful
       to do this in an ostentatious manner.
     \item  Turn off all cell phones or don't even bring them. 
     \item  If you cannot make it to class for whatever reason, make sure that
       you know what happened during the lecture that you missed. It is
       your responsibility, and nobody else's, to do so!
     \item  If you have to leave a class early, inform your instructor in
       advance. It is very rude to simply walk out in the middle of a
       lecture. 
     \item  If you make an appointment with your instructor keep it, or at
       least notify your instructor a.s.a.p. that you cannot make it. You
       would also do this for your physician, plumber or car mechanic.
       Your instructor deserves the same respect.
    \end{itemize}
   \ \\
       
   {\bf Makeups}\\
   If you have a valid reason for a makeup exam, inform your instructor
   a.s.a.p. A valid reason is a medical emergency, a death in the family, and, quite frankly, 
   very little else. In all cases, you will be expected to bring in proof.\\
   \ \\

   {\bf Student Athletes}\\
   If you are a student athlete, inform your instructor a.s.a.p. of
   interferences with your commitments as an athlete (especially
   conflicts with exam dates). You will be expected to bring in a
   letter from the athletics department. The sooner you notify us, the
   better that we will be able to accommodate you.\\
   \ \\

   {\bf Students with Disabilities}\\
   Inform your instructor a.s.a.p. of special needs that you may have like
   larger printouts of quizzes and exams or extra time on an exam.
   You will be expected to bring in a letter from the center for students 
   with disabilities. The sooner you notify us, the better that we will 
   be able to accommodate you.\\
   \ \\

   \vfill\eject

   {\bf Academic Integrity}\\
   A fundamental tenet of all educational institutions is academic
   honesty; academic work depends upon respect for and acknowledgment of
   the work and ideas of others. Misrepresenting someone else's work as
   one's own is a serious offense in any academic setting and it will not
   be condoned.
   Academic misconduct includes, but is not limited to, providing or
   receiving assistance in a manner not authorized by the instructor in
   the creation of work to be submitted for academic evaluation (e.g.
   papers, projects, examinations and assessments - whether online or in
   class); presenting, as one's own, the ideas, words or calculations of
   another for academic evaluation; doing unauthorized academic work for
   which another person will receive credit or be evaluated; using
   unauthorized aids in preparing work for evaluation (e.g. unauthorized
   formula sheets, unauthorized calculators, unauthorized programs or
   formulas loaded into your calculator, etc.); and presenting the same
   or substantially the same papers or projects in two or more courses
   without the explicit permission of the instructors involved.
   A student who knowingly assists another student in committing an act
   of academic misconduct shall be equally accountable for the violation,
   and shall be subject to the sanctions and other remedies described in
   The Student Code. Sanctions shall include, but are not limited to, a
   letter sent to the Dean of Students of the University; a grade of 0 on
   the assignment, quiz or exam; a grade of F for the course.
   It is well known that in some courses cheating is not punished as
   harshly as described above. This is not such a course.\\
   \ \\
   
%   \vfill\eject

   {\bf  Recommended Exercises}\\

\begin{tabular}{ll}
 1.1 & p. 11 \# 2,3,8,11,15,16,20,21,25,28,33,36,40 \\
 1.2 & p. 22 \# 1,2,7,8,11,14-16,19,22,26,31 \\
 1.3 & p. 37 \# 1,2,5,8-10,13,16-18,21,25,26 \\
 1.4 & p. 50 \# 1,2,5,6,9,10,15 \\
\end{tabular}

\vskip 2mm

\begin{tabular}{ll}
 2.1 & p. 70 \# 2,5,6,9,10,15,18,21,22,29,30 \\
 2.2 & p. 84 \# 1,2,3,6,8,9,16,19,20 \\                       
 2.3 & p. 96 \# 1,2,9,10,17,21,22 \\ 
\end{tabular}

\vskip 2mm

\begin{tabular}{ll}
 3.1 & p. 112 \# 1-10,20,21 \\
 3.2 & p. 125 \# 1,2,6,7,11,12,16,17,26 \\
 3.3 & p. 137 \# 1-8,17,18,30 \\                     
 3.4 & p. 151 \# 1,2,9,10,11,15,16,21\\      
\end{tabular}

\vskip 2mm

\begin{tabular}{ll}
 4.1 & p. 170 \# 1,3,8,11,12,15,17,18,21 \\
 4.2 & p. 182 \# 17,18,22,25-27,29 \\
 4.3 & p. 197 \# 1-4,7,8,21,22 \\
\end{tabular}

\vskip 2mm

\begin{tabular}{ll}
 5.1 & p. 218 \# 1-5,11,14 \\
 5.2 & p. 233 \# 1-4,7,15,18,21 \\
 5.3 & p. 245 \# 1,2,5,6,11,12,23 \\
 5.4 & p. 260 \# 1,4,9,14,19,24 \\
\end{tabular}

\vskip 2mm

\begin{tabular}{ll}
 6.1 & p. 282 \# 1-4,7,11-15,18,21 \\
 6.2 & p. 290 \# 1,3,6,7,10-13 \\
\end{tabular}

\vskip 2mm

\begin{tabular}{ll}
 7.1 & p. 307 \# 1-5,7,14-16,29 \\
 7.2 & p. 317 \# 1,2,7,8,23 \\
\end{tabular}

\vskip 2mm

\begin{tabular}{ll}
 8.1 & p. 348 \# 1-5,19,22,23 \\
 8.2 & p. 355 \# 1-5,20,21 \\
\end{tabular}


\end{document}
