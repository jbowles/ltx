\documentclass[a4paper,12pt]{article}
\usepackage[utf8x]{inputenc}

%opening
\title{Eng 2020\\ Department of English, \\ Utah Valley University}
\author{Instructor: Joshua Bowles}
\date{}
\begin{document}

\maketitle

\section{Course Description}
In this course you will continue to develop reading, writing, and thinking skills related to
college-level inquiry. The course emphasizes rhetorical, analytical, and argumentative
techniques intended to help you produce well-reasoned and carefully researched written
arguments in the context of science and technology. The course embodies the spirit of
exploration and inquiry into multiple perspectives of scientific issues. It highlights
literacy in scientific/technical reading and writing (and thinking), and encourages
discovery of how scientific ideas relate to society as a whole.
\subsection{What you will get out of this class}
Upon completing this course you will be able to demonstrate in written and oral form
\begin{enumerate}
    \item An informed notion of scientific and academic inquiry
\item Technical and analytic strategies for argumentation
\item How to use information and evidence for arguments in a scientific context
\item The logical/rhetorical use of ethos, pathos, and logos in scientific arguments
\end{enumerate}

\section{Supplemental Readings}
I do not expect you to understand supplemental readings, but you must make an effort.
The texts you will be exposed to are primary research material written by academics (physicists, biologists, neuroscientists, linguists, computer scientists, mathematicians, philosophers of science, {\sl et cetera}). They are very advanced, highly abstract, and at times very technical. The goal of reading these texts is not to understand all the details but to get a feel for the general compositional style, mechanics, rhetoric, and format. Your responses to them need to be thoughtful and can be about the composition of the piece or the scientific details. The rationale behind assigning very difficult readings and asking you to respond is simple: good readers make good writers. The more you see writing done by and for scientists (as opposed to writing intended for lay audiences) the better equipped you are to write about technical and scientific issues.
Responses represent a format for experimenting with how you come to perceive complex
technical and scientific ideas through composition (or composition through complex
technical and scientific ideas). Think of them as your laboratory. If successful, your
ability to respond to complex scientific writing (and thinking) will develop throughout
the course.


\end{document}
