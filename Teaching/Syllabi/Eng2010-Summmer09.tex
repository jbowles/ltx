\documentclass [11pt]{article}
\usepackage{setspace,amssymb,latexsym,amsmath,amscd,epsfig,amsthm,wasysym}
\usepackage{graphicx}
\usepackage[normalem]{ulem}
\usepackage[usenames]{xcolor}
\definecolor{jblinkcolor}{rgb}{.2,.2,1}
\usepackage[colorlinks,breaklinks,
			linkcolor=jblinkcolor,
			citecolor=jblinkcolor,
			urlcolor=jblinkcolor,
			plainpages=false,
			bookmarks=false]{hyperref}
			\urlstyle{rm}

\font\minihelv=phvr at 6pt
\font\helv=phvr at 10pt
\font\bighelv=phvr at 20pt
\font\hugehelv=phvr at 36pt
\font\mybigfont=phvr at 16pt
\font\mymediumfont=phvr at 14pt
\font\mediumhelv=phvr at 14pt
\font\mybfit=ptmbi at 12pt

\def\minivskip{\vskip 1.5mm}
\def\myspace{\phantom{\Biggr\|}}
\def\leavespace{\vskip 4mm}

\def\cosec{\hbox{cosec}}
\def\sec{\hbox{sec}}
\def\cotan{\hbox{cotan}}

\parindent=0pt
\setlength{\evensidemargin}{0.0cm}
\setlength{\oddsidemargin}{0.0cm}
\setlength{\topmargin}{-1.5cm}
%\setlength{\baselineskip}{20pt}
\setlength{\textwidth}{17cm}
\setlength{\textheight}{23.5cm}
\hoffset = -0.25cm

\begin{document}
\begin{figure}
%\includegraphics[width=2in]{closeupred}
\includegraphics[width=2in]{greenflower}
\end{figure}

   \begin{center}

     {\bighelv ENG 2010 Intermediate Composition - Summer 2009} \\
     {\mediumhelv Humanities \& Social Sciences} \\
     \ \\
     \ \\

     \begin{tabular}{ l r }

       \begin{tabular}{l}
          Instructor: Joshua Bowles          \\                                             
          Office:                    \\                                                  
          E-mail: bowlesling@gmail.com     \\                                               
          Website: \href{http://sites.google.com/site/bowleslinguistics/}{http://sites.google.com/site/bowleslinguistics/}          \\                                             
                                            \\
       \end{tabular}

       &

       \begin{tabular}{l}
          Class: MWF 8:45 - 10:25 LA 122 \\
          Office Hours: By appointment \\                                                  
          %\phantom{Office Hours:} By appointment                \\                                               
                                            \\        
                                            \\                                     
       \end{tabular}

     \end{tabular}

   \end{center}

   \ \\

   {\bf Textbook}\\
{\bf {\sl A\&B}}: Ramage, John D., and John C. Bean, and June Johnson. {\sl The Allyn and Bacon Guide to Writing}. 5th edition. New York: Pearson Longman, 2009. ISBN-13:978-0-205-59873-1. (A link for used versions:  \href{http://www.amazon.com/Allyn-Bacon-Guide-Writing-MyCompLab/dp/0205598730/ref=sr_1_3/175-1452668-8158623?ie=UTF8&s=books&qid=1245772101&sr=1-3}{here}.) \\
   \ \\

   
   {\bf Course Outline and Grades}\\
   We will focus on methods of analyzing ideas and how to communicate such methods (and their results) in written form. You will demonstrate your ability to think in written form (and your development) through the following:
   \begin{enumerate}
     \item Informative \& Surprising, 4-6 pgs
     \item Analysis \& Synthesis, 4-6 pgs
     \item Abstract \& Annotated Bibliography, 3 pgs
     \item Final: Classical Argument, 10-12 pgs
     \item A number of responses
   \end{enumerate}

The breakdown of grades and letter grade assignment will be according to the following:
\vskip 2mm
  
\begin{tabular}{|l|l|}
\hline
Attendance & 15\\
Responses & 15\\
3 Short Papers & 45\\
1 Final Paper & 20\\
Portfolio & 5\\
\hline
\end{tabular} \  \  \  \  \ \begin{tabular}{|l|l|}
   \hline
   93-100 gives A & 70-72 gives C-\\
   90-92 gives A- & 67-69 gives D+\\
   87-89 gives B+ & 77-79 gives C+\\
   83-86 gives B & 63-66 gives D\\
   80-82 gives B- & 60-62 gives D-\\
   73-76 gives C & 00-59 gives F\\
   \hline 
   \end{tabular}
   \ \\

 {\bf Portfolio}\\
   Save {\bf ALL} written assignments in this class. I elaborate on this as the semester proceeds. \\
   \ \\ 

   {\bf Notes on the Calendar}\\
   If you have never taken a Summer course before, do not blink. \textcolor{red}{These semesters are lightening fast (only 7.5 of the usual 16 weeks---with normal workload)}. As the course proceeds we usually change the schedule quite a bit. I say {\it we} because I take input from students seriously. However, if you miss class you miss the chance to help shape the class, and more importantly, you miss changes to the schedule. It is your responsibility to find out any information you miss. \\
   \ \\  

   {\bf Class Conduct}\\
   Class atmosphere will be quite relaxed. Just a few (sarcastic) guidelines to make sure things are smooth:
   \begin{itemize}
   \item  Dozing off in class is not a cardinal sin, but be sure to make a spectacle of yourself.
   \item I do not take role, so clearly this means that I do not care if you come to class. In fact, do not come to class at all, everything will be fine.
    \item Talking during lectures is forgivable, but talking about mindless dribble is not forgivable.
     \item  Eating and drinking in class should be reduced to a minimum. It is not forbidden, but please make sure that you are as loud as possible so that you can disturb others with the noise and enticing aromas.
        
    \end{itemize}
   \ \\
       
   %\vfill\eject

   {\bf Academic Integrity}\\
 Cheating, plagiarism, or any unethical academic behavior is not tolerated. It will be reported immediately to your Major Department and Student Services. See plagiarism policy \href{http://webstaging.uvu.edu/english/student/plagiarism.html/}{here} or at http://webstaging.uvu.edu/english/student/plagiarism.html/.\\
   \ \\
  
  
  {\bf Basic Resources}\\
  The following is a list (with active links) of some basic research corpora.
  \begin{enumerate}
\item UVU library: \href{http://www.uvu.edu/library}{http://www.uvu.edu/library}
\item Library search engines: \href{http://www.uvu.edu/library/search/index.php}{http://www.uvu.edu/library/search/index.php}
\item Useful databases: JSTOR, MEDLINE, especially ACADEMIC SEARCH PREMIER
\item Cornell science archive: \href{http://www.arXiv.org}{http://www.arXiv.org}
\item UVU writing center: \href{http://www.uvsc.edu/owl}{http://www.uvsc.edu/owl}
\end{enumerate}
\ \\
   %\vfill\eject

   {\bf  Schedule of Readings and Assignments}\\
   The schedule here is {\it tentative} and will most likely be modified. It is your responsibility to keep track of the changes as I announce them in class.
   \begin{enumerate}
\item \sout{(Week 1)} June 26 F:  Introductions;  Open discussion about convention, audience, why we write. \textbf{READ: A\&B: 139, 115-17; START A\&B ch.9; \href{http://www.newyorker.com/reporting/2007/04/16/070416fa_fact_colapinto}{Assigned I}} 

\item \sout{(Week 2)} June 30 M: Critical reading of Assigned I. EXCERCISE: Research Questions in A\&B:243-5; research topics/issues---statement of interests. \texttt{Start paper 1}
\item[] July 1 W: \texttt{RESPONSE TO I DUE.} What is Evidence? EXERCISE: Thesis statement / Extend thesis statement / Peer review. Library for second-half;  \textbf{ READ: finish A\&B ch.9; A\&B:579, 602-4; \href{http://www.llc.ilstu.edu/dlevere/docs/currentanthroarticle.web.pdf}{Assigned II}}
\item[] \textcolor{magenta}{July 3 F: Holiday}

\item \sout{(Week 3)} July 6 M: \textcolor{red}{\texttt{PAPER 1 DUE(?)}} What is Analysis/Rhetoric? Rhetorical strategies. \textsl{GROUP PEER REVIEW of II.} {\bf READ: Assigned III, IV, V}   
\item[] July 8 W: \texttt{Response to Assigned II due.} Finish group peer review. Start peer review for Paper 2  
\item[] July 10 F: Comments on research. Peer review for Paper 2

\item \sout{(Week 4)} July 13 M: RESEARCH DAY: Abstracts/Bibliographies. {\bf READ: A\&B ch.14, 15}
\item[] July 15 W: \textcolor{red}{\texttt{PAPER 2 DUE.}} Informal Logic: How to be a con-artist: exploiting the audience. Library for second-half: Inter-library Loan (ILL) / Database / Research / Source documentation. (Homework assigned)
\item[] July 17 F: Informal Logic: Syllogism \& Fallacy and Reasoning: logical style and its limits; style and voice are worth a thousand words. 

\item \sout{({\sl Light Week--5})} July 20 M: RESEARCH DAY: Journal articles. {\bf READ: A\&B ch.14, 15}
\item[] July 22 W: \textcolor{red}{\texttt{ABSTRACT \& ANN. BIBLIOGRAPHY (\#3) DUE.}} {\bf READ: A\&B ch.14, 15} 
\item[] \textcolor{magenta}{July 24 F: Holiday}

\item (Week 6) July 27 M: RESEARCH DAY 
\item[] July 29 W: RSRCH. PRESENTATIONS
\item[] July 31 F: RSRCH. PRESENTATIONS

\item (Week 7) Aug. 3 M: Writing Review Conferences 
\item[] Aug. 5 W: \textcolor{red}{\texttt{REASONING AND FALLACY HMWK DUE}} RSRCH. PRESENTATIONS. Peer review. 
\item[] Aug. 7 F: Open discussion about logic, style, voice, audience. Peer review (Earliest day to accept Portfolios)

\item (Week 8) Aug 10 M: STUDY DAY. Will accept Portfolios.
\item[] Aug. 12 W: \textcolor{red}{\texttt{PAPER 4 and PORTFOLIOS DUE}}
\item[] Aug. 14 F: Pick up Portfolios (for those who specify)
\end{enumerate}


\section{Possible Research Topics}
\begin{enumerate}
\item Symmetry and Fibonacci patterns in nature and art
\item Mathematics of probability (i.e., medicine, insurance, game shows, gambling, physics, marketing)
\item Encryption and security
\item Economics of personal and corporate debt
\item Biology of stem cells, e.g., differentiation and reverse engineering\footnote{\emph{Many people} write papers on this topic.}
\item Quantum computers, quantum computation, and the theory of quantum consciousness
\item Science of alternative energy\footnote{Again, many people write on this subject.}
\item Biographical life of a scientist---Newton, Einstein, Galileo, Galois, Abels, Cantor, Turing, Boole, Godel\ldots{}.
\item Philosophy of science, religion, etc.\ldots
\item Literature and consciousness
\item Artificial Intelligence and Gaming
\item Gaming and Social Networks
\item Social Networking and `Hooking up'
\item War
\item Homosexuality: nature or nurture?
\item Does racism exist?\end{enumerate}

\subsection{Comment}  
You should find something that engages you. Even topics you think might not be ``allowed'' for a college paper (e.g., I have had students write about such varied things as time-travel, medieval knights, Mayan calender, alien life, \textsl{et cetera}). The point of attending University is to get the chance to have a \emph{transformative experience}. You can have a good experience in this class, and do interesting research, if you pick topics that are really cool.

\section{Citation formats, Organizations, Websites}
\begin{itemize}
\item \href{http://www.bedfordstmartins.com/online/citex.html}{Bedsford's St. Martins Press} MLA, APA, Chicago, and other styles 
\item \href{http://www.library.cornell.edu/resrch/citmanage/mla}{Cornell University Library} MLA style guide
\item \href{http://owl.english.purdue.edu/owl/resource/557/01/}{Purdue University Library} MLA style guide
\item \href{http://owl.english.purdue.edu/owl/resource/560/01/}{Purdue University Library} APA style guide
\item \href{http://www.lib.berkeley.edu/instruct/guides/mlastyle.pdf}{UC Berkeley Library} MLA style guide
\item \href{http://languagelog.ldc.upenn.edu/nll/}{Modern Language Society (MLA)}
\item\href{http://www.amwa.org/}{American Medical Writers Association}
\item\href{http://www.attw.org/}{Association of Teachers of Technical Writing}
\item\href{http://www.ieeepcs.org/}{IEEE Professional Communication Society}
\item \href{http://standards.ieee.org/guides/style/2009_Style_Manual.pdf}{2009 IEEE Standards Style Manual}
\item\href{http://nasw.org/}{National Association of Science Writers}
\item\href{http://www.stc.org/}{Society for Technical Communication}\end{itemize}

\section{Research}
\subsection{Databases}\begin{itemize}  
\item Research tools accessible at the \href{http://www.uvu.edu/library/researchtools/index.html}{UVU-Library}
\item \href{http://www.uvu.edu/library/guides/index.html}{Research guides} by topic at UVU library
\item \href{http://www.uvu.edu/library/researchtools/electronic_encyclopedias.html}{Electronic Encyclopedias and Dictionaries}
\item JSTOR 
\item ScienceDirect 
\item ERIC 
\item Academic Search Premier 
\item Project Muse
\end{itemize}

\subsection{Other Links and Research Sources}
\begin{itemize}
\item \href{http://languagelog.ldc.upenn.edu/nll/}{Language Log}
\item \href{http://mitpress.mit.edu}{MIT Press} 
\item \href{http://arxiv.org}{Cornell ArXiv} 
\item \href{http://plato.stanford.edu/}{Stanford Encyclopedia of Philosophy} 
\item Academic web pages of \href{http://www.uvu.edu/profpages/profiles/show/user_id/530}{professors} for downloadable papers\footnote{Make sure these sites are connected to a University or College---as the example here shows, if you are on-line!}
\end{itemize}
\end{document}
