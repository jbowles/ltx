\documentclass [11pt]{article}
\usepackage{setspace,amssymb,latexsym,amsmath,amscd,epsfig,amsthm,wasysym}

\font\minihelv=phvr at 6pt
\font\helv=phvr at 10pt
\font\bighelv=phvr at 20pt
\font\hugehelv=phvr at 36pt
\font\mybigfont=phvr at 16pt
\font\mymediumfont=phvr at 14pt
\font\mediumhelv=phvr at 16pt
\font\mybfit=ptmbi at 12pt

\def\minivskip{\vskip 1.5mm}
\def\myspace{\phantom{\Biggr\|}}
\def\leavespace{\vskip 4mm}

\def\cosec{\hbox{cosec}}
\def\sec{\hbox{sec}}
\def\cotan{\hbox{cotan}}

\parindent=0pt
\setlength{\evensidemargin}{0.0cm}
\setlength{\oddsidemargin}{0.0cm}
\setlength{\topmargin}{-1.5cm}
%\setlength{\baselineskip}{20pt}
\setlength{\textwidth}{17cm}
\setlength{\textheight}{23.5cm}
\hoffset = -0.25cm

\begin{document}

   \begin{center}

     {\bighelv MATH 109Q - Pre Calculus} \\
     {\phantom{\minihelv x}}
     {\bighelv May Intersession 2006} \\
     \ \\
     \ \\

     \begin{tabular}{ l r }

       \begin{tabular}{l}
          Instructor: Marc Corluy           \\                                             
          Office: MSB 335                   \\                                                  
          E-mail: corluy@math.uconn.edu     \\                                               
          Office Phone: 486.12.88           \\                                             
                                            \\
       \end{tabular}

       &

       \begin{tabular}{l}
          Class: Mo to Fr 9:00-12:00 in MSB 319 \\
          Office Hours: Mo to Fr 12:00-12:30  \\                                                
          \phantom{Office Hours: }and by appointment                \\                                               
                                            \\        
                                            \\                                     
       \end{tabular}

     \end{tabular}

   \end{center}



   {\bf Textbook}\\
   Pre Calculus (Third Edition) by J.Douglas Faires and James DeFranza, Thomson Brooks/Cole \\           
   (ISBN 0-534-46279-0) \\
   \ \\

   {\bf Course Outline and Grades}\\
   The goal of this course is to prepare you - and in fact to prepare you well - for a calculus course
   which you should preferably take next semester. We will focus our attention on general properties
   of functions and then study several classes of functions in more detail. 
   These classes will be polynomial, rational, root, trigonometric, cyclometric, exponential and logarithmic functions.
   The idea of asymptotes will prepare you for the concept "limit".  
   The exams are {\it tentatively} scheduled as follows:
   \begin{itemize}
     \item Exam I: General Properties of Functions (1.2, 1.3, 1.6, 1.7, 1.8, 2.2, 2.3, 2.4) - May 12          
     \item Exam II: Algebraic Functions (3.2 to 3.5) - May 17           
     \item Exam III: Trigonometric Functions (4.2 to 4.7 + Law of Sines and Law of Cosines) - May 23                
     \item Final Exam : cumulative (all of the above + 2.5, 4.8, 4.9, 5.2, 5.3) - May 26 (at 10:00am)
   \end{itemize}
   Each in class exam accounts for 22\% of your grade and the final accounts for 34\% of your grade.
   There will not be a curve, but there will be 5\% extra credit on each exam.
   The letter grade will be assigned according to the standard table:   
   93-100 gives A, 90-92 gives A-, 87-89 gives B+, 83-86 gives B, 80-82 gives B-,
   77-79 gives C+, 73-76 gives C, 70-72 gives C-, 67-69 gives D+, 63-66 gives D,
   60-62 gives D-, and 00-59 gives F.\\
   \ \\

   {\bf Notes on the Calendar}\\
   In the event of emergency closing (which is very rare in a summer intersession), the          
   makeup dates are May 13, 20 and 275. Emergency closing is announced on the web, for details 
   see \\
   {\it http://www.news.uconn.edu/emergencyclosing/index.html}. \\
   \ \\  
 
   {\bf Calculator Policy}\\
   The exams will be without pocket calculator. If you decide to use a graphing calculator while making
   exercises in class or at home, do so only to verify already obtained results. The problems will be 
   adapted to this restriction, e.g.  you will never be expected to multiply 1.2305 and 0.9746 on an 
   exam as this would take way too much time.
   \ \\

   \vfill\eject

   {\bf Different Editions of the Textbook} \\
   The second and third edition of the textbook are slightly different. The chapters are cut
   differently. 1.9, 1.10 and 1.11 in the second edition correspond to 2.2, 2.3 and 2.4 in the
   third edition.  For the rest, all sections shift up one number, e.g. 3.5 in the second edition
   becomes 4.5 in the third edition and so on. \\
   \ \\
 
   {\bf Administrative Information} \\
   This intersession is organized by the School of Continuing Studies. For any administrative
   information (late enrollment, add/drop policy or financial issues) look at \\
   {\it http://continuingstudies.uconn.edu/specialsessions/summer/catalog\_index.html}. \\
   \ \\
 
   {\bf Student Athletes}\\
   If you are a student athlete, inform your instructor a.s.a.p. of
   interferences with your commitments as an athlete (especially
   conflicts with exam dates). You will be expected to bring in a
   letter from the athletics department. The sooner you notify me, the
   better that I will be able to accommodate you.\\
   \ \\

   {\bf Students with Disabilities}\\
   Inform your instructor a.s.a.p. of special needs that you may have like
   larger printouts of quizzes and exams or extra time on an exam.
   You will be expected to bring in a letter from the center for students 
   with disabilities. The sooner you notify me, the better that I will 
   be able to accommodate you.\\
   \ \\

   {\bf Academic Integrity}\\
   A fundamental tenet of all educational institutions is academic
   honesty; academic work depends upon respect for and acknowledgment of
   the work and ideas of others. Misrepresenting someone else's work as
   one's own is a serious offense in any academic setting and it will not
   be condoned.
   Academic misconduct includes, but is not limited to, providing or
   receiving assistance in a manner not authorized by the instructor in
   the creation of work to be submitted for academic evaluation (e.g.
   papers, projects, examinations and assessments - whether online or in
   class); presenting, as one's own, the ideas, words or calculations of
   another for academic evaluation; doing unauthorized academic work for
   which another person will receive credit or be evaluated; using
   unauthorized aids in preparing work for evaluation (e.g. unauthorized
   formula sheets, unauthorized calculators, unauthorized programs or
   formulas loaded into your calculator, etc.); and presenting the same
   or substantially the same papers or projects in two or more courses
   without the explicit permission of the instructors involved.
   A student who knowingly assists another student in committing an act
   of academic misconduct shall be equally accountable for the violation,
   and shall be subject to the sanctions and other remedies described in
   The Student Code. Sanctions shall include, but are not limited to, a
   letter sent to the Dean of Students of the University; a grade of 0 on
   the assignment, quiz or exam; a grade of F for the course.
   It is well known that in some courses cheating is not punished as
   harshly as described above. This is not such a course.\\
   \ \\
   
\end{document}
