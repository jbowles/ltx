\documentclass [11pt]{article}
\usepackage{setspace,amssymb,latexsym,amsmath,amscd,epsfig,amsthm,wasysym}
\usepackage{graphicx}
\usepackage[normalem]{ulem}
\usepackage[usenames]{xcolor}
\definecolor{jblinkcolor}{rgb}{.2,.2,1}
\usepackage[colorlinks,breaklinks,
			linkcolor=jblinkcolor,
			citecolor=jblinkcolor,
			urlcolor=jblinkcolor,
			plainpages=false,
			bookmarks=false]{hyperref}
			\urlstyle{rm}

\font\minihelv=phvr at 6pt
\font\helv=phvr at 10pt
\font\bighelv=phvr at 20pt
\font\hugehelv=phvr at 36pt
\font\mybigfont=phvr at 16pt
\font\mymediumfont=phvr at 14pt
\font\mediumhelv=phvr at 14pt
\font\mybfit=ptmbi at 12pt

\def\minivskip{\vskip 1.5mm}
\def\myspace{\phantom{\Biggr\|}}
\def\leavespace{\vskip 4mm}

\def\cosec{\hbox{cosec}}
\def\sec{\hbox{sec}}
\def\cotan{\hbox{cotan}}

\parindent=0pt
\setlength{\evensidemargin}{0.0cm}
\setlength{\oddsidemargin}{0.0cm}
\setlength{\topmargin}{-1.5cm}
%\setlength{\baselineskip}{20pt}
\setlength{\textwidth}{17cm}
\setlength{\textheight}{23.5cm}
\hoffset = -0.25cm

\begin{document}

\begin{figure}
\includegraphics[width=2in]{RoothOrdinary}\includegraphics[width=2in]{nasaM101}\includegraphics[width=2in]{KrifkaFocus}
\end{figure}

   \begin{center}

     {\bighelv ENG 1010 Introduction to Composition - Fall 2009} \\
     {\mediumhelv Humanities \& Sciences Research and Writing} \\
     \ \\
     \ \\

     \begin{tabular}{ l r }

       \begin{tabular}{l}
          Instructor: Joshua Bowles          \\                                             
          Office:                    \\                                                  
          E-mail: uvucompbowles@gmail.com   \\                                               
          Website: \href{http://sites.google.com/site/bowleslinguistics/}{http://sites.google.com/site/bowleslinguistics/}          \\                                             
                                            \\
       \end{tabular}

       &

       \begin{tabular}{l}
          Class: MWF \\
          Office Hours: By appointment \\                                                  
          %\phantom{Office Hours:} By appointment                \\                                               
                                            \\        
                                            \\                                     
       \end{tabular}

     \end{tabular}

   \end{center}

   \ \\

   {\bf Textbooks}\\
{\bf \textsl{A\&B}}: Ramage, John D., and John C. Bean, and June Johnson. {\sl The Allyn and Bacon Guide to Writing}. 5th edition. New York: Pearson Longman, 2009. ISBN-13:978-0-205-59873-1. (A link for used versions:  \href{http://www.amazon.com/Allyn-Bacon-Guide-Writing-MyCompLab/dp/0205598730/ref=sr_1_3/175-1452668-8158623?ie=UTF8&s=books&qid=1245772101&sr=1-3}{here}.) 
\\
{\bf \textsl {DK}}: Wysocki, Anne Frances, and Dennis A. Lynch. {\sl The DK Handbook.} New York: Pearson Longman, 2008. ISBN-13 978-0-321-42053-4. 
   \ \\

   
   {\bf Course Outline and Grades}\\
   We will focus on methods of analyzing ideas and how to communicate such methods (and their results) in written form. You will demonstrate your ability to think in written form (and your development) through the following:
   \begin{enumerate}
     \item Problematizing a Topic, 2 pgs
     \item Solving a Local Problem, 4-6 pgs
     \item Summary \& Strong Response, 4-6 pgs
     \item Proposal \& Annotated Bibliography, 3 pgs
     \item Final: Exploratory Research, 10-12 pgs
     \item In-Logic Notebook (In-LN)
   \end{enumerate}

The breakdown of grades and letter grade assignment will be assigned according to the following:
\vskip 2mm
  
\begin{tabular}{|l|l|}
\hline
Attendance & 15\\
In-Logic Notebook & 15\\
4 Short Papers & 45\\
1 Final Paper & 20\\
Portfolio & 5\\
\hline
\end{tabular} \  \  \  \  \ \begin{tabular}{|l|l|}
   \hline
   93-100 gives A & 70-72 gives C-\\
   90-92 gives A- & 67-69 gives D+\\
   87-89 gives B+ & 77-79 gives C+\\
   83-86 gives B & 63-66 gives D\\
   80-82 gives B- & 60-62 gives D-\\
   73-76 gives C & 00-59 gives F\\
   \hline 
   \end{tabular}
   \ \\

 {\bf Portfolio}\\
   Save {\bf ALL} written assignments in this class. I elaborate on this as the semester proceeds. \\
   \ \\ 

   {\bf Notes on the Calendar}\\
  \textcolor{red}{ATTENDANCE IS MANDATORY}. As the course proceeds we usually change the schedule quite a bit. I say {\it we} because I take input from students seriously. However, if you miss class you miss the chance to help shape the class, and more importantly, you miss changes to the schedule. It is your responsibility to find out this information. \\
   \ \\  

   {\bf Class Conduct}\\
   Class atmosphere will be quite relaxed. Just a few (sarcastic) guidelines to make sure things are smooth:
   \begin{itemize}
   \item  Dozing off in class is not a cardinal sin, but be sure to make a spectacle of yourself.
   \item I do not take role, so clearly this means that I do not care if you come to class. In fact, do not come to class at all, everything will be fine.
    \item Talking during lectures is forgivable, but talking about mindless dribble is not forgivable.
     \item  Eating and drinking in class should be reduced to a minimum. It is not forbidden, but please make sure that you are as loud as possible so that you can disturb others with the noise and enticing aromas.
        
    \end{itemize}
   \ \\
       
   %\vfill\eject

   {\bf Academic Integrity}\\
 Cheating, plagiarism, or any unethical academic behavior is not tolerated. It will be reported immediately to your Major Department and Student Services. See plagiarism policy \href{http://www.uvu.edu/english/student/plagiarism.html}{here} or at http://www.uvu.edu/english/student/plagiarism.html.\\
   \ \\
  
  
  {\bf Basic Resources}\\
  The following is a list (with active links) of some basic research corpora.
  \begin{enumerate}
\item UVU library: \href{http://www.uvu.edu/library}{http://www.uvu.edu/library}
\item Library search engines: \href{http://www.uvu.edu/library/search/index.php}{http://www.uvu.edu/library/search/index.php}
\item Useful databases: JSTOR, MEDLINE, especially ACADEMIC SEARCH PREMIER
\item Cornell science archive: \href{http://www.arXiv.org}{http://www.arXiv.org}
\item UVU writing center: \href{http://www.uvsc.edu/owl}{http://www.uvsc.edu/owl}
\end{enumerate}
   \vfill\eject

   


\section{Possible Research Topics}
\begin{enumerate}
\item Symmetry and Fibonacci patterns in nature
\item Mathematics of probability (i.e., medicine, insurance, game shows, gambling, physics, marketing)
\item Encryption and security
\item Economics of personal and corporate debt
\item Biology of stem cells, e.g., differentiation and reverse engineering\footnote{\emph{Many people} write papers on this topic.}
\item Quantum computers, quantum computation, and the theory of quantum consciousness
\item Science of alternative energy\footnote{Again, many people write on this subject.}
\item Biographical life of a scientist---Newton, Einstein, Galileo, Galois, Abels, Cantor, Turing, Boole, G\"odel\ldots{}.
\item Philosophy of science, religion, etc.\ldots
\item Literature and consciousness
\item Artificial Intelligence and Gaming
\item Gaming and Social Networks
\item Social Networking and `Hooking up'
\item War
\item Homosexuality: nature or nurture?
\item Does racism exist?
\end{enumerate}

\subsection{Comment}  
You should find something that engages you. Even topics you think might not be ``allowed'' for a college paper (e.g., I have had students write about such varied things as time-travel, medieval knights, Mayan calender, alien life, \textsl{et cetera}). The point of attending University is to get the chance to have a \emph{transformative experience}. You can have a good experience in this class, and do interesting research, if you pick topics that are really cool.

\section{Citation formats, Organizations, Websites}
\begin{itemize}
\item \href{http://www.bedfordstmartins.com/online/citex.html}{Bedsford's St. Martins Press} MLA, APA, Chicago, and other styles 
\item \href{http://www.library.cornell.edu/resrch/citmanage/mla}{Cornell University Library} MLA style guide
\item \href{http://owl.english.purdue.edu/owl/resource/557/01/}{Purdue University Library} MLA style guide
\item \href{http://owl.english.purdue.edu/owl/resource/560/01/}{Purdue University Library} APA style guide
\item \href{http://www.lib.berkeley.edu/instruct/guides/mlastyle.pdf}{UC Berkeley Library} MLA style guide
\item \href{http://languagelog.ldc.upenn.edu/nll/}{Modern Language Society (MLA)}
\item\href{http://www.amwa.org/}{American Medical Writers Association}
\item\href{http://www.attw.org/}{Association of Teachers of Technical Writing}
\item\href{http://www.ieeepcs.org/}{IEEE Professional Communication Society}
\item \href{http://standards.ieee.org/guides/style/2009_Style_Manual.pdf}{2009 IEEE Standards Style Manual}
\item\href{http://nasw.org/}{National Association of Science Writers}
\item\href{http://www.stc.org/}{Society for Technical Communication}\end{itemize}

\section{Research}
\subsection{Databases}\begin{itemize}  
\item Research tools accessible at the \href{http://www.uvu.edu/library/researchtools/index.html}{UVU-Library}
\item \href{http://www.uvu.edu/library/guides/index.html}{Research guides} by topic at UVU library
\item \href{http://www.uvu.edu/library/researchtools/electronic_encyclopedias.html}{Electronic Encyclopedias and Dictionaries}
\item JSTOR 
\item ScienceDirect 
\item ERIC 
\item Academic Search Premier 
\item Project Muse
\end{itemize}

\subsection{Other Links and Research Sources}
\begin{itemize}
\item \href{http://languagelog.ldc.upenn.edu/nll/}{Language Log}
\item \href{http://mitpress.mit.edu}{MIT Press} 
\item \href{http://arxiv.org}{Cornell ArXiv} 
\item \href{http://plato.stanford.edu/}{Stanford Encyclopedia of Philosophy} 
\item Academic web pages of \href{http://www.uvu.edu/profpages/profiles/show/user_id/530}{professors} for downloadable papers\footnote{Make sure these sites are connected to a University or College---as the example here shows, if you are on-line!}
\end{itemize}

\end{document}
