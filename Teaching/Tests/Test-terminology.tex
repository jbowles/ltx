\documentclass{article}
% Change "article" to "report" to get rid of page number on title page
\usepackage{amsmath,amsfonts,amsthm,amssymb}
\usepackage{expex}
\usepackage{setspace}
\usepackage{Tabbing}
\usepackage{fancyhdr}
\usepackage{lastpage}
\usepackage{extramarks}
\usepackage{chngpage}
\usepackage{soul,color}
\usepackage{graphicx,float,wrapfig}

% In case you need to adjust margins:
\topmargin=-0.45in      %
\evensidemargin=0in     %
\oddsidemargin=0in      %
\textwidth=6.5in        %
\textheight=9.0in       %
\headsep=0.25in

% Exam Specific Information
\newcommand{\examHeading}{Exam (3 points)}
\newcommand{\examTitle}{Terms \& Analysis}
\newcommand{\examDueDate}{June 2, 2010}
\newcommand{\examClass}{Eng 2020}
\newcommand{\examClassTime}{8:45--10:25}
\newcommand{\examClassInstructor}{Joshua Bowles}
\newcommand{\examAuthorName}{Your Name:}

% Setup the header and footer
\pagestyle{fancy}                                                       %
\lhead{}                                                 %
\chead{\examClass\ (\examClassInstructor\ \examClassTime): \examTitle}  %
\rhead{\firstxmark}                                                     %
\lfoot{\lastxmark}                                                      %
\cfoot{Ethical considerations apply.}                                   %
\rfoot{Page\ \thepage\ of\ \pageref{LastPage}}                          %
\renewcommand\headrulewidth{0.4pt}                                      %
\renewcommand\footrulewidth{0.4pt}                                      %

% This is used to trace down (pin point) problems
% in latexing a document:
%\tracingall

%%%%%%%%%%%%%%%%%%%%%%%%%%%%%%%%%%%%%%%%%%%%%%%%%%%%%%%%%%%%%
% Some tools
\newcommand{\enterProblemHeader}[1]{\nobreak\extramarks{#1}{#1 continued on next page\ldots}\nobreak%
                                    \nobreak\extramarks{#1 (continued)}{#1 continued on next page\ldots}\nobreak}%
\newcommand{\exitProblemHeader}[1]{\nobreak\extramarks{#1 (continued)}{#1 continued on next page\ldots}\nobreak%
                                   \nobreak\extramarks{#1}{}\nobreak}%

\newlength{\labelLength}
\newcommand{\labelAnswer}[2]
  {\settowidth{\labelLength}{#1}%
   \addtolength{\labelLength}{0.25in}%
   \changetext{}{-\labelLength}{}{}{}%
   \noindent\fbox{\begin{minipage}[c]{\columnwidth}#2\end{minipage}}%
   \marginpar{\fbox{#1}}%

   % We put the blank space above in order to make sure this
   % \marginpar gets correctly placed.
   \changetext{}{+\labelLength}{}{}{}}%

\setcounter{secnumdepth}{0}
\newcommand{\examProblemName}{}%
\newcounter{examProblemCounter}%
\newenvironment{examProblem}[1][Problem \arabic{examProblemCounter}]%
  {\stepcounter{examProblemCounter}%
   \renewcommand{\examProblemName}{#1}%
   \section{\examProblemName}%
   \enterProblemHeader{\examProblemName}}%
  {\exitProblemHeader{\examProblemName}}%

\newcommand{\problemAnswer}[1]
  {\noindent\fbox{\begin{minipage}{7in}\hspace{1in}\vspace{7in}\end{minipage}}}%

\newcommand{\problemLAnswer}[1]
  {\labelAnswer{\examProblemName}{#1}}

\newcommand{\examSectionName}{}%
\newlength{\examSectionLabelLength}{}%
\newenvironment{examSection}[1]%
  {% We put this space here to make sure we're not connected to the above.
   % Otherwise the changetext can do funny things to the other margin

   \renewcommand{\examSectionName}{#1}%
   \settowidth{\examSectionLabelLength}{\examSectionName}%
   \addtolength{\examSectionLabelLength}{0.25in}%
   \changetext{}{-\examSectionLabelLength}{}{}{}%
   \subsection{\examSectionName}%
   \enterProblemHeader{\examProblemName\ [\examSectionName]}}%
  {\enterProblemHeader{\examProblemName}%

   % We put the blank space above in order to make sure this margin
   % change doesn't happen too soon (otherwise \sectionAnswer's can
   % get ugly about their \marginpar placement.
   \changetext{}{+\examSectionLabelLength}{}{}{}}%

\newcommand{\sectionAnswer}[1]
  {% We put this space here to make sure we're disconnected from the previous
   % passage

   \noindent\fbox{\begin{minipage}[c]{\columnwidth}#1\end{minipage}}%
   \enterProblemHeader{\examProblemName}\exitProblemHeader{\examProblemName}%
   \marginpar{\fbox{\examSectionName}}%

   % We put the blank space above in order to make sure this
   % \marginpar gets correctly placed.
   }%

%%%%%%%%%%%%%%%%%%%%%%%%%%%%%%%%%%%%%%%%%%%%%%%%%%%%%%%%%%%%%


%%%%%%%%%%%%%%%%%%%%%%%%%%%%%%%%%%%%%%%%%%%%%%%%%%%%%%%%%%%%%
% Make title
\title{\textmd{\textbf{ \examHeading \\ \examClass:\ \examTitle}}\\\normalsize\vspace{0.1in}\small{Given\ on\ \examDueDate}\\\vspace{0.1in}\large{ \textsl{\examClassInstructor}\ \examClassTime}}
\date{}
\author{\textbf{\examAuthorName}}
%%%%%%%%%%%%%%%%%%%%%%%%%%%%%%%%%%%%%%%%%%%%%%%%%%%%%%%%%%%%%

\begin{document}

\maketitle
Overview: This exam is {\bf open book} and {\bf closed-notes}. It is worth 3 points ($\approx$ 3\% of your grade). Each problem is worth 1.5 points. The first problem asks you to define briefly the term. The second problem asks you to do some basic analysis.

\begin{examProblem}
\texttt{Directions: Briefly define the following terms. Each definition is worth 1.5 points.({\sl Term \#4 is 0.375 points for each part}.)}

 \ex
Rhetoric:
{}\\
\xe

 \ex
Argument:
{}\\
\xe

\ex
Logic:
{}\\
\xe

 \pex Evidence (define and give me 3 types):
{}\\
{}\\
\a
 \a
 \a
\xe

 \ex
Rhetoric:
{}\\
\xe

 \ex
Composition:
{}\\
\xe

 \ex
Claim:
{}\\
\xe

 \ex
Fallacy:
{}\\
\xe

 \ex
Audience:
{}\\
\xe

 \ex
Belief:
{}\\
\xe

 \pex This is extra credit
\a Knowledge:
 \xe
\end{examProblem}

\begin{examProblem}
\texttt{Directions: Choose one text excerpt and answer the question (use the next page to write your analysis). Your answer should be in the form of an analysis or argument. Answers will be evaluated by how you do your analysis or argument, not whether you are right or wrong. I am looking for justification, support, evidence, and the reasoning you employ in making your analysis or argument. Evidence should come straight from the text itself, not an outside, unverifiable, source.}

\ex \textbf{Which student is {\sl less/more} personal or subjective and why?}
 \begin{quote}
\textbf{Student A:} I found that scientists engage in research in order to make discoveries and generate new ideas. Such research by scientists is hard work and often involves collaboration with other scientists which leads to discoveries which make the scientists famous. Such collaboration may be informal, such as when they share new ideas over lunch, or formal, such as when they are co-authors of a paper.\\

\textbf{Student B:} It was hard work to research famous scientists engaged in collaboration and I made many informal discoveries. My research showed that scientists engaged in collaboration with other scientists are co-authors of at least one paper containing their new ideas. Some scientists make formal discoveries and have new ideas.
 \end{quote}
\xe

\ex {\bf What is wrong with this excerpt?}
\begin{quote}
Why do computer scientists need to know algebra? Clearly, algebra and computer science have deep connections. This theory that algebraic methods can be implicated as software programs has been around a long time. The data shows that it is an effecient way to prove theorems will take a long time. Therefore, software design is a type of algebra. Studies have been done to see if algebraic programming can solve polynomial time problems, but they have not been effective.
\end{quote}
\xe

\newpage
\problemLAnswer{{\bf Write your analysis in the box below.}\hfill Score: \quad \quad  /1.5\\
Notes:       \quad \\
{}\\
              \quad\\
{}\\
               \quad\\ }
\problemAnswer{}
Do not write below here. \hfill Do not write outside the box.
\end{examProblem}
\end{document}

%%%%%%%%%%%%%%%%%%%%%%%%%%%%%%%%%%%%%%%%%%%%%%%%%%%%%%%%%%%%%
