%HANDOUT

\documentclass[11pt]{article}


\usepackage{natbib,graphicx,times,linguex,cgloss4e,latexsym}
\usepackage{qtree}
\usepackage{amsmath}
\usepackage{amsthm}
\usepackage{amsfonts}

\begin{document}
\author{Joshua Bowles}
\title{Statement of Purpose: Santa Cruz}
\date{\today}


\maketitle

\section{Research}
I am generally interested in {\sc Binding, Quantification,} and {\sc Phrase Structure} in theoretical syntax and semantics. I enjoy finding ways to use definitions and ideas from mathematical logic, computability theory, and abstract algebra to build rigorous models for explaining natural language data. 

\subsection{Recent Projects}

\ex. binding conditions for resumption in Irish

\ex. noun-classifiers and quantification in Mandarin and selected South American languages

\ex. quantification in Pirah\~a is `classificatory' (based on noun-classifiers)

\ex. evidentials and functional hierarchy in Pirah\~a show recursion, configurationality\label{recursion}

\ex. syntax-semantics interface of evidentials and interacting tense/aspect/mood
\ex. algebraic group-theory definition for Merge (and biolinguistic considerations)\label{group}

I am in the process of writing numerous manuscripts that deal with the above areas; \ref{group} has been submitted for publication and a combination of \ref{recursion} and \ref{group} has been submitted as an abstract to WCCFL 2010. 

While I cannot say exactly what my dissertation will be on, I think the list above gives a clear sketch of my specific interests.

\subsection{Languages}
For the last year I have been (i) learning modern Irish on my own and (ii) auditing Mandarin classes at the University where I teach (as an adjunct instructor in the English Department). I have also studied Spanish, Portuguese, German, and Old English. My MA thesis was on agreement and verbal morphosyntax of the Amazonian language Tuyuca (Tukanoan family), which I read at a beginner level. I have done limited field work (3 months) on the related Tukanoan language Desano. I intend to continue my Irish and Mandarin studies, investigating issues of binding and quantification in both these languages, and of course, my native English. I will also use my typological background in American Indian languages for novel and comparative data when and if appropriate.  

\subsection{External Funding}
I have applied for a National Endowment of Humanities Summer Stipend to study the syntax and semantics of evidentiality in Tuyuca and English (as well as some non-linguistic issues in the use of academic rhetoric and informal logic in composition classes). I will also apply (under a special appeal) to the National Science Foundation Graduate Research Fellowship Program. I applied once before to the NSF-GRFP for anthropological linguistics at the University of Utah under Lyle Campbell and was not awarded. This time, my application will be for theoretical linguistics (the basis for the appeal is the substantial change of focus in study). I am always looking for sources of external funding and fully intend to apply for anything relevant to me or my research. 

\end{document}  