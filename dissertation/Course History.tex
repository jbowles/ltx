%Template <LingArticle-linguex> for mathematical or philosophical papers.

%Runs hyperref.style, linguex.style, and a bit of ams.xx stuff.
\documentclass[11pt,twosided]{article}

\usepackage{natbib,graphicx,times,linguex,cgloss4e,latexsym}
\usepackage{qtree}
\usepackage[normalem]{ulem}
% Allow \wh-word, \wh-phrases, \wh-movement, etc., to italicize the "wh"----Paul Hagstrom------
\newcommand{\wh}{\emph{wh}}
\newcommand{\Wh}{\emph{Wh}}
% Some X-bar stuff from gb4e==================
\newcommand{\oxbar}{{\sc X$^{\circ}$}}
\newcommand{\ixbar}{{\sc X$'$}}
\newcommand{\maxbar}{{\sc X$^{max}$}}
%=============
% From sp.cl and Paul Hagstrom***********************************************
\usepackage{natbib}
\bibpunct[: ]{(}{)}{;}{a}{}{,}
\newcommand{\BIBand}{\&}
\setlength{\bibsep}{0pt}
\setlength{\bibhang}{0.25in}
\bibliographystyle{sp}%-----------------------sp.bst------------------
\newcommand{\posscitet}[1]{\citeauthor{#1}'s (\citeyear{#1})}
\newcommand{\possciteauthor}[1]{\citeauthor{#1}'s}
\newcommand{\pgposscitet}[2]{\citeauthor{#1}'s (\citeyear{#1}:~#2)}
\newcommand{\secposscitet}[2]{\citeauthor{#1}'s (\citeyear{#1}:~$\S$#2)}
\newcommand{\pgcitealt}[2]{\citealt{#1}:~#2}
\newcommand{\seccitealt}[2]{\citealt{#1}:~$\S$#2}
\newcommand{\pgcitep}[2]{(\citealt{#1}:~#2)}
\newcommand{\seccitep}[2]{(\citealt{#1}:~$\S$#2)}
\newcommand{\pgcitet}[2]{\citeauthor{#1} (\citeyear{#1}:~#2)}
\newcommand{\seccitet}[2]{\citeauthor{#1} (\citeyear{#1}:~$\S$#2)}
%***********************************************************sp.bst****************
\usepackage{amsmath,amsthm,amsfonts}
\theoremstyle{plain}
\newtheorem{thesis}{Thesis}
\newtheorem{axiom}{Axiom}
\newtheorem{claim}{Claim}
\newtheorem{conditon}{Condition}
\newtheorem{conjecture}{Conjecture}
\newtheorem{corollary}{Corollary}
\newtheorem{definition}{Definition}
\newtheorem{proposition}{Proposition}
\newtheorem{remark}{Remark}
\theoremstyle{definition}
\newtheorem{phrase string}{Phrase String}
\newtheorem{notation}{Notation}
% Paper/text size option:
%\paperwidth=8.5in \textwidth=5.5in \paperheight=11in \textheight=8in 
% From sp.cl*****************************************
\usepackage[usenames]{xcolor}
\definecolor{jblinkcolor}{rgb}{.0,.2,.4}
\usepackage[pdftex,colorlinks,breaklinks,
			linkcolor=jblinkcolor,
			citecolor=jblinkcolor,
			urlcolor=jblinkcolor,
			plainpages=false,
			bookmarks=false,
			hypertexnames=false]{hyperref}
			\urlstyle{rm}
%****************************************************

\begin{document} 
\title{\texttt{Course History}}

\author{Joshua Bowles}
\maketitle



\section{Undergraduate---selected}

\begin{tabular}{|p{12cm}|}
\hline
\textbf{Honors Thesis Title:} \textsl{Dislocated poetics: Towards a neurobiological explanation of poetic language}\\
\hline 
\end{tabular}
\newline
\newline

\noindent LINGUISTICS\footnote{My degree was a BA in English and Linguistics from Metropolitan State College of Denver. My research there was on language and consciousness, trying to apply functionalist linguistic theories and neuropsychology to a theory of literature.}
\begin{itemize}
\item Introduction to Language\\
(Stephanie Robin Quizar)
\item Semantics\\
(Stephanie Robin Quizar)
\item Language and Society\\
(Stephanie Robin Quizar)
\item Morphology and Syntax\\
(Marina Gorlach)
\item Language Acquisition\\
(Catherine Curran)
\item History of English\\
(Elizabeth Holtze)
\item Old English\\
(Jefferey Taylor)
\end{itemize}

\noindent PHILOSOPHY
\begin{itemize}
\item Philosophy of Language\\
(Drew Doxsee)
\item Philosophy of Science\\
(Antonio Chu)
\item Epistemology\\
(Antonio Chu)
\end{itemize}

\noindent LOGIC 
\begin{itemize}
\item Art of Critical Thinking (Logic)\\
(Paul Saalbach)
\item Symbolic Logic\\
(Paul Saalbach)
\item Symbolic Logic---audited\\
(Paul Saalbach) 
\end{itemize}

\noindent BIOLOGY
\begin{itemize}
\item Biology I\\
(Cynthia Chruch)
\item Biology Lab\\
(Cynthia Church)
\end{itemize}

\section{Graduate courses at University of Utah---MA}

\begin{tabular}{|p{12cm}|} 
\hline
\textbf{Thesis Title:} \textsl{Agreement in Tuyuca}\\
Committee: Lyle Campbell (Chair), Edward J. Rubin, Mauricio Mixco\\
\hline
\end{tabular}
\newline
\newline
 
 
\noindent ACADEMIC YEAR 2008-2009\footnote {During my time at Utah I focused on typology and language documentation with Lyle Campbell. I eventually turned to theoretical syntax.}
\begin{itemize}
\item LING 6970 Thesis Research\\
(Lyle Campbell; Fa08)
\item LING 6991 Individual Research\\
(Lyle Campbell; Fa08)
\end{itemize}

\noindent ACADEMIC YEAR 2007-2008
\begin{itemize}
\item LING 6050 Typology and Universals\\
(Lyle Campbell; Spr08)
\item LING 6080 Seminar: Morphology\\
(Aniko Csirmaz; Spr08)
\item LING 6200 Structure of Portuguese\\
(Mauricio Mixco; Spr08)
\item LING 6970 Thesis Research\\
(Lyle Campbell; Spr08)
\item LING 6012 Advanced Phonology\\
(Rachel Hayes-Harb; Fa07)
\item LING 6022 Advanced Syntax\\
(Aniko Csirmaz; Fa07)
\item LING 6080 Seminar: Language Documentation\\
(Lyle Campbell; Fa07)
\item LING 6070 Thesis Research\\
(Lyle Campbell; Fa07)
\end{itemize}


ACADEMIC YEAR 2006-2007
\begin{itemize}
\item LING 6011 Intermediate Phonology\\
(Randall Gess; Spr07)
\item LING 6021 Intermediate Syntax\\
(Edward J. Rubin; Spr07)
\item LING 6130 Historical Linguistics\\
(Lyle Campbell; Spr07)
\item LING 6010 Phonology\\
(Randall Eggert; Fa06)
\item LING 6020 American Indian Languages\\
(Lyle Campbell; Fa06)
\item LING 6030 Semantics\\
(Patricia Hannah; Fa06)
\end{itemize}








\end{document}

