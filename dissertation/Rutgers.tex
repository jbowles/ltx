%HANDOUT

\documentclass[11pt]{article}


\usepackage{natbib,graphicx,times,linguex,cgloss4e,latexsym}

\usepackage{amsmath}
\usepackage{amsthm}
\usepackage{amsfonts}

\begin{document}
\author{Joshua Bowles}
\title{Statement of Purpose: Northwestern}
\date{\today}


\maketitle
\section{Personal}
I am looking to study formal syntax and computational semantics-pragmatics. I have a good base in descriptive, typological-functional theories from study under Stephanie Robin Quizar and Lyle Campbell, as well as in minimalist syntax from Aniko Csirmaz and Ed Rubin. Now I want to study more formal techniques. One of my main interests is Evidentiality. This came from working with Tukanoan languages, and presently I am looking into how languages without grammatical evidentials may possibly encode evidential (or epistemic modal) interpretations. I am also interested in building ontologies and formal models of context for evidential reasoning, presupposition, and implicature. 

\section{Research}
I am interested in applying various linguistic theories to solve real-world problems in scientific and industry related domains. I enjoy finding ways to use ideas from mathematical logic, computability theory, and abstract algebra to build rigorous models for natural languages. Presently I am focusing my energy on basic skills needed for computational linguists. This includes (i) basic command line usage (having recently switched to a Linux environment), (ii) natural language processing (including languages such as Python, Lisp, Prolog), and (iii) statistics and probablility. These are life-long goals and I have no pretention to mastery in a few short months.   

The following subsections list current and future research projects I would like to continue in my doctoral studies. 

\subsection{Current Projects: 2008--2009}
\ex. \textsl{English Corpus of Evidentiality Data} (goal of 1 million sentences/clauses; presently at about .02\% of goal).

\ex. \textsl{Evidentiality and the Human Mind: Language, Information, and Human Discourse} (\texttt{Project under review for funding by the NEH Summer Stipend 2010}). 

\ex. \textsl{Some Syntax, Semantics, and Pragmatics of Evidentiality in English}. Ms., 29 pgs: version 1.3. 

\ex. \textsl{Recursion for Computation's Sake}. Ms., 24 pgs: version 1.

\ex. \textsl{Ambiguous Merge, Symmetry, and Group Theory}. Ms., 27 pgs: version 2. 
\a.  \small{Revision in progress for future squib submission. \textsl{A Note on Ideal Structure for a Hypothetical Operation}. 9 pgs: version 1}.

\subsection{Future Project Interests: Things that have been on my mind}
\ex. Typology of Evidentiality

\ex. Patterns for Pragmatic Structures in English Evidentiality. 

\ex. Formal Typology.

\ex. Ontology and Knowledge Representation for Evidential Reasoning.

I am considering studying a language with a large web presence: starting Mandarin, or continuing German or Spanish. But I am still committed to the study and preservation of less studied languages.  

\end{document}  
