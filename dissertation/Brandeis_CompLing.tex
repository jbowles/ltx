%HANDOUT

\documentclass[11pt]{article}


\usepackage{natbib,graphicx,times,linguex,cgloss4e,latexsym}
\usepackage{qtree}
\usepackage{amsmath}
\usepackage{amsthm}
\usepackage{amsfonts}
\usepackage{fancyhdr,ifpdf}
\usepackage{fancyhdr}
\lhead{\scriptsize Joshua Bowles} % force lhead all the way left
\rhead{\scriptsize Page \thepage}  % put page number at right
\chead{\scshape \scriptsize \ttfamily MA Computational Linguistics}
\cfoot{} % the footer is empty
\pagestyle{fancy}


\begin{document}
\author{\Large Joshua Bowles}
\title{Statement of Purpose\\ 
MA Computational Linguistics:\\ 
\small Department of Computer Science; Linguistics and Computational Linguistics Programs}
\date{\today}



\maketitle
\section{Personal}
I am looking to change research focus from documentary linguistics to more formal and computational techniques. This means that I probably do not have the particular educational background of most people going into computational linguistics, but my commitment and passion are as strong as anybody's. Ultimately I want my PhD in (computational) linguistics.

I was successful in previous MA studies, having produced or co-produced work that was accepted to high level conferences (see CV). However, after two years of typical linguistics training---including specializations in the morphology and syntax of native American languages in both a typological and Minimalist (generative) setting---I realized that I had become increasingly interested in more rigorous approaches. Capitalizing on my undergraduate training in mathematical logic I started studying modern algebra. I acquired an internship as a linguist working on computational linguistics and liked the work (see CV). Presently I am studying the core mathematics and programming needed for computational linguistics tasks and have begun building my own corpus and experimenting with it. (I have applied for a National Endowment of Humanities Summer Stipend 2010 to work full time on the project.) 

At this point, I need someone to take a chance on me and provide me the training I need. I do not have the appropriate educational background in computer science or mathematics to apply to typical natural language processing or AI programs. 

\subsection{GRE Scores}
I realize that my GRE scores are not good and that this does not reflect well on my merit as a student. And if you measure the merit of an active mind by a static remainder of standardized expectations then I will not measure up. However, putting aside the rhetoric, I assure you that the GRE is not an accurate reflection of my potential and that my true worth is meted out in the daily grind of working on techniques, ideas, data, and theories: this cannot be tested. I can also assure you that although I, apparently, do not test well, my commitment to acquiring the mathematical and computational background necessary for being successful in the field of natural language investigation is solid.    

\section{Research}
I am broadly interested in semantics and pragmatics---tense, modals, evidentials, idioms. I have recently become interested in: the ambiguity of idiomatic speech, relation between polysemy and idioms, theory and formalization of context, and probability models for discourse. My MA thesis was on evidential morphosyntax in Tuyuca (Tukanoan family).


Manuscripts or projects that I have worked on in the past year, \textbf{but put off indefinitely due to present concerns with mathematics and programming}, include: (i) looking into how one might construct a formal model of context based on symmetry relations of reference between information and information-source using basic axioms from Group theory (i.e., direct information = an Abelian group and then applying a measure of entropy to that and working out to indirect information); (ii) biolinguistic ideas about mathematical series and tree-structures---specifically, I provided a complementary model to David Medieros' idealized X-bar structure that he based on the Fibonacci sequence; (iii) consolidating formal-generative models of evidentiality into a coherent model that correlates with proposed cartography structures of the left-periphery; (iv) developing a way of computing ambiguity of idiomatic phrases based on NER and normalization---this is a spin-off of work I did as a linguist intern at Attensity Corporation (no manuscript available); (v) building the English Corpus of Evidentiality Data, which is a corpus of English lexical items (adverbials, modals, verbs of `sense') and specific clause structures presumed to yield evidential interpretations (no manuscript available). 

Needless to say, these projects have been put off indefinitely, but they should give a sense of the research activity I have been engaged in lately (I am more than willing to provide manuscripts upon request).

\end{document}  