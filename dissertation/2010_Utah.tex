\documentclass{article}
\usepackage{paperp}
\usepackage{paperc}

\begin{document}
 \title{PhD 2010}
\author{Joshua Bowles}
\date{\today}

\maketitle

\section{Research}
My research is in syntax, semantics, and pragmatics. Specifically, I want to focus on `context' in human language; this includes tense, modality, evidentials, illocutionary force, deixis, and the nature of adverbs; as well as other phenomena related to space and time in natural language (i.e., topics usually dealt with in situation and event semantics, dynamic/udpate semantics, epistemic logics, modal logics, tense logics, left-periphery cartography, functional categories, {\sl et cetera}). My interest in `context' derives from work I did on evidentiality in my master's thesis. %page 85{}

My research interests have also grown out of work I have been doing since recieving my degree. Such work includes, generally, an internship as an industry linguist at a computational linguistics software company and teaching udergraduate courses in science composition with a foucs on reasoning and logic. I have long-term interests in Artificial Intelligence (natural language processing, knowledge representation, and evidential reasoning) and am very interested in building computationally realistic models from current linguistic theorizing in the Minimalist Program (and to some extent Optimality approaches).

Any detailed research focus I decide on for my dissertation would emerge through collaboration and interaction with faculty and peers, but I would hope to highlight, complement, or extend current faculty research/interests. 

I  am dedicated to employing various methods of data collection and analysis as a means to testing hypothetical claims in theoretical linguistics. I am not committed to one theory or the other, but am committed to finding solutions to both real-world problems and theoretical problems. Currently, I have been acquiring proficiency in a number of programming languages and applications; some designed to deal with large corpora of language data (e.g., Python, Ruby, R) while others are on the experimental edge (e.g., the hybrid probalistic/rule-based Church, which specializes in building generative models). 

I have found work in computational linguistics interesting and relevant to core linguistic work and I hope to continue to use and develop my new skills within the broader goals of linguistic inquiry. For example, adverbial word order patterns hypothesized to display a universal functioanl heirarchy are amenable to large-scale corpora searches for such patterns. The use of computational techniques in dealing with the search for linguistic patterns in very large data sets (e.g., greater than 10 million words) could be very productive in supporting general theoretical claims. Another example employs techniques from algebraic set-theory with biolingusitic work on Fibonacci sequences and iterated patterns with symmetrical relations with goal of producing clearly defined algorithms capable of deriving differnet models of phrase strucutre (and phase sturcutures).
 

\section{Personal}
I received my MA in (general/theoretical) linguistics from the University of Utah in 2008. During my two years in the Master's program I was able to engage in collaborative research with other students that culminated in various conference presentations to the {\sl Berkeley Linguistics Society}, the {\sl Association of Linguistic Typology}, and the University of Utah's {\sl Conference on Endangered Languages and Cultures of Native America}. I was also able to produce high quality research on my own that resulted in conference acceptances to the Univeristy of Madison-Wisconsin's {\sl 6th Workshop in General Linguistics} and the Canadian based {\sl 13 Workshop on Structure and Constituency in the Languages of the Americas} (I was not able to attend the latter due to lack of funding).

I want to stay in the state of Utah for personal reasons and I would like to pursue advanced research and study at the University of Utah. I feel that some of my research interests and experiences are a good fit for the department. I am also interested in trying to foster greater research connections with other departments in order to carry out interdepartmental and interdisciplinary work (specifically the natural language processing group in the School of Computing).

\subsection{Comments}
I have not had success with the general GRE exam and feel compelled to comment on it. The mathematics on the exam is basic college level algebra and I do not use the rudimentary basic algebra that one is tested on in the first few questions of the exam. I do use mathematical logic all the time and have been actively studying calculus, statistics \& probability, and abstract modern algebra.  As far the analytical writing section goes, compare my writing sample against my GRE score and make your own judgement about whether or not the GRE accurately reflects my ability. 


While my comments here may not be strong enough to counter any results from the GRE, consider that I have shown `enough' aptitude in logical, abstract, and numerical reasoning to pass assessments for the entry level computational programming job I now have (the computer science I learned on my own, I have no formal education in that field). I should also comment on my poor performance (I got a B) in the Semantics course I took in Fall 2006---as Semantics is one of my interests and you may find it impractical that a student pursue research in a domain they have performed poorly in: My first child was born the week of midterms Fall 2006, I worked a part-time job for 30 hours a week, and I was preparing a co-written abstract which was subsequently accepted to the BLS that Fall. My performance in that class (and that semester) reflected non-typical external conditions and should not be taken as a characterization of my actual ability.


\section{Languages}
My language interests are driven by my interest in computational techniques for dealing with large sets of language data: (i) languages with a large web-presence (i.e., English, Mandarin, Spanish, German); (ii) endangered languages that both need and define standard methods and practices for digital archiving. The latter type of language, for me, is Tuyuca. My master's thesis included an appendix of Tuyuca sentences that I had collected from various academic sources; I am still committed to collecting all the available Tuyuca sources into a digital format for preservation in a standard archive. I also have very long-term personal projects that include developing a pedagogical `chatbot' (an interactive computer program that one can ``have conversations'' with) and a spell-chekcer for either Paraguayan Guarani or Irish Gaelic. 
I do not yet know which of the langauges I will focus on for doctoral work.  

\end{document}

