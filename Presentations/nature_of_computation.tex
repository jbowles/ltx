% $Header: /cvsroot/latex-beamer/latex-beamer/solutions/generic-talks/generic-ornate-15min-45min.en.tex,v 1.5 2007/01/28 20:48:23 tantau Exp $

\documentclass{beamer}

% This file is a solution template for:

% - Giving a talk on some subject.
% - The talk is between 15min and 45min long.
% - Style is ornate.



% Copyright 2004 by Till Tantau <tantau@users.sourceforge.net>.
%
% In principle, this file can be redistributed and/or modified under
% the terms of the GNU Public License, version 2.
%
% However, this file is supposed to be a template to be modified
% for your own needs. For this reason, if you use this file as a
% template and not specifically distribute it as part of a another
% package/program, I grant the extra permission to freely copy and
% modify this file as you see fit and even to delete this copyright
% notice. 


\mode<presentation>
{
  \usetheme{Warsaw}
  % or ...

  \setbeamercovered{transparent}
  % or whatever (possibly just delete it)
}


\usepackage[english]{babel}
% or whatever

\usepackage[latin1]{inputenc}
% or whatever

\usepackage{times}
\usepackage[T1]{fontenc}
% Or whatever. Note that the encoding and the font should match. If T1
% does not look nice, try deleting the line with the fontenc.


\title[Supervised Learning] % (optional, use only with long paper titles)
{Machine Learning}

\subtitle
{Classification?} % (optional)

\author[josh bowles, Another] % (optional, use only with lots of authors)
{josh bowles\inst{1} \and S.~Another\inst{2}}
% - Use the \inst{?} command only if the authors have different
%   affiliation.

\institute[Universities of Somewhere and Elsewhere] % (optional, but mostly needed)
{
  \inst{1}%
  Department of Experimental Philosophy\\
  University of Somewhere
  \and
  \inst{2}%
  Department of Theoretical Philosophy\\
  University of Elsewhere}
% - Use the \inst command only if there are several affiliations.
% - Keep it simple, no one is interested in your street address.

\date[\Today] % (optional)
{Date / Bayes Group}

\subject{Reverend Bayes}
% This is only inserted into the PDF information catalog. Can be left
% out. 



% If you have a file called "university-logo-filename.xxx", where xxx
% is a graphic format that can be processed by latex or pdflatex,
% resp., then you can add a logo as follows:

\pgfdeclareimage[height=0.9cm]{university-logo}{code-merge}
\logo{\pgfuseimage{university-logo}}



% Delete this, if you do not want the table of contents to pop up at
% the beginning of each subsection:
\AtBeginSubsection[]
{
  \begin{frame}<beamer>{Outline}
    \tableofcontents[currentsection,currentsubsection]
  \end{frame}
}


% If you wish to uncover everything in a step-wise fashion, uncomment
% the following command: 

\beamerdefaultoverlayspecification{<+->}



\AtBeginSubsection[]
{
 \begin{frame}<beamer>{Outline}
   \tableofcontents[currentsection,currentsubsection]
  \end{frame}
}

\begin{document}

\begin{frame}
  \titlepage
\end{frame}

\begin{frame}{Outline}
  \tableofcontents
  %% You might wish to add the option [pausesections]
\end{frame}

\section{What is Supervised Learning?}

\begin{frame}{Target Approximation}
  Less Science-Fiction name would be more appropriate: Target Approximation
  \begin{enumerate} According to Ai taxonomy
\item Fully observalbe
\item Stochastic
\item Continuous
\item 
  \end{enumerate}
Machine learning is really just a group of algorithms that all have similar end-game strategies: to make predictions given new data based on training data that specified a target to be optimally approximated.
\end{frame}

\begin{frame}{More definitions}
	\begin{block}
		Supervised learning algorithms assume that some variable X is designated as the target for prediction, explanation, or inference, and that the values of X in the dataset constitute the “ground truth” values for learning. That is, supervised learning algorithms use the known values of X to determine what should be "learned"
	\end{block}
  
\end{frame} 

\subsection{Is the Universe a Computer?}
\begin{frame}{Universe a computer}\label{universe}
If the Universe is a Computer then so are we. And the computers we use are artifiacts of natural phenomena
\begin{enumerate}
\item Thinking and consciousness are some kind of computational procedures amenable to formalization
\item Programming is the act of translating naturally effective procedures into artifacts implementable through other artifacts (computers)
\item Implemtation or Procedure
\item Are we philosophers or engineers?
\end{enumerate}
\end{frame}


\subsection{Definition}
\begin{frame}{Computation defined (informal)}
\begin{exampleblock}{\sc Church-Turing Thesis}
Every function which would naturally be regarded as computable can be computed by the universal Turing machine.
\end{exampleblock}

\begin{block}{Church-Turing Principle}
  Every finitely realizable physical system can be perfectly simulated by a universal model computing machine operating by finite means.
\end{block}
 \end{frame}\label{informal}


\section{What is Knowledge?}
\subsection{Formal}
\begin{frame}{The subjective status of logic}
  \textbf{Math and Logic and Language} are formal calculi (system for calculation) with a syntax (rules for combination) and semantics (rules for interpretation/representation) <++>
\\
\pause
{\bf BUT}
There's no system for deciding between two procedures or how to build an aggregation of procedures; 
\end{frame}

\subsection{Informal}
\begin{frame}{The formal status of creativity}
  If the universe (and our brains) are really computers that implement effective procedures of some kind, then creativity has some systematically formalizable model.
\end{frame}

\subsection{What do we get?}
\begin{frame}{The takeaway}
  By considering the natur of computation and its place in the physical universe, we force ourselves to think about the limits of what we can do with our programming.\\
  {\bf If we understand mental processes and how they relate to the nature oc computation we gain an advantage in formalizing these processes and automating them.}<++>
\end{frame}
% Since this a solution template for a generic talk, very little can
% be said about how it should be structured. However, the talk length
% of between 15min and 45min and the theme suggest that you stick to
% the following rules:  

% - Exactly two or three sections (other than the summary).
% - At *most* three subsections per section.
% - Talk about 30s to 2min per frame. So there should be between about
%   15 and 30 frames, all told.


\end{document}



